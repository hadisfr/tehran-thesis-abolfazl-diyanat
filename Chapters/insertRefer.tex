\chapter{وارد کردن مراجع}
برای وارد کردن مراجع در پایان‌نامه بهترین روش استفاده از \lr{bibtex} است. 


برای مثال مرجع {\cite{Mackey2005_Effects}} در مورد شبکه .... . و این هم مرجع دوم {\cite{Kappler2009}} 

و سپس مرجع سوم {\cite{Kappler2009}}

برای آوردن مراجع باید مراحل زیر را انجام دهید.
\begin{itemize}
\begin{LTRitems}
\item
\verb+ xelatex -interaction=nonstopmode -synctex=-1 %.tex+
\item
\verb+ bibtex  % +
\item
\verb+ xelatex -interaction=nonstopmode -synctex=-1 %.tex+
\item
\verb+ xelatex -interaction=nonstopmode -synctex=-1 %.tex+
\end{LTRitems}
\end{itemize}
اگر از ویرایشگر {\lr{Texmaker}} استفاده می‌کنید، دستور اولی، سومی و چهارمی همان {\lr{Quick Build}} است. یعنی اگر دکمه {\lr{Quick Build}} را بزنید، انگار دستور مورد اشاره را اجرا کرده اید. در مورد دستور دوم، در {\lr{Texmaker}} همان دستور {\lr{bibtex}} است. در اکثر ویرایشگرها چنین چیزی وجود دارد.  

% انواع دیگر استایل ها در راهنمای persian-bib آمده است. 




