\chapter{وارد کردن کد در متن}
برای وارد کردن کدهای برنامه نویسی خود در محیط لاتک، بسته \lr{listings} یکی از بهترین بسته های موجود است. برای استفاده از این بسته فقط به نکات زیر دقت کنید:
\begin{itemize}
\hand
در شروع امر بسته \lr{listings} را  با دستور \lr{usepackage} فراخوانی کنید. دقت کنید که این بسته را با بسته \lr{listing} اشتباه نکنید.
\hand
در مرحله بعدی می توانید توسط دستور \lr{lstset} هرجایی از متن که خواستید تنظیمات این بسته را تغییر دهید. 
\hand
در هنگام استفاده از این بسته فقط دقت داشته باشید که محیط آن باید بین محیط \lr{latin}‌قرار گیرد. 
\hand
دو راهنمایی خوب برای این بسته یکی سایت \lr{http://en.wikibooks.org/wiki/LaTeX/Packages/Listings}‌ ودیگری راهنمای این بسته است. 
\hand
برای فهم بهتر این مثال بهتر است که مثال را از فایل \lr{tex} دنبال کنید نه از فایل \lr{pdf} چراکه بسیاری از توضیحات به صورت \lr{comment} در فایل \lr{tex} داده شده است. 
\end{itemize}


مثالی از نوشتن کد مطلب درون یک نوشتار:

\begin{latin}
\lstinputlisting[language=Matlab]{Code/code3.m}
\end{latin}

در این مثال یک کد \lr{MATLAB} دیگر وارد می کنیم، با این تفاوت که می خواهیم یکسری از کلمات کلیدی را مشخص کنیم که لاتک آن ها را با رنگی به خصوصی نشان دهد. 
\begin{latin}
\lstset{emph={binornd},emphstyle=\color{Magenta}}
\lstinputlisting[language=Matlab, morekeywords={ksdensity}]{Code/prog3.m}
\end{latin}

مثالی دیگر از نوشتن کد مطلب در یک نوشتار. فقط در این حالت می خواهیم برخی از تنظیمات پیش فرض را که قبل از شروع نوشتار تعیین کرده ایم، تغییر دهیم. 
\begin{latin}
\lstinputlisting[numbers=right,language=Matlab, framexleftmargin=5mm, frame=shadowbox,rulesepcolor=\color{Yellow}]{Code/code4.m}
\end{latin}

مثالی از نوشتن یک کد {\lr{JAVA}} درون یک نوشتار:

\definecolor{codeColor}{rgb}{0.9,0.9,0.9}
\begin{latin}
\lstset{emph={pMax,pMin,transP,waitingUser,waitQueue},emphstyle=\color{red},backgroundcolor=\color{codeColor},lineskip=.2cm}
\lstinputlisting[language=Java]{Code/threadQueue.java}
\end{latin}

در ضمن شما می توانید حتی در خود همین نوشتار اصلی خود کد مورد نظرتان را بنویسید. 
\begin{latin}
\lstset{frameshape={RYRYNYYYY}{yny}{yny}{RYRYNYYYY}}
\begin{lstlisting}
for i:=maxint to 0 do
begin
{ do nothing }
end;
\end{lstlisting}
\end{latin}



