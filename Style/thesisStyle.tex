%% Data: 2013/09/28    Time: 17:59:07 
%% در مورد تقدم و تاخر وارد کردن بسته ها تنها باید به چند نکته دقت کرد:
%% الف) بسته xepersian حتما حتما باید آخرین بسته ای باشد که فراخوانی می شود
%% ب) بسته hyperref جزو آخرین بسته هایی باید باشد که فراخوانی می شود.
%% ج) بسته glossaries حتما باید بعد از hyperref فراخوانی شود. 
%% د) بسته listings باید حتما قبل از  hyperref فراخوانی شود. 

%%%  این سه بسته برای آوردن استایل‌های فصل‌های memoir مورد نیاز است. 
\usepackage[scaled=.92]{helvet}
\usepackage{calc}
\usepackage{framed}

%%% تمام بسته های مورد نیاز برای ایجاد یک پایان نامه به صورت کامل اینجا آورده شده است در صورتی که بخواهید از بسته های دیگر استفاده کنید بهتر است که آن‌ها را به گونه ای انتخاب کنید که با این بسته ها تداخل نداشته باشد. به نظر من استفاده از همین بسته ها کافی است.
\usepackage{amsthm,amssymb,amsmath}
\usepackage{thmtools}
\usepackage{dsfont}
\usepackage{etoolbox}
\usepackage{wasysym}

%%% بسته‌ای برای فعال‌سازی پارامتر H در وارد کردن شکل
\usepackage{float}

%%% بسته‌ای برای  تولید نمایه در متن
\usepackage{makeidx}
\makeindex
%%% برای تنظیم حاشیه صفحات

\usepackage{geometry}

%%% بسته‌ای برای تنظیم حاشیه و کادر دور فرمول و ... 
\usepackage{empheq,fancybox}

%%% برای رنگی کردن متن و استفاده از رنگ در متن این دو بسته مورد نیاز است.
\usepackage[usenames,dvipsnames]{color,xcolor}

%%% بسته ای برای وارد کردن Watermarking
\usepackage{draftwatermark}

%%% بسته ای برای استفاده از اشکال برای آیتم‌ها
\usepackage{pifont}

%%% بسته‌ای برای رنگ آمیزی جداول
\usepackage{colortbl}

%%% بسته ای برای این که در جدول یک متن را در چند سطر بیاوریم. 
\usepackage{multirow}

%%% بسته‌ای برای رسم اشکال و تصاویر با Latex
\usepackage{tikz,times}
\usepackage{pstricks}
\usepackage{tikz-qtree}
%% برای ترسیم گانت چارت در گزارش و پیشنهاد پروژه
\usepackage{pgfgantt}
%% رسم یکسری تصاویر با Latex به مانند عکس یک پرنده
\usepackage{eso-pic}
\usepackage{pst-fun}

%%% بسته ای برای وارد کردن کدهای برنامه نویسی (MATLAB، JAVA و ...) در متن. بسته listings باید قبل از hyperref باشد و گرنه با خطا مواجه خواهیم شد.
\usepackage{listings}

%%% بسته‌ای برای رفع خطای کمبود فضا در Latex و خطای morewrite
\usepackage{morewrites}

%%% بسته ای برای وارد کردن الگوریتم در متن
%\usepackage{algorithm}
%\usepackage{algorithmicx}
%\usepackage{algpseudocode}

%%% در این قالب از بسته graphx برای انجام کارهای گرافیکی استفاده می‌شود. این بسته برای اضافه کردن تصویرها به متن استفاده شده است.
\usepackage{graphicx}

%%% بسته‌ای است که توسط آن می‌توان شماره صفحه و آخرین صفحه را استخراج نمود. 
\usepackage{lastpage}

%%% بسته‌ای برای وارد کردن صفحات pdf در متن
\usepackage{pdfpages}

%%% دوبسته برای اضافه کردن دستورات if و else به برنامه.
\usepackage{xparse}
\usepackage{ifthen}

%%% بسته‌ای برای تنظیم فرمت استایل بخش‌ها، زیربخش‌ها و ... . 
\usepackage{titlesec}

%%% بسته ای برای رنگی کردن لینک ها و فعال سازی لینک ها در یک نوشتار، بسته hyperref باید جزو آخرین بسته‌هایی باشد که فراخوانی می‌شود. 
\usepackage{hyperref}

%%% بسته‌ای برای وارد کردن واژه نامه در متن، این بسته باید بعد از hyperref حتما صدا زده شود. 
\usepackage[sanitize={name=false,description=false,sort=false,symbol=false},nomain,xindy,acronym,acronymlists={main,tem}]{glossaries}

%%%زی‌پرشین (به انگلیسی: XePersian) یک بسته حروف‌چینی رایگان و متن‌باز برای نگارش مستندات پارسی/انگلیسی با زی‌لاتک است.
%%% در واقع، زی‌پرشین، کمک می‌کند تا به آسانی، مستندات را به پارسی، حروف‌چینی کرد. این بسته را وفا خلیقی نوشته است،
%%% و به طور منظم، آن را بروز‌رسانی کرده و باگ‌های آن را رفع می‌کند.
%%% نکته مهم این جا است که بسته Xepersian برای پشتیبانی از زبان فارسی آورده شده است، و 
%%% می بایست آخرین بسته ای باشد که شما وارد می کنید، دقت کنید: آخرین بسته 
\usepackage[localise=on]{xepersian}

%%% OOOOOOOOOOOOOOOOOOOOOOOOOOOOOOOOOOOOOOOOOOOOOOOOOOOOOOOOOOOOOOOOOOOOOO

%% تعریف برخی محیط‌ها 
\newenvironment{problem}{}{}
\newenvironment{info}{}{}
\newenvironment{refer}{}{}
\newenvironment{warning}{}{}
\newenvironment{goal}{}{}
\newenvironment{note}{}{}
\newenvironment{mydef}{}{}
\newenvironment{myshadowbox}{}{}
\newenvironment{colorBox}{}{}
\newenvironment{lemmaproof}{}{}
\newenvironment{mycomment}{}{}


% تعریف برخی دستورات 
\newcommand{\goodRef}{}
\newcommand{\probsec}{}
\newcommand{\idx}{}
\newcommand{\arcm}{}
\newcommand{\arcmO}{}
\newcommand{\tick}{}
\newcommand{\tickO}{}
\newcommand{\X}{}
\newcommand{\XO}{}
\newcommand{\hand}{}
\newcommand{\handO}{}
\newcommand{\tree}{}
\newcommand{\treeO}{}
\newcommand{\two}{}
\newcommand{\twoO}{}
\newcommand{\sci}{}
\newcommand{\sciO}{}
\newcommand{\starE}{}
\newcommand{\starEO}{}
\newcommand{\music}{}
\newcommand{\musicO}{}
\newcommand{\gol}{}
\newcommand{\golO}{}

%%% تنظیم فاصله خطوط در متن اصلی
\newlength{\baselineskipVar}
%%% تنظیم فاصله خطوط در فهرست‌ها
\newlength{\listlineskipVar}
%%% تعریف فونت‌‌های پیش فرض برای متن
\newcommand{\defaultFont}{}

%% این متغیر برای استایل presentation مورد استفاده قرار می‌گیرد. در این استایل در صفحات اول سربرگ قرار داده نمی‌شود. این شمانده تعیین می‌کند که در چه صفحه‌ای سربرگ قرار داده شود. 
\newcounter{setfirstpage}

%% OOOOOOOOOOOOOOOOOOOOOOOOOOOOOOOOOOOOOOOOOOOOOOOOOOOOOOOOOOOOOOOOOOOOOO
%%% تعریف یکسری دستور برای صفحه عنوان 

%% توسط دستور \myData می توانید تاریخ و ساعت را وارد متن خود کنید. 
\newcommand{\myData}{
\شمارجدید\ساعت
\شمارجدید\دقیقه
\تر\زمان‌به‌وقت‌امروز{%
\ساعت \زمان \تقسیم \ساعت 60  ساعت \محتوای\ساعت {}
\دقیقه \زمان \ضرب \ساعت 60 \بیفزابر \دقیقه -\ساعت
 \گرعدد\دقیقه=0\گرنه و \محتوای\دقیقه{} دقیقه\رگ }
 \امروز{} در  \زمان‌به‌وقت‌امروز{} 
} %M

%%% تعریف یکسری متغیر برای صفحه عنوان مطالب
\makeatletter
\gdef\@type{نوع پروژه}
\def\type#1{\gdef\@type{#1}}
%% عنوان محصول را تعیین می‌کند. این عنواند در ایجاد عنوان در مستند استفاده
%% می‌شود این عنوان در هر مستند باید ایجاد شود در غیر این صورت از عنوان
%% پیشفرض استفاده خواهد شد.
\gdef\@title{عنوان پروژه}
\def\title#1{\gdef\@title{#1}}
%% زیر عنوان یک متن ساده را تعیین می‌کند که یک هدف مهم محصول را تعیین می‌کند
%% این عنوان می تواند برای یک محصول در نظر گرفته نشود. از این داده برای 
%% ایجاد عنوان و سایر مکان های محصول استفاده می‌شود.
\gdef\@subtitle{کاربرد محصول برای استفاده در زیر عنوان}
\def\subtitle#1{\gdef\@subtitle{#1}}
%% افراد و گروه های که در تهیه این مستند و محصول همکاری داشته اند را تعیین
%% می کند این داده همواره باید بیان شود. این داده در نوشتن عنوان و دیگر قسمت
%% های مستند مورد استفاده قرار می‌گیرد.
\gdef\@author{افراد و گروه‌های پدید آورنده}
\def\author#1{\gdef\@author{#1}}
%% تاریخ نهایی نوشتن مستند را تعیین می‌کند این تاریخ در نوشتن عنوان استفاده
%% می‌شود این تارخ باید تعیین شود در غیر این صورت به صورت پیش فرض یک تاریخ
%% برای آن استفاده می شود.
\gdef\@date{ آخرین ویرایش: \myData}
\def\date#1{\gdef\@date{#1}}

\gdef\@supervisor{نام ناظر} 
\def\supervisor#1{\gdef\@supervisor{#1}}

\gdef\@adviser{نام استاد مشاور}
\def\adviser#1{\gdef\@adviser{#1}}

\gdef\@session{جلسه ارایه}
\def\session#1{\gdef\@session{#1}}

\gdef\@institute{پژوهشکده}
\def\institute#1{\gdef\@institute{#1}}

\gdef\@faculity{ نام دانشکده}
\def\faculity#1{\gdef\@faculity{#1}}

\gdef\@group{ نام گروه}
\def\group#1{\gdef\@group{#1}}

\gdef\@community{نام زیرگروه}
\def\community#1{\gdef\@community{#1}}

\gdef\@projectManager{پیشنهاد دهنده}
\def\projectManager#1{\gdef\@projectManager{#1}}

\gdef\@comment{}
\def\comment#1{\gdef\@comment{#1}}

\gdef\@version{اول}
\def\version#1{\gdef\@version{#1}}
% هر نسخه از یک پیشنهاد طرح ممکن است در طول زمان به روز شود. این شماره تعیین
% می‌کند که این نسخه از پیشنهاد طرح چندتا به روز رسانی شده است.
\gdef\@update{1}
\def\update#1{\gdef\@update{#1}}

\gdef\@startData{تاریخ شروع}
\def\startData#1{\gdef\@startData{#1}}

\gdef\@stopData{تاریخ اتمام}
\def\stopData#1{\gdef\@stopData{#1}}

\gdef\@executionTime{مدت زمان اجرای پروژه}
\def\executionTime#1{\gdef\@executionTime{#1}}

%%% نام فایلی لوگوی مورد استفاده در نوشتار توسط این پارامتر مشخص می‌شود. 
\gdef\@logofile{logonotfound}
\def\logofile#1{\gdef\@logofile{#1}}
%%% اندازه فایل لوگوی موجود در متن توسط این پارامتر مشخص می‌شود.
\gdef\@logoScale{.3\textwidth}
\def\logoScale#1{\gdef\@logoScale{#1}}

\gdef\@titleStyle{lshort}
\def\titleStyle#1{\gdef\@titleStyle{#1}}

\makeatother

\newcommand{\Godpage}[1]{}
\newcommand{\pejoheshTitle}[1]{}
\newcommand{\lshortTitle}[2]{}
\newcommand{\presTitle}{}

%%% OOOOOOOOOOOOOOOOOOOOOOOOOOOOOOOOOOOOOOOOOOOOOOOOOOOOOOOOOOOOOOOOOOOOOO

%%% اضافه کردن نماد استقلال به علایم ریاضی
\newcommand{\Perp}{\perp \! \! \! \perp}
\newcommand{\independent}{\protect\mathpalette{\protect\independenT}{\perp}} \def\independenT#1#2{\mathrel{\rlap{$#1#2$}\mkern2mu{#1#2}}}

%% Data: 2013/09/28    Time: 17:59:07 
%%% نام  قالب را تعیین می کند و همچنین بیان می کند که آخرین به روز رسانی  این قالب
%%% در چه زمانی بوده است. یک توصف مختصر هم از این بسته در اینجا امده است.
\ProvidesClass{Boostan-UserManual}
%%% تمام پارامترهای ورودی برای تنظیم متن را به کلاس زیر ارسال می‌کنیم.
\DeclareOption*{\PassOptionsToClass{\CurrentOption}{memoir}}
\ProcessOptions

\LoadClass[10pt,a4paper,oneside]{memoir}
%% وارد کردن یکسری دستورات ابتدایی و بسته‌های مورد نیاز به استایل.
\input{Style/Boostan-BasicStyle}

%%% 0000000000000000000000000000000000000000000000000000000000000000000000000000000000000000000000000000000000000000000000000000000000000000 
%%%==================== تنظیمات بسته geometry
%%% تنظیمات مربوط به حاشیه صفحه
\geometry{top=2.8cm, bottom=3cm, left=2.2cm, right=2.3cm}

%%% 0000000000000000000000000000000000000000000000000000000000000000000000000000000000000000000000000000000000000000000000000000000000000000 

%%==================== تنظیمات listing
%%  در این قسمت تمام ابزارهای مورد نیاز در نوشتن برنامه ها اورده شده 
%%  است. با استفاده از این ابزارهای می‌توان برنامه های مورد نیاز را در مستند جای داد.
\definecolor{listinggray}{gray}{.98}

\lstset{% general command to set parameter(s)
	% زبان برنامه نویسی که به طور پیش فرض انتخاب می شود.
	language=Java,
	% رنگ پیش فرض برای پیش زمینه
	backgroundcolor=\color{listinggray},
	%% میزان طول محیط listings را مشخص می کند، به صورت پیش فرض \textwidth است. 
	%linewidth=\textwidth ,
	%% نوع قالب دور محیط listings را تعیین می کند. 
	frameround=fttt,
	frame=trBL,
	%% is selected at the beginning of each listing. You could use \footnotesize,
	%% \small, \itshape, \ttfamily, or something like that. The last token of
	%% basic style must not read any following characters.
	basicstyle=\ttfamily, % print whole listing small
	%%   با این دستور استایل keyword ها را مشخص می کنیم. مثلا در این حالت گفته ایم که keyword ها را با رنگ آبی مشخص کند، و آن ها را bold‌کند. دقت کنید که keyword های زبان‌هایی که این بسته پشتیبانی می‌کند، 
	%% در این بسته تعریف شده است. مثلا در JAVA کلمه main به صورت پیش فرض تعریف شده است و در صورت وجود آن در کد شما آن را Latex آبی رنگ می‌کند. 
	keywordstyle=\color{blue}\bfseries,
	% underlined bold black keywords
	%identifierstyle=, % nothing happens
	%framexleftmargin=5mm, frame=shadowbox, rulesepcolor=\color{red}
	%% استایل String را در متن مشخص می کند. مثلا در این جا گفته شده است که رشته ها را با رنگ قرمز و به صورت ایتالیک نمایش بده.
	stringstyle=\ttfamily\color{red}, % typewriter type for strings
	%% نحوه استایل comment را مشخص می کند. دقت کنید که رنگ انتخاب شده نوعی رنگ سبز است، برای این که این رنگ شناخته شود می بایست دو بسته color و xcolor به صورتی که فراخوانی شده است، فراخوانی شود. 
	commentstyle=\color{LimeGreen},
	lineskip = .5pt,
	%% سه دستور بعدی نحوه نمایش شماره خطوط را مشخص می کند. 
	numberstyle=\footnotesize, 
	%% تعیین فاصله بین شماره خطوط و محیط listings
	numbersep=10pt,
	%% محل قرارگیری شماره خطوط
	numbers=left,
	%% تعیین محل قرارگیری caption محیط. بطور پیش فرض در بالای محیط است که به پایین محیط تغییر داده شده است. 
	captionpos=b, 
	%% توسط breakline می توانید خاصیت شکسته شدن خطوط بلند را در محیط listings فعال و یا غیرفعال کنید.
	%% activates or deactivates automatic line breaking of long lines.
	breaklines=true,
	%% باعث می شود که فاصله های بین رشته های نمایان شود.
	%% lets blank spaces in strings appear  or as blank spaces
	showstringspaces=true
}% 

%% البته شما می توانید این موارد پیش فرض را به ازای هر کد تغییر دهید. به عنوان مثال، ما یک کد در پوشه Code در شاخه فعلی قرار دادیم، می خواهیم آن را وارد متن کنیم، کافی است که خطوط زیر را در محل مناسبی که می خواهیم کد را قرار دهیم وارد کنیم. در این مثال یک فایل کد JAVA به نام myCode.java را می خواهیم وارد کنیم. 
%%\begin{latin}
%%\lstinputlisting[breaklines=true,numbers=left,language=Java, basicstyle=\ttfamily, numberstyle=\footnotesize, numbersep=10pt, captionpos=b, frame=single, breakatwhitespace=false]{Code/myCode.java}
%%\end{latin}

%%% 0000000000000000000000000000000000000000000000000000000000000000000000000000000000000000000000000000000000000000000000000000000000000000 
%%==================== تنظیمات hyperref

%%% برای وارد کردن کلمه (بخش) در فهرست مطالب بسته hyperref برای حالت فارسی یک مشکل دارد. بدین منظور این 
%%% مشکل را به صورت دستی حل شده است. برای این که رنگ keywordstyle که تعیین کننده رنگ کل قسمت فهرست مطالب
%%% نیز هست یکسان در آید یک پارامتر رنگ برای keywordstyle این جا تعریف می‌کنیم، و سپس از آن هم در تنظمیات hypperref 
%%% و هم در اون کدهایی که به صورت دستی وارد شده است، استفاده می‌شود. 
\newcommand{\keywordstyleColor}{blue}

\newsubfloat{figure}
%% در این قسمت تنظیمات بسته hyperref را قرار می دهیم.
%% این تنظیمات شامل موارد زیر است.
\hypersetup{
%% موقعی که فایل پی دی اف خروجی را باز می کنید صفحه به صورت عریض و بزرگ باز می شود.
	pdfmenubar=false, pdfstartview=FitH, 
	%% در قسمت مراجع شماره صفحه ای که به آن مرجع ارجاع داده است را وارد می کند،
	%% مواردی که برای فعال سازی این که شماره اشکال را به صورت ارجاعی نشان دهد
	%pagebackref =true,hyperfigures=true,
	%% به جای استفاده از مربع قرمز دور موارد ارجاعی از لینک های رنگی استفاده کند.
	colorlinks=true, 
	%% رنگ برخی از لینک ها در زیر تعریف شده است. 
	linkcolor=\keywordstyleColor, anchorcolor=green, citecolor=magenta, urlcolor=cyan, filecolor=magenta, pdftoolbar=true,bookmarkstype=toc
	%bookmarksopen = true,
	%bookmarksopenlevel = 1
	%%% اگر این option را true‌ بکنیم، آن‌گاه در کنار bookmark شماره فصل و بخش و زیربخش نیز می آید. مثلا می‌نویسد: ۱.۲ طراحی شبکه
	%bookmarksnumbered = true
} % M

%% 0000000000000000000000000000000000000000000000000000000000000000000000000000000000000000000000000000000000000000000000000000000000000000 

%%==================== تنظیمات tikz
\usetikzlibrary{mindmap,backgrounds,shadows,trees,arrows,shapes,positioning}

%% 0000000000000000000000000000000000000000000000000000000000000000000000000000000000000000000000000000000000000000000000000000000000000000 

%%==================== تنظیمات graphicx
% برای اضافه کردن تصاویر به متن این امکان وجود دارد که تصاویر را در پوشه‌های
% متفاوت قرار داد. با این کار از زیاد شدن پرونده‌ها در مسیر مستند جلوگیری می
% شود. علاوه بر این دسته‌ای از تصاویر وجود دارد که بین همه مستندها مشترک است
% برای نمونه نماد پژوهشکده که بین همه مشترک است.  از این رو تعداد مسیر به عنوان
% مسیرهای پیش فرض برای جستجوی تصاویر تعیین شده است.
\graphicspath{{./}{./Pic/}{./images/}{../}{../Pic/}}

%% 0000000000000000000000000000000000000000000000000000000000000000000000000000000000000000000000000000000000000000000000000000000000000000 

%% تنظیم caption جداول و اشکال و ... توسط دستورات memoir
\captionnamefont{\small}
\captiontitlefont{\small}
%\captionstyle{}
%\captionwidth{\linewidth}
%\normalcaptionwidth
%\normalcaption
\captiondelim{: }

%% 0000000000000000000000000000000000000000000000000000000000000000000000000000000000000000000000000000000000000000000000000000000000000000 

%%==================== تنظیمات algorithm و algorithmic
%\floatname{algorithm}{الگوریتم}

%% 0000000000000000000000000000000000000000000000000000000000000000000000000000000000000000000000000000000000000000000000000000000000000000 

%%==================== تنظیمات مربوط به ایجاد watermarking

%% زاویه متن Watermark
\SetWatermarkAngle{45}
%% اندازه watermark
\SetWatermarkScale{1.5}

\let\oldSetWatermarkText\SetWatermarkText
%% اگر بخواهید watermark شما یک رنگ دیگر داشته باشد، این دو خط را فعال کنید و رنگ مورد نظر خود را انتخاب کنید
%\definecolor{orange}{RGB}{229,252,219} 
%\renewcommand{\SetWatermarkText}[1]{\oldSetWatermarkText{\textcolor{orange}{#1}}}

\DeclareDocumentCommand{\SetWatermarkText}{m g}{
	\oldSetWatermarkText{#1}
	\IfValueTF{#2}{
		\SetWatermarkLightness{#2}
	}{%%
		\SetWatermarkLightness{.94}
	}%%
}%

%% 0000000000000000000000000000000000000000000000000000000000000000000000000000000000000000000000000000000000000000000000000000000000000000 
%%% تنظیم فونت
%%% تعریف یک دستور به عنوان فونت پیش فرض
\renewcommand{\defaultFont}{
	%%%  با دستور زیر می توانید فونتی مخصوص عبارات فارسی تعیین کنید:
	\settextfont[Scale=1.3]{B Nazanin} 
	%%\settextfont[Scale=1.2]{XB Niloofar}
	%%% شما با دستور زیر بعد از فراخوانی بسته xepersian می توانید فونت انگلیسی را تعیین کنید
	%%% دقت کنید که عبارات انگلیسی شما باید در دستور \lr{} قرار گیرد تا xepersian بتواند بفمهد که این عبارات انگلیسی است
	%%\setlatintextfont[Scale=1]{Times New Roman}
	\setlatintextfont[Scale=1.1]{Linux Libertine}
	%% % تعریف برای فونت اعداد و ارقام
	%\setdigitfont[Scale=1.1]{XB Zar}
} % M

\defaultFont

%% توسط دستورات defpersianfont و deflatinfont به ترتیب می توان یکسری فونت فارسی و انگلیسی دیگر تعریف کرد که در جاهای دیگر متن بتوان از آن استفاده کرد. برای استفاده کافی است که عبارتی که می خواهیم فونت آن عوض شود را به صورت زیر به عنوان نمونه بنویسیم.
%% \versionfont{این یک مثال است. }

%% تعریف یکسری فونت برای قسمت عنوان پروژه و ما بقی قسمت ها فونت قسمت "موسسه " در صفحه عنوان

\defpersianfont\pejoheshfont[Scale=1.4]{Titr}
%%فونت اسم گروه XB Titre
\defpersianfont\groupfont[Scale=1.4]{XB Zar}
%%% فونت عنوان گزارش
\defpersianfont\titlefont[Scale=2.4]{Titr}
%% فونت نسخه گزارش
\defpersianfont\versionfont[Scale=1.6]{B Mitra}
\defpersianfont\payanFont[Scale=1.8]{XB Yas}
\defpersianfont\nastaliq[Scale=2]{IranNastaliq}
\defpersianfont\farsifontshafigh[Scale=1.3]{XB Shafigh}
\defpersianfont\titrt[Scale=1]{XB Titre}
\defpersianfont\traffict[Scale=1]{XM Traffic}
\defpersianfont\farsifontsayeh[Scale=1.5]{XB Kayhan Sayeh}
\defpersianfont\titlefont[Scale=2.4]{Titr}
\defpersianfont\godFont[Scale=1]{B Nazanin}
\defpersianfont\titleFontEn[Scale=1]{XM Traffic}
%% فونت‌های مورد نیاز برای صفحه شروع 
%\defpersianfont\tablefont[Scale=.8]{XM Traffic}
\deflatinfont\tableFontEn[Scale=.9]{XB Shafigh}
%%  با استفاده از این دستور می‌توان فونت و فارسی و یا انگلیسی بودن اعداد در فرمول‌ها را به حالت اولیه (یعنی پیش‌فرض لاتک) برگرداند.
\DefaultMathsDigits

%% نحوه تغییر اندازه فونت عبارات ریاضی و فرمول‌ها. این کار توسط دستور زیر انجام می‌شود. 
%%\DeclareMathSizes{textsize}{mathsize}{scriptsize}{scriptscriptsize}
%% گزینه اول: این برای چه دسته فونتی است. پیش فرض استایل ما فونت 10pt است. 
%% گزینه دوم: اندازه فونت توابع و موجودات ریاضی درون متن.
%% گزینه سوم: برای اسکریپت ها، اندازه زیرنویس و بالانویس.
%% گزینه چهارم: برای زیرنویس زیرنویس.

%% در دستورات زیر ما برای سه حالت، اندازه‌های مورد نظر را تعریف کرده ایم. 
%%\DeclareMathSizes{10}{11}{9}{8}   % For size 10 text
%%\DeclareMathSizes{11}{12}{11}{10}   % For size 11 text
%%\DeclareMathSizes{12}{13}{12}{11}  % For size 12 text

%%% 0000000000000000000000000000000000000000000000000000000000000000000000000000000000000000000000000000000000000000000000000000000000000000 

%% فایل  Environments  در برگیرنده تعریف یکسری محیط نوین است. 
%%% Data: 2013/04/15    Time: 20:27:46 
%% تعریف برخی نمادهای item و شماره گذاری برای استفاده
\renewcommand{\idx}[1]{\index{#1}#1}
%% می تواند برای ارایه نکات در محیط itemize به کار رود، روند این کار به این صورت است،  (شکل یک تیر)
\renewcommand{\arcm}{\item[\Large\color{red}\ding{247}]}
\renewcommand{\arcmO}{\noindent\textcolor{red}{\Large\ding{247}}\;}
%% این شکل می‌تواند برای بیان مزایای یک قضیه بکار رود (شکل تیک)
\renewcommand{\tick}{\item[\large\color{green}\ding{52}]}
\renewcommand{\tickO}{\noindent\textcolor{green}{\Large\ding{52}}\;}
%% برای  بیان معایب و یا نکات منفی (شکل یک ضربدر)
\renewcommand{\X}{\item[\Large\color{red}\ding{56}]}
\renewcommand{\XO}{\noindent\textcolor{red}{\LARGE\ding{56}}\;}
%% بیان موارد یک قضیه (شکل یک دست)
\renewcommand{\hand}{\item[\Large\color{blue}\ding{45}]}
\renewcommand{\handO}{\noindent\textcolor{blue}{\LARGE\ding{45}}\;}
%% برای مواردی که: این موارد شامل .... می شود، توسط عناصر زیر مشخص می شود (شکل یک درخت)
%% برای نوشتن  پارامتر‌ها، 
\renewcommand{\tree}{\item[\Large\color{ForestGreen}\ding{171}]}
\renewcommand{\treeO}{\noindent\textcolor{ForestGreen}{\Large\ding{171}}\;}
%% برای این که چند مورد را تعریف کنیم (علامت دست که دو گرفته)
\renewcommand{\two}{\item[\LARGE\color{blue}\ding{44}]}
\renewcommand{\twoO}{\noindent\textcolor{blue}{\LARGE\ding{44}}\;}
%% (شکل یک قیچی)
\renewcommand{\sci}{\item[\footnotesize\color{OrangeRed}\ding{108}]}
\renewcommand{\sciO}{\noindent\textcolor{OrangeRed}{\footnotesize\ding{108}}\;}

\renewcommand{\starE}{\item[\Large\color{Plum}\ding{97}]}
\renewcommand{\starEO}{\noindent\textcolor{Plum}{\Large\ding{97}}\;}

%% برای حالت‌هایی که فرضیات داریم. 
\renewcommand{\music}{\item[\Large\color{Green}\ding{161}]}
\renewcommand{\musicO}{\noindent\textcolor{Green}{\Large\ding{161}}\;}

\renewcommand{\gol}{\item[\Huge\color{RubineRed}\ding{96}]}
\renewcommand{\golO}{\noindent\textcolor{RubineRed}{\Huge\ding{96}}\;}

%% OOOOOOOOOOOOOOOOOOOOOOOOOOOOOOOOOOOOOOOOOOOOOOOOOOOOOOOOOOOOOOOOOOOOOOOOO
%% با دستور newtheoremstyle شما می توانید یک استایل جدید برای محیط هایی چون plain، definition‌ و ... تعریف کنید. شکل کلی این دستور به صورت زیر است.

%%\newtheoremstyle{stylename}% name of the style to be used
%%  {spaceabove}% measure of space to leave above the theorem. E.g.: 3pt
%%  {spacebelow}% measure of space to leave below the theorem. E.g.: 3pt
%%  {bodyfont}% name of font to use in the body of the theorem
%%  {indent}% measure of space to indent
%%  {headfont}% name of head font
%%  {headpunctuation}% punctuation between head and body
%%  {headspace}% space after theorem head; " " = normal interword space
%%  {headspec}% Manually specify head
% % تعریف محیط‌های گوناگون مانند محیط برای قضیه و ... 
%% theoremstyle = > plain, definition, remark 
%% با دستور newtheorem یک نوع از استایلی که در بالای آن تعریف شده است ایجاد می کنیم. 
\theoremstyle{plain}
\newtheorem{theorem}{قضیه}
\newtheorem{principle}{اصل}
\newtheorem{lemma}{لم}
\newtheorem{proposition}{گزاره}
\theoremstyle{definition}
%\newtheorem{definition}{تعریف}
\newtheorem{example}{مثال}
\newtheorem{prob}{سوال}
\theoremstyle{remark}
\newtheorem{corollary}{نتیجه}
\newtheorem{property}{نکته}
\newtheorem{remark}{ملاحظه}

\newtheoremstyle{mystyleDef}% name of the style to be used
  {3pt}% measure of space to leave above the theorem. E.g.: 3pt
  {3pt}% measure of space to leave below the theorem. E.g.: 3pt
  {}% name of font to use in the body of the theorem
  {}% measure of space to indent
  {\bfseries}% name of head font
  {\newline}% punctuation between head and body
  {.5em}% space after theorem head; " " = normal interword space
  {}% Manually specify head

\declaretheoremstyle[headfont=\bfseries, notefont=\bfseries , postheadspace=\newline]{mystyle}
\declaretheorem[style=mystyle,name=تعریف]{definition}

\let\olddefnition\definition
\let\oldenddefnition\enddefinition
\renewenvironment{definition}{
\olddefnition
}{
\begin{latin}
\vskip -8mm
\ding{36} ------------------
\end{latin}
\oldenddefnition
}



%\theoremstyle{mystyle}
%\declaretheorem[style=]{definition}{تعریف}
%% در این جا محیط proof را باز تعریف می‌کنیم.
\let\oldproof\proof
\let\oldendproof\endproof
\def\proof{\paragraph{\textcolor{red}{\textbf{اثبات.}}} }
\def\endproof{\hfill$\blacksquare$\oldendproof}

%% تعریف یک محیط برای  اثبات لم ها. در این محیط بر خلاف محیط proof ساده، یک مربع توخالی می‌گذارد. 
\def\lemmaproof{\paragraph{\textcolor{ForestGreen}{\textbf{اثبات لم.}}} }
\def\endlemmaproof{\hfill$\square$\oldendproof}

%%% OOOOOOOOOOOOOOOOOOOOOOOOOOOOOOOOOOOOOOOOOOOOOOOOOOOOOOOOOOOOOOOO

%% %% در ادامه یکسری محیط جالب به صورت کادر رنگی برای استفاده های مختلف تعریف می شود. 
 
\makeatletter
\newdimen\errorsize \errorsize=0.2pt
% Frame with a label at top
\newcommand{\LabFrame}[2]{
	\baselineskip=.4cm
	\fboxrule=\FrameRule
	\fboxsep=-\errorsize
	\textcolor{FrameColor}{
	\fbox{
	\vbox{\nobreak
	\advance\FrameSep\errorsize
	\begingroup
	\advance\baselineskip\FrameSep
	\hrule height \baselineskip
	\nobreak
	\vskip-\baselineskip
	\endgroup
	\vskip 0.5\FrameSep
	\hbox{\hskip\FrameSep \strut
	\textcolor{TitleColor}{\textbf{#1}}}
	\nobreak \nointerlineskip
	\vskip 1.3\FrameSep
	\hbox{\hskip\FrameSep
	{\normalcolor#2}
	\hskip\FrameSep}
	\vskip\FrameSep
}}}}

\definecolor{FrameColor}{rgb}{0.25,0.25,1.0}
\definecolor{TitleColor}{rgb}{1.0,1.0,1.0}

\newenvironment{contlabelframe}[2][\Frame@Lab\ (ادامه)]{% 
	% Optional continuation label defaults to the first label plus
	\def\Frame@Lab{#2}
	\def\FrameCommand{\LabFrame{#2}}
	\def\FirstFrameCommand{\LabFrame{#2}}
	\def\MidFrameCommand{\LabFrame{#1}}
	\def\LastFrameCommand{\LabFrame{#1}}
	\MakeFramed{\advance\hsize-\width \FrameRestore} 
}{\endMakeFramed}
%\newcounter{theoremu}

\renewenvironment{colorBox}[1]{%
	\par
	%\refstepcounter{theoremu}
	\begin{contlabelframe}{{#1}}
	\noindent\ignorespaces
}{
	\end{contlabelframe}
}% 
\makeatother  

%%% ============================================================================================

%% این محیط به صورت یک کادر سایه دار با سایه سیاه رنگ 
\newsavebox\mybox
\renewenvironment{myshadowbox}{%
	\begin{lrbox}{\mybox}
	\begin{minipage}{\dimexpr(\textwidth-2\fboxsep-2\fboxrule-\shadowsize)}
	\baselineskip=.90cm
}{%
	\end{minipage}
	\end{lrbox}
	\vskip10pt
	\noindent
	\shadowbox{\usebox\mybox}
	\vskip10pt
}%

%%% ============================================================================================

%% این محیط برای مواقعی مفید است که می خواهیم یک تمرین و یا سوال طرح کنیم. در این حالت دستوری به نام probsec به صورت زیر تعریف شده است:
%% \probsec{....}
%% که قسمت نقطه چین را می توان به صورت خالی رها نمود. با نوشتن این دستور عبارت سوال به طور خودکار نوشته می شود و سپس شماره آن نیز به طور خودکار قرار داده می شود. اگر شما در نقطه چین موردی را بنویسید این مورد به صورت عنوان سوال قرار می گیرد.  یعنی مثلا در کد زیر:
%% \probsec{شبکه}
%% در این صورت مثلا می نویسد: سوال ۱: شبکه و خود سوال از خط بعدی شروع می شود. 

%% برای شماره گذاری محیط یاد شده ابتدا یک counter‌ تعریف می کنیم. 
\newcounter{problemcount}
\addtocounter{problemcount}{1} % set them to some other numbers than 0

\renewcommand{\probsec}[1]{{\noindent\normalfont\bfseries{\textcolor{blue}{
	سوال
 \arabic{problemcount}\, {#1}}}\medskip }
	\addtocounter{problemcount}{1} 
}

%%% ============================================================================================

\newcommand{\handBS}{\noindent\textcolor{ForestGreen}{\Huge\ding{45}}}
\RenewDocumentEnvironment{note}{g g}{
	\tikzstyle{mybox1} = [draw=YellowGreen, fill=green!15,very thick, rectangle, rounded corners, inner sep=10pt, inner ysep=20pt]
	\tikzstyle{fancytitle1} =[fill=YellowGreen, text=white]
	\tikzstyle{fancytitle2} =[fill=YellowGreen!5, text=white]
	\tikzstyle{fancytitle3} =[fill=white, text=white]
	\begin{center}
		\begin{tikzpicture}
			\node [mybox1] (box)\bgroup
			\IfValueTF{#2}{
				\IfFileExists{#2}{\begin{minipage}{.85\textwidth}}{\begin{minipage}{.93\textwidth}}
			}{%%
				\IfFileExists{note.png}{\begin{minipage}{.85\textwidth}}{\begin{minipage}{.93\textwidth}}
			}%%
			\baselineskip=.95cm
				\begin{RTL}
}{%
				\end{RTL}
			\end{minipage}
			\egroup;
			\IfValueTF{#1}{\node[fancytitle1, left=10pt] at (box.north east) {\hboxR{#1}};}{\node[fancytitle1, left=10pt] at (box.north east) {\hboxR{نکته}};}%
			\IfValueTF{#2}{
				\IfFileExists{#2}
				{\node[fancytitle3, left=3pt,   rounded corners] at (box.west) {\includegraphics[width=.07\textwidth]{#2}}; }
				{\node[fancytitle2,  rounded corners] at (box.west) {\handBS};}			
			}{%%
				\IfFileExists{note.png}
				{\node[fancytitle3, left=3pt,   rounded corners] at (box.west) {\includegraphics[width=.07\textwidth]{note}}; }
				{\node[fancytitle2,  rounded corners] at (box.west) {\handBS};}
			}%%
		\end{tikzpicture}
	\end{center}
}%

%%% ============================================================================================

\newcommand{\treeBS}{\noindent\textcolor{blue}{\Huge\ding{171}}}
\RenewDocumentEnvironment{goal}{g g}{
	\tikzstyle{mybox1} = [draw=blue, fill=blue!15,very thick, rectangle, rounded corners, inner sep=10pt, inner ysep=20pt]
	\tikzstyle{fancytitle1} =[fill=blue!90, text=white]
	\tikzstyle{fancytitle2} =[fill=blue!5, text=white]
	\tikzstyle{fancytitle3} =[fill=white, text=white]
	\begin{center}
		\begin{tikzpicture}
			\node [mybox1] (box)\bgroup
			\IfValueTF{#2}{
				\IfFileExists{#2}{\begin{minipage}{.85\textwidth}}{\begin{minipage}{.93\textwidth}}
			}{%%
				\IfFileExists{archeryf.pdf}{\begin{minipage}{.85\textwidth}}{\begin{minipage}{.93\textwidth}}
			}%%
			\baselineskip=.95cm
				\begin{RTL}
}{%
				\end{RTL}
			\end{minipage}
			\egroup;
			\IfValueTF{#1}{\node[fancytitle1, left=10pt] at (box.north east) {\hboxR{#1}};}{\node[fancytitle1, left=10pt] at (box.north east) {\hboxR{هدف}};}%
			\IfValueTF{#2}{
				\IfFileExists{#2}
				{\node[fancytitle3, left=3pt,   rounded corners] at (box.west) {\includegraphics[width=.07\textwidth]{#2}}; }
				{\node[fancytitle2,  rounded corners] at (box.west) {\treeBS};}			
			}{%%
				\IfFileExists{archeryf.pdf}
				{\node[fancytitle3, left=3pt,   rounded corners] at (box.west) {\includegraphics[width=.07\textwidth]{archeryf}}; }
				{\node[fancytitle2,  rounded corners] at (box.west) {\treeBS};}
			}%%
		\end{tikzpicture}
	\end{center}
}%

%%% ============================================================================================

\newcommand{\arcBS}{\noindent\textcolor{red}{\Huge\ding{247}}}
\RenewDocumentEnvironment{warning}{g g}{
	\tikzstyle{mybox1} = [draw=red, fill=red!15,very thick, rectangle, rounded corners, inner sep=10pt, inner ysep=20pt]
	\tikzstyle{fancytitle1} =[fill=red!90, text=white]
	\tikzstyle{fancytitle2} =[fill=red!4, text=white]
	\tikzstyle{fancytitle3} =[fill=white, text=white]
	\begin{flushleft}
	\begin{tikzpicture}
	\node [mybox1] (box)\bgroup
	\IfValueTF{#2}{
		\IfFileExists{#2}{\begin{minipage}{.85\textwidth}}{\begin{minipage}{.93\textwidth}}
	}{%%
		\IfFileExists{warining.png}{\begin{minipage}{.85\textwidth}}{\begin{minipage}{.93\textwidth}}
	}%%
	\baselineskip=.95cm
		\begin{RTL}
}{%
		\end{RTL}
	\end{minipage}
	\egroup;
	\IfValueTF{#1}{\node[fancytitle1, left=10pt] at (box.north east) {\hboxR{#1}};}{\node[fancytitle1, left=10pt] at (box.north east) {\hboxR{توجه}};}%
	\IfValueTF{#2}{
		\IfFileExists{#2}
		{\node[fancytitle3, left=3pt,   rounded corners] at (box.west) {\includegraphics[width=.07\textwidth]{#2}}; }
		{\node[fancytitle2,  rounded corners] at (box.west) {\arcBS};}			
	}{%%
		\IfFileExists{warining.png}
		{\node[fancytitle3, left=3pt,   rounded corners] at (box.west) {\includegraphics[width=.07\textwidth]{warining.png}}; }
		{\node[fancytitle2,  rounded corners] at (box.west) {\arcBS};}
	}%%
	\end{tikzpicture}
	\end{flushleft}
}%

%%% ============================================================================================

\newcommand{\envBS}{\noindent\textcolor{Violet}{\Huge\ding{41}}}
\RenewDocumentEnvironment{refer}{g g}{
	\tikzstyle{mybox1} = [draw=Violet, fill=Violet!10,very thick, rectangle, rounded corners, inner sep=10pt, inner ysep=20pt]
	\tikzstyle{fancytitle1} =[fill=Violet!50, text=white]
	\tikzstyle{fancytitle2} =[fill=Violet!20, text=white]
	\tikzstyle{fancytitle3} =[fill=white, text=white]
	\begin{flushleft}
	\begin{tikzpicture}
	\node [mybox1] (box)\bgroup
	\IfValueTF{#2}{
		\IfFileExists{#2}{\begin{minipage}{.85\textwidth}}{\begin{minipage}{.93\textwidth}}
	}{%%
		\IfFileExists{referO.pdf}{\begin{minipage}{.85\textwidth}}{\begin{minipage}{.93\textwidth}}
	}%%
	\baselineskip=.95cm
	\begin{RTL}
}{%
	\end{RTL}
	\end{minipage}
	\egroup;
	\IfValueTF{#1}{\node[fancytitle1, left=10pt] at (box.north east) {\hboxR{#1}};}{\node[fancytitle1, left=10pt] at (box.north east) {\hboxR{مراجع مفید}};}%
	\IfValueTF{#2}{
		\IfFileExists{#2}
		{\node[fancytitle3, left=3pt,   rounded corners] at (box.west) {\includegraphics[width=.07\textwidth]{#2}}; }
		{\node[fancytitle2,  rounded corners] at (box.west) {\envBS};}			
	}{%%
		\IfFileExists{referO.pdf}
		{\node[fancytitle3, left=3pt,   rounded corners] at (box.west) {\includegraphics[width=.07\textwidth]{referO}}; }
		{\node[fancytitle2,  rounded corners] at (box.west) {\envBS};}
	}%%
	\end{tikzpicture}
	\end{flushleft}
}%

\renewcommand{\goodRef}[1]{ \begin{refer} #1 \end{refer} }

%%% ============================================================================================

\newcommand{\twoBS}{\noindent\textcolor{YellowOrange}{\Huge\ding{44}}}
\RenewDocumentEnvironment{info}{g g}{
	\tikzstyle{mybox1} = [draw=YellowOrange, fill=YellowOrange!10,very thick, rectangle, rounded corners, inner sep=10pt, inner ysep=20pt]
	\tikzstyle{fancytitle1} =[fill=YellowOrange!50, text=white]
	\tikzstyle{fancytitle2} =[fill=YellowOrange!15, text=white]
	\tikzstyle{fancytitle3} =[fill=white, text=white]
	\begin{flushleft}
	\begin{tikzpicture}
	\node [mybox1] (box)\bgroup
	\IfValueTF{#2}{
		\IfFileExists{#2}{\begin{minipage}{.85\textwidth}}{\begin{minipage}{.93\textwidth}}
	}{
		\IfFileExists{infoRR.png}{\begin{minipage}{.85\textwidth}}{\begin{minipage}{.93\textwidth}}
	}%%
	\baselineskip=.95cm
	\begin{RTL}
}{%
	\end{RTL}
	\end{minipage}
	\egroup;
	\IfValueTF{#1}{\node[fancytitle1, left=10pt] at (box.north east) {\hboxR{#1}};}{\node[fancytitle1, left=10pt] at (box.north east) {\hboxR{مطالب بیشتر}};}%%
	\IfValueTF{#2}{
		\IfFileExists{#2}
		{\node[fancytitle3, left=3pt,   rounded corners] at (box.west) {\includegraphics[width=.07\textwidth]{#2}}; }
		{\node[fancytitle2,  rounded corners] at (box.west) {\twoBS};}			
	}{
		\IfFileExists{infoRR.png}
		{\node[fancytitle3, left=3pt,   rounded corners] at (box.west) {\includegraphics[width=.07\textwidth]{infoRR}}; }
		{\node[fancytitle2,  rounded corners] at (box.west) {\twoBS};}
	}%%
	\end{tikzpicture}
	\end{flushleft}
}%

%%% ============================================================================================


\newcommand{\teleBS}{\noindent\textcolor{Mulberry}{\Huge\ding{37}}}
\RenewDocumentEnvironment{problem}{g g}{
	\tikzstyle{mybox1} = [draw=Mulberry, fill=Mulberry!10,very thick, rectangle, rounded corners, inner sep=10pt, inner ysep=20pt]
	\tikzstyle{fancytitle1} =[fill=Mulberry!50, text=white]
	\tikzstyle{fancytitle2} =[fill=Mulberry!15, text=white]
	\tikzstyle{fancytitle3} =[fill=white, text=white]
	\begin{flushleft}
	\begin{tikzpicture}
	\node [mybox1] (box)\bgroup
	\IfValueTF{#2}{
		\IfFileExists{#2}{\begin{minipage}{.85\textwidth}}{\begin{minipage}{.93\textwidth}}
	}{
		\IfFileExists{home.png}{\begin{minipage}{.85\textwidth}}{\begin{minipage}{.93\textwidth}}
	}%%
	\baselineskip=.95cm
	\begin{RTL}
}{%
	\end{RTL}
	\end{minipage}
	\egroup;
	\IfValueTF{#1}{\node[fancytitle1, left=10pt] at (box.north east) {\hboxR{#1}};}{\node[fancytitle1, left=10pt] at (box.north east) {\hboxR{سوال}};}%%
	\IfValueTF{#2}{
		\IfFileExists{#2}
		{\node[fancytitle3, left=3pt,   rounded corners] at (box.west) {\includegraphics[width=.07\textwidth]{#2}}; }
		{\node[fancytitle2,  rounded corners] at (box.west) {\twoBS};}			
	}{
		\IfFileExists{home.png}
		{\node[fancytitle3, left=3pt,   rounded corners] at (box.west) {\includegraphics[width=.07\textwidth]{home}}; }
		{\node[fancytitle2,  rounded corners] at (box.west) {\teleBS};}
	}%%
	\end{tikzpicture}
	\end{flushleft}
}%


%%%% 000000000000000000000000000000000000000000000000000000000000000000000000000000000000000000000000000000000000000000000000000000

\newcommand{\defBS}{\noindent\textcolor{ForestGreen}{\Huge\ding{45}}}
\RenewDocumentEnvironment{mydef}{g g}{
	\tikzstyle{mybox1} = [draw=Plum, fill=Plum!15,very thick, rectangle, rounded corners, inner sep=10pt, inner ysep=20pt]
	\tikzstyle{fancytitle1} =[fill=Plum, text=white]
	\tikzstyle{fancytitle2} =[fill=Plum!5, text=white]
	\tikzstyle{fancytitle3} =[fill=white, text=white]
	\begin{center}
		\begin{tikzpicture}
			\node [mybox1] (box)\bgroup
			\IfValueTF{#2}{
				\IfFileExists{#2}{\begin{minipage}{.85\textwidth}}{\begin{minipage}{.93\textwidth}}
			}{%%
				\IfFileExists{defi.png}{\begin{minipage}{.85\textwidth}}{\begin{minipage}{.93\textwidth}}
			}%%
			\baselineskip=.95cm
				\begin{RTL}
}{%
				\end{RTL}
			\end{minipage}
			\egroup;
			\IfValueTF{#1}{\node[fancytitle1, left=10pt] at (box.north east) {\hboxR{#1}};}{\node[fancytitle1, left=10pt] at (box.north east) {\hboxR{تعریف}};}%
			\IfValueTF{#2}{
				\IfFileExists{#2}
				{\node[fancytitle3, left=3pt,   rounded corners] at (box.west) {\includegraphics[width=.07\textwidth]{#2}}; }
				{\node[fancytitle2,  rounded corners] at (box.west) {\defBS};}			
			}{%%
				\IfFileExists{defi.png}
				{\node[fancytitle3, left=3pt,   rounded corners] at (box.west) {\includegraphics[width=.07\textwidth]{defi}}; }
				{\node[fancytitle2,  rounded corners] at (box.west) {\defBS};}
			}%%
		\end{tikzpicture}
	\end{center}
}%



\newcommand{\commentBSS}{\noindent\textcolor{ForestGreen}{\Huge\ding{23}}}
\RenewDocumentEnvironment{mycomment}{g g}{
	\tikzstyle{mybox1} = [draw=red,fill=none,very thick, rectangle, rounded corners, inner sep=10pt, inner ysep=20pt]
	\tikzstyle{fancytitle1} =[fill=red, text=white]
	\tikzstyle{fancytitle2} =[fill=red!5, text=white]
	\tikzstyle{fancytitle3} =[fill=white, text=white]
	\begin{center}
		\begin{tikzpicture}
			\node [mybox1] (box)\bgroup
			\IfValueTF{#2}{
				\IfFileExists{#2}{\begin{minipage}{.85\textwidth}}{\begin{minipage}{.93\textwidth}}
			}{%%
				\IfFileExists{robah.png}{\begin{minipage}{.85\textwidth}}{\begin{minipage}{.93\textwidth}}
			}%%
			\baselineskip=.95cm
				\begin{RTL}
}{%
				\end{RTL}
			\end{minipage}
			\egroup;
			\IfValueTF{#1}{\node[fancytitle1, left=10pt] at (box.north east) {\hboxR{#1}};}{\node[fancytitle1, left=10pt] at (box.north east) {\hboxR{توضیح}};}%
			\IfValueTF{#2}{
				\IfFileExists{#2}
				{\node[fancytitle3, left=3pt,   rounded corners] at (box.west) {\includegraphics[width=.07\textwidth]{#2}}; }
				{\node[fancytitle2,  rounded corners] at (box.west) {\commentBSS};}			
			}{%%
				\IfFileExists{robah.png}
				{\node[fancytitle3, left=3pt,   rounded corners] at (box.west) {\includegraphics[width=.07\textwidth]{robah}}; }
				{\node[fancytitle2,  rounded corners] at (box.west) {\commentBSS};}
			}%%
		\end{tikzpicture}
	\end{center}
}%




%% فایل  Environments  در برگیرنده تعریف یکسری استایل برای فصل‌ها است. 
% %% انواع مختلف استایل‌های برای عنوان فصل‌ها 
%% در ادامه برخی از استایل های زیبا برای عنوان فصل تعریف می شود. برای استفاده از این استایل ها کافی است که  از دستور  \chapterstyle به همراه نام استایل استفاده کنید. 
%% کد این استایل ها از بسته MemoirChapStyles استخراج شده است. 

% % تعاریف مربوط به استایل اول: با استفاده از کلاس memoir 
\newsavebox{\ChpNumBox}
\definecolor{ChapBlue}{rgb}{0.00,0.65,0.65}
\makeatletter
\newcommand *{\thickhrulefill}{\leavevmode\leaders\hrule height 1\p@ \hfill \kern \z@}
\newcommand *\BuildChpNum[2]{
\begin{tabular}[t]{@{}c@{}}
	\makebox[0pt][c]{#1\strut} \\[.5ex]
	\colorbox{ChapBlue}{
	\rule[-10em]{0pt}{0pt}
	\rule{1ex}{0pt}\color{black}#2\strut
	\rule{1ex}{0pt}}
	\end{tabular}
}%

% تعریف استایل اول برای شروع فصل‌ها
\makechapterstyle{BlueBox}{%
	\renewcommand{\chapnamefont}{\LARGE\scshape}
	\renewcommand{\chapnumfont}{\Huge\bfseries}
	\renewcommand{\chaptitlefont}{\Huge\bfseries}
	\setlength{\beforechapskip}{5pt}
	\setlength{\midchapskip}{26pt}
	\setlength{\afterchapskip}{40pt}
	\renewcommand{\printchaptername}{}
	\renewcommand{\chapternamenum}{}
	\renewcommand{\printchapternum}{%
	\sbox{\ChpNumBox}{%
	\BuildChpNum{\chapnamefont\@chapapp}%
	{\chapnumfont\thechapter}}}
	\renewcommand{\printchapternonum}{%
	\sbox{\ChpNumBox}{%
	\BuildChpNum{\chapnamefont\vphantom{\@chapapp}}%
	{\chapnumfont\hphantom{\thechapter}}}}
	\renewcommand{\afterchapternum}{}
	\renewcommand{\printchaptertitle}[1]{%
	\usebox{\ChpNumBox}\hfill
	\parbox[t]{\hsize-\wd\ChpNumBox-1em}{%
	\vspace{\midchapskip}%
	\thickhrulefill\par
	\chaptitlefont ##1\par}}%
}%
%%
%%%%% تعریف استایل دوم  برای شروع فصل‌ها
%%\makechapterstyle{asuappendix}{%
%%	\setlength{\beforechapskip}{-\topfiddle}
%%	\setlength{\midchapskip}{1.0\onelineskip}
%%	\setlength{\afterchapskip}{1.0\onelineskip}
%%	\renewcommand*{\chapnamefont}{\normalfont}
%%	\renewcommand*{\chapnumfont}{\chapnamefont}
%%	\renewcommand*{\printchaptername}{%
%%	\chapnamefont\MakeUppercase{\@chapapp}}
%%	\renewcommand*{\printchapternum}{\centering\chapnumfont \thechapter}
%%	\renewcommand*{\chaptitlefont}{\normalfont\centering}
%%	\renewcommand*{\printchapternonum}{}
%%	\renewcommand*{\afterchaptertitle}{\clearpage}z
%%}%
%%
%%
%%\newif\ifNoChapNumber
%%\newcommand{\Vlines}{%
%%	\def\VL{\rule[2cm]{1pt}{5cm}\hspace{1mm}\relax}
%%	\VL\VL\VL\VL\VL\VL\VL}
%%	\setlength\midchapskip{0pt}
%%	\makechapterstyle{VZ43}{
%%	\renewcommand\chapternamenum{}
%%	\renewcommand\printchaptername{}
%%	\renewcommand\printchapternum{}
%%	\renewcommand\chapnumfont{\Huge\bfseries\centering}
%%	\renewcommand\chaptitlefont{\Huge\bfseries\raggedright}
%%	\renewcommand\printchaptertitle[1]{
%%		%\begin{tabular}{@{}p{1cm} p{\textwidth3cm}}%
%%		%\ifNoChapNumber\relax\else%
%%		\colorbox{black}{\color{white}%
%%		\makebox[.8cm]{\chapnumfont\strut \thechapter}}
%%		%\fi
%%		 \chaptitlefont ##1
%%		%\end{tabular}
%%		\NoChapNumberfalse
%%	}%%
%%	\renewcommand\printchapternonum{\NoChapNumbertrue}
%%}%
%%
%%
%%
%%\definecolor{nicered}{rgb}{.647,.129,.149}
%%\makeatletter
%%\newlength\dlf@normtxtw
%%\setlength\dlf@normtxtw{\textwidth}
%%\def\myhelvetfont{\def\sfdefault{mdput}}
%%\newsavebox{\feline@chapter}
%%\newcommand\feline@chapter@marker[1][4cm]{
%%	\sbox\feline@chapter{
%%	\resizebox{!}{#1}{\fboxsep=1pt%
%%	\colorbox{nicered}{\color{white}\bfseries\sffamily\thechapter}%
%%	}}%
%%	\rotatebox{90}{%
%%	\resizebox{%
%%	\heightof{\usebox{\feline@chapter}}+\depthof{\usebox{\feline@chapter}}}%
%%	{!}{\scshape\so\@chapapp}}\quad%
%%	\raisebox{\depthof{\usebox{\feline@chapter}}}{\usebox{\feline@chapter}}%
%%}
%%\newcommand\feline@chm[1][4cm]{
%%	\sbox\feline@chapter{\feline@chapter@marker[#1]}
%%	\makebox[0pt][l]{\makebox[1cm][r]{\usebox\feline@chapter}}
%%}
%%\makechapterstyle{daleif1}{
%%	\renewcommand\chapnamefont{\normalfont\Large\scshape\raggedleft\so}
%%	\renewcommand\chaptitlefont{\normalfont\huge\bfseries\scshape\color{nicered}}
%%	\renewcommand\chapternamenum{}
%%	\renewcommand\printchaptername{}
%%	\renewcommand\printchapternum{\null\feline@chm[2.5cm]\par}
%%	\renewcommand\afterchapternum{\par\vskip\midchapskip}
%%	\renewcommand\printchaptertitle[1]{\chaptitlefont\raggedleft ##1\par}
%%}
%%\makeatother


















%%% تعریف یکسری استایل برای صفحه شروع 
%%% Data: 2013/04/16    Time: 23:46:34 

%%  برگه نخست در مستند را ایجاد می‌کند. در این برگه تنها نام پروردگار
%% ذکر می‌شود. برای ایجاد نام پرودگار یک تصویر استفاده می‌شود به نام god.ps
%% که باید در پوشه image کنار پرونده اصلی مستند قرار گرفته شده باشد.
\DeclareDocumentCommand{\Godpage}{g g}{	
	\thispagestyle{empty}
	\begin{center}
	\IfValueTF{#1}{
		\IfValueTF{#2}{
			\includegraphics[width=#2\textwidth]{#1}
		}{%%%
			\includegraphics[width=.7\textwidth]{#1}
		}%%%
	}{%%
		\IfValueTF{#2}{
			\includegraphics[width=#2\textwidth]{god}
		}{%%%
			\includegraphics[width=.7\textwidth]{god}
		}%%%
	}%%
	\end{center}
	\newpage
}%



%%% تعریف یکسری فونت مورد استفاده اختصاصی در صفحه عنوان
\defpersianfont\logofontR[Scale=1.2]{Zar}
\defpersianfont\typefontR[Scale=1.8]{Zar}
\defpersianfont\titlefontR[Scale=2]{Titr}
\defpersianfont\datafontR[Scale=1.4]{B Zar}
\defpersianfont\proposalFont[Scale=1]{Titr}
\defpersianfont\InstituteFont[Scale=.8]{XM Traffic}
\defpersianfont\godFont[Scale=.9]{XM Traffic}


\makeatletter
\newcommand{\pejoheshStyle}{
%	%% این یک استایل ساده و رسمی.
%	%% توسط این دستور تمامی footer و header های صفحه را حذف می‌کنیم. 
	\thispagestyle{empty}
	\begin{center}
		\includegraphics[width=\@logoScale]{\@logofile}\\*[10pt]
		\logofontR{\@institute}\medskip
		\vspace*{\stretch{2}}
		\typefontR\textbf{\@type}\\*[35pt]
		\titlefontR\textbf{\@title}\medskip
		\vspace*{\stretch{2}}
		\datafontR{\@author}\medskip
		\vspace*{\stretch{2}}
		\datafontR{\@date}
		\vspace*{\stretch{1}}
	\end{center}
	\clearpage
} %

\newcommand{\lshortStyle}{
	%% استایلی شبیه به استایل شروع کتاب "مقدمه‌ای نه چندان کوتاه بر Latex"
	\definecolor{authorcol}{rgb}{.51,0,.51}
	\begin{flushleft}
	\vspace*{\stretch{.1}}
	\begin{flushright}\includegraphics[width=\@logoScale]{\@logofile}\end{flushright}
	\newlength{\centeroffset}
	\setlength{\centeroffset}{-0.5\oddsidemargin}
	\addtolength{\centeroffset}{0.5\evensidemargin}
	\addtolength{\textwidth}{-\centeroffset}
	%% توسط این دستور تمامی footer و header های صفحه را حذف می‌کنیم. 
	\thispagestyle{empty}
	
	\vspace*{\stretch{2}}
	
	\noindent\hspace*{\centeroffset}\makebox[0pt][l]{
		\begin{minipage}{\textwidth}
			\flushleft
			{ \titlefont\textcolor{magenta}\@title \\*[10pt]}
			\noindent\color{gray}{\rule[-1ex]{\textwidth}{5pt}\\[2.5ex]
			\hfill{\farsifontsayeh\large\@type }}
		\end{minipage}
	}%
	
	\vspace{\stretch{2}}
	
	\noindent\hspace*{\centeroffset}\makebox[0pt][l]{
		\begin{minipage}{\textwidth}
			{\flushleft\textcolor{authorcol}{\bfseries\@author\\*[5pt]}}
			{\flushleft\textcolor{authorcol}{\bfseries\@supervisor\\*[5pt]}}
			{\flushleft\textcolor{authorcol}{\bfseries\@date}\\}
		\end{minipage}
	}%
		
	\vspace{\stretch{1.5}}

	\end{flushleft}
	\clearpage
}%
	
\newcommand{\presStyle}{
%%% ایجاد عنوان برای فایل ارایه (Presentation) 
%%% توسط این دستور تمامی footer و header های صفحه را حذف می‌کنیم. 
	\thispagestyle{empty}

	\begin{titlepage}
		\distance{1}
		\centering
		\textcolor{Green}{\huge\titr{\@title}}\\
		\distance{2}
		\textcolor{Magenta}{\textbf{\@session}}\\*[14pt]
		\textcolor{Magenta}{\small\textbf{\@date}}
		\distance{1}
	\end{titlepage}
	\clearpage\newpage
	\pagestyle{empty}
	\pagenumbering{arabic}
}%
	
\newcommand{\presStyleLogo}{
%%% ایجاد عنوان برای فایل ارایه (Presentation) 
%%% توسط این دستور تمامی footer و header های صفحه را حذف می‌کنیم. 
	\thispagestyle{empty}

	\begin{titlepage}
	\begin{table}[H]
		\begin{tabular}{rl}
		\includegraphics{logo} & \includegraphics{logo2} \\
		\end{tabular}

			\end{table}
		\distance{1}
		\centering
		\textcolor{Green}{\huge\titr{\@title}}\\
		\distance{2}
		\textcolor{Magenta}{\textbf{\@session}}\\*[14pt]
		\textcolor{Magenta}{\small\textbf{\@date}}
		\distance{1}
	\end{titlepage}
	\clearpage\newpage
	\pagestyle{empty}
	\pagenumbering{arabic}
}%
	
	
\newcommand{\techStyle}{
	\thispagestyle{empty}
	\centering
	\godFont\textbf{باسمعه تعالی}\\
	\vspace*{\stretch{1}}
	 {\includegraphics[width=\@logoScale]{\@logofile}}\\
	\InstituteFont{\@institute}\\
% 	\rule{0.25\textwidth}{.5pt}\\
 	\InstituteFont{\@group}\\
	\vspace*{\stretch{1}}
	\proposalFont\textbf{\@type}\\*[8pt]
	\proposalFont\textbf{\@title}\\
	\vspace*{\stretch{1}}
	\raggedleft\tableFont\textbf{اطلاعات کلي:}\\
	\begin{table}
	\renewcommand{\arraystretch}{1.2}
	%\setlength{\arrayrulewidth}{1pt}
	\begin{tabular}{Im{0.3\textwidth}|m{0.5\textwidth}I}\whline
	\tableFont\textbf{عنوان پروژه:}&\tableFont\textbf{\@title}\\\hline
	\tableFont\textbf{نام کارفرما:}&\tableFont\textbf{\@supervisor}\\\hline
	\tableFont\textbf{نام پيشنهاد دهنده:}&\tableFont\textbf{\@author}\\\hline
	\tableFont\textbf{گروه تخصصي :}&\tableFont\textbf{\@community}\\\hline
	\tableFont\textbf{مدير پروژه:}&\tableFont\textbf{\@projectManager}\\\hline
	\tableFont\textbf{تاريخ ارائه:}&\tableFont\textbf{\@date}\\\hline
	\tableFont\textbf{شماره نسخه:}&\tableFont\textbf{ \@version.\@update}\\
	\whline
	\end{tabular}
	\end{table}
	\vspace*{\stretch{1}}
	
%	\raggedleft\tableFont\textbf{مشخصات مجريان پروژه: }\\
%	\begin{table}
%	\centering
%	\renewcommand{\arraystretch}{1.2}
%	%\setlength{\arrayrulewidth}{1pt}
%	\begin{tabular}{Im{0.2\textwidth}|m{0.2\textwidth}|m{0.2\textwidth}|m{0.2\textwidth}I}\whline
%	\tableFont\textbf{نام}&\tableFont\textbf{مدرک تحصيلي}&\tableFont\textbf{گرايش}&\tableFont\textbf{مسئوليت}\\\hline
%	\tableFont\textbf{}&\tableFont\textbf{}&\tableFont\textbf{}&\tableFont\textbf{}\\\hline
%	\tableFont\textbf{}&\tableFont\textbf{}&\tableFont\textbf{}&\tableFont\textbf{}\\\hline
%	\tableFont\textbf{}&\tableFont\textbf{}&\tableFont\textbf{}&\tableFont\textbf{}\\\whline
%	\end{tabular}
%	\end{table}
	\vspace*{\stretch{1}}
	\begin{table}
	\renewcommand{\arraystretch}{1.2}
	\begin{tabular}{Im{0.2\textwidth}|m{0.2\textwidth}I}\whline
	\tableFont\textbf{مدت زمان اجرا:}&\tableFont\textbf{\@executionTime}\\\hline
	\tableFont\textbf{تاريخ شروع:}&\tableFont\textbf{\@startData}\\\hline
	\tableFont\textbf{تاريخ پايان فاز اول:}&\tableFont\textbf{\@stopData}\\\whline
	\end{tabular}
	\end{table}
	\vspace*{\stretch{1}}
	\clearpage
}%

\newcommand{\thesisStyleOO}{
	\ignorespaces
	\begin{center}
	\thispagestyle{empty}
	\begin{table}
		\begin{tabular}{ccc}
			\includegraphics[width=.2\textwidth]{logo}
			&
			\begin{minipage}{0.55\linewidth}
				\vskip 0.9cm
				\begin{center}
					\typefontR{\@institute}\\* [0.4cm]
					\@faculity\\ [0.4cm]
					\@group \\*[1cm]
				\end{center}
			\end{minipage}
			&
			\includegraphics[width=.2\textwidth]{logo2}
		\end{tabular}
	\end{table}
	\vspace*{\stretch{1}}
	\textbf{\LARGE{\@title}}
	\vspace*{\stretch{1}}
	\vskip 1.2cm
	\LARGE{
	نگارش:}
	 \\ [0.4cm] \Large{\@author}
	\vskip 1.5cm
	\LARGE{
	استاد راهنما:}
	 \\ [0.4cm] \Large{\@supervisor}
%	\vskip 1.5cm
%	\LARGE{
%	استاد مشاور:}
%	 \\ [0.4cm] \Large{\@adviser}
	\vskip 2cm
	%\@comment \\* [0.3cm]
	\large{\@date}
	\end{center}
	\clearpage
}%
\newcommand{\thesisStyleO}{
	\ignorespaces
	\begin{center}
	\thispagestyle{empty}
	\includegraphics[width=\@logoScale]{\@logofile}$ $\\* [0.3cm]
	{{\@institute}$ $\\* [-1mm]
	{\@faculity}$ $\\* [-1mm]
	{\@group} $ $\\* [-1mm]}
	\vspace*{\stretch{1}}
	{\textbf{\LARGE{\@title}}}
	\vspace*{\stretch{1}}
	\vskip 2cm
	{\Large
	نگارش:
	 \\ [0.2cm] \@author}
	\vskip 1.5cm
	{\Large
	استاد راهنما:
	 \\ [0.2cm] \@supervisor}
	\vskip 1.5cm
	{\Large
	استاد مشاور:
	 $ $\\ [0.2cm] \@adviser}
	\vskip 2cm
	 {\@comment $ $\\* [0.3cm]}
	\large{\@date}
	\end{center}
	\clearpage
}%
\newcommand{\proposalStyle}{
	{\thispagestyle{empty}
	\centering
	{\godFont\textbf{باسمعه تعالی}}\\
	\vspace*{\stretch{.5}}
%	 \IfFileExists{logo.jpg}
	 {\includegraphics[width=0.15\textwidth]{logo}} \\
	{\InstituteFont{\@institute}}\\*[-2pt]
 	{\InstituteFont{\@group}}\\*[12pt]
	\vspace*{\stretch{3}}
	{\large\proposalFont\textbf{\@type}}\\*[19pt]
	{\Huge\proposalFont\textbf{\@title}}\\
	\vspace*{\stretch{3}}

	\raggedleft\tablefont\textbf{اطلاعات کلي:}\\
	\begin{table}[H]
	\renewcommand{\arraystretch}{1.2}
	%\setlength{\arrayrulewidth}{1pt}
	\begin{tabular}{Im{0.3\textwidth}|m{0.5\textwidth}I}\whline
	\tablefont\textbf{عنوان پروژه:}        &\tablefont\textbf{\@title}\\\hline
	\tablefont\textbf{نام کارفرما:}         &\tablefont\textbf{\@supervisor}\\\hline
	\tablefont\textbf{نام پيشنهاد دهنده:} &\tablefont\textbf{\@author}\\\hline
	\tablefont\textbf{گروه تخصصي :}   &\tablefont\textbf{\@community}\\\hline
	\tablefont\textbf{مدير پروژه:}          &\tablefont\textbf{\@projectManager}\\\hline
	\tablefont\textbf{تاريخ ارائه:}          &\tablefont\textbf{\@date}\\\hline
	\tablefont\textbf{شماره نسخه:}       &\tablefont\textbf{ \@version-\@update}\\
	\whline
	\end{tabular}
	\end{table}
	\vspace*{\stretch{1}}
	
	\raggedleft\tablefont\textbf{مشخصات مجريان پروژه: }\\
	\begin{table}[H]
	\centering
	\renewcommand{\arraystretch}{1.2}
	%\setlength{\arrayrulewidth}{1pt}
	\begin{tabular}{Im{0.2\textwidth}|m{0.2\textwidth}|m{0.2\textwidth}|m{0.2\textwidth}I}\whline
	\tablefont\textbf{نام و نام‌خانوادگی}&\tablefont\textbf{مدرک تحصيلي}&\tablefont\textbf{گرايش}&\tablefont\textbf{مسئوليت}\\\hline
	\fulltable\\
        \whline
	\end{tabular}
	\end{table}
	\vspace*{\stretch{1}}
	\begin{table}[H]
	\renewcommand{\arraystretch}{1.2}
	\begin{tabular}{Im{0.2\textwidth}|m{0.2\textwidth}I}\whline
	\tablefont\textbf{مدت زمان اجرا:}&\tablefont\textbf{\@executionTime}\\\hline
	\tablefont\textbf{تاريخ شروع:}&\tablefont\textbf{\@startData}\\\hline
	\tablefont\textbf{تاريخ پايان فاز اول:}&\tablefont\textbf{\@stopData}\\\whline
	\end{tabular}
	\end{table}
	\vspace*{\stretch{1}}
	\clearpage}
}%


\let\oldmaketitle\maketitle
\renewcommand{\maketitle}{
	\ifthenelse{\equal{\@titleStyle}{presentation}}{\presStyle}{
		\ifthenelse{\equal{\@titleStyle}{pejohesh}}{\pejoheshStyle}{
			\ifthenelse{\equal{\@titleStyle}{lshort}}{\lshortStyle}{
				\ifthenelse{\equal{\@titleStyle}{proposal}}{\proposalStyle}{
					\ifthenelse{\equal{\@titleStyle}{thesis}}{\thesisStyleOO}{
						\ifthenelse{\equal{\@titleStyle}{proposalStyle}}{\proposalStyle}{
							\ifthenelse{\equal{\@titleStyle}{presStyleLogo}}{\presStyleLogo}{
								\oldmaketitle
							}
						} %L6
					} %L5
				}%L4
			}%L3
		}%L2
	}%L1
}%


\makeatother









%%% 0000000000000000000000000000000000000000000000000000000000000000000000000000000000000000000000000000000000000000000000000000000000000000 
%% تنظیم فاصله خطوط در متن اصلی
\setlength{\baselineskipVar}{1cm}
%% تنظیم فاصله خطوط در فهرست‌ها
\setlength{\listlineskipVar}{0.9cm}
%%============================ پاورقی
%% تنظیم‌های مربوط به پاورقی: فاصله پاورقی با متن + فاصله بین خطوط در پاورقی
\setlength{\footnotesep}{0.5cm}
\setlength{\skip\footins}{2cm}

%%%% تعریف یکسری دستور به منظور حل مشکل قرار دادن part در متن و آمدن آن در فهرست مطالب. 
\makeatletter
\renewcommand*\l@part[2]{
  \ifnum \c@tocdepth >-2\relax
    \addpenalty{-\@highpenalty}
    \addvspace{2.25em \@plus\p@}
    \setlength\@tempdima{3em}
    \begingroup
      \parindent \z@ \if@RTL\leftskip\else\rightskip\fi \@pnumwidth
      \parfillskip -\@pnumwidth
      {\leavevmode
       \large \bfseries\textcolor{\keywordstyleColor}{بخش} #1
       \hfil \hb@xt@\@pnumwidth{\hss #2}}\par
       \nobreak
         \global\@nobreaktrue
         \everypar{\global\@nobreakfalse\everypar{}}
    \endgroup
  \fi}
\makeatother

%%\part{part}	     				        -1	not in letters
%%\chapter{chapter}		 	                   0	only books and reports
%%\section{section}		 	                   1	not in letters
%%\subsection{subsection}		         2	not in letters
%%\subsubsection{subsubsection}	         3	not in letters
%%\paragraph{paragraph}		                   4	not in letters
%%\subparagraph{subparagraph}	         5	not in letters

%% این دستور تعیین می‌کند که در متن تا چه عمقی شماره‌گذاری انجام شود. 
\setcounter{secnumdepth}{3}

%%% 0000000000000000000000000000000000000000000000000000000000000000000000000000000000000000000000000000000000000000000000000000000000000000
%% باز تعریف محیط شکل

\makeatletter
\renewenvironment{figure}[1][]{
	\baselineskip = .8cm
	 \ifthenelse{\equal{#1}{}}{
		   \@float{figure}
	 }{%%
		   \@float{figure}[#1]
	 }%%
%% این دستور centering در این قسمت موجب می‌شود که عکس شما در وسط متن قرار گیرد. 
	 \centering
}{%
	 \end@float
}%
\makeatother

%%% 0000000000000000000000000000000000000000000000000000000000000000000000000000000000000000000000000000000000000000000000000000000000000000 
%%============================ تنظیمات مربوط به فونت و اندازه جداول
%% بازنویسی محیط جدول
\makeatletter
\renewenvironment{table}[1][]{
	\ifthenelse{\equal{#1}{}}{\@float{table}}{\@float{table}[#1]}
	\centering
}{%
	\end@float
}%
\makeatother
	
%% بازنویسی محیط tabular به منظور تنظیم فونت‌های جدول
\let\oldtabular\tabular
\let\endoldtabular\endtabular
\renewenvironment{tabular}{
	\bgroup
	\settextfont[Scale=.8]{XM Traffic}
	\setlatintextfont[Scale=.9]{Linux Libertine}
	\oldtabular
}{%
	\endoldtabular 
	\egroup
}%

%% تنظیم کننده فاصله بین خطوط (ردیف‌ها) در یک جدول
\renewcommand{\arraystretch}{1.3}
%% تنظیم کننده ضخامت خطوط جدول
%%\renewcommand{\arrayrulewidth}{.55pt}
%% تنظیم فاصله بین خطوط دو خطه (||) و یا (حالت افقی ||)
%%\renewcommand{\doublerulesep}{1pt}

%%% 0000000000000000000000000000000000000000000000000000000000000000000000000000000000000000000000000000000000000000000000000000000000000000

%% دستوری برای سیاه و سفید کردن متن برای گرفتن پرینت. در این دستور در اولین گام رنگ تمامی ارجاعات که توسط hyperref فعال می‌شوند، سیاه می‌شود. 
\newcommand{\printver}{
	\renewcommand{\keywordstyleColor}{black}
	\hypersetup{linkcolor=black, anchorcolor=black, citecolor=black, urlcolor=black, filecolor=black, pdftoolbar=true}
}%

%%% 0000000000000000000000000000000000000000000000000000000000000000000000000000000000000000000000000000000000000000000000000000000000000000

%% باز تعریف محیط document، هر دستوری که می خواهید در ابتدای برنامه اجرا شود را در این قسمت بنویسید.

%% بازنویسی محیط \begin{document}
\let\olddocument\document
\let\endolddocument\enddocument

\makeatletter
\renewenvironment{document}{
	\olddocument
	%% تنظیم استایل سرفصل
	\chapterstyle{BlueBox}
	\pagestyle{plain}
	%% تنظیم فاصله بین خطوط با دستور \baselineskip
	\baselineskip = \baselineskipVar
	\pagenumbering{arabic}
	\SetWatermarkText{}
}{%
	\endolddocument
}%
\makeatother

%%% 0000000000000000000000000000000000000000000000000000000000000000000000000000000000000000000000000000000000000000000000000000000000000000 


%% محیطی برای قرار دادن abstract در گزارش و یا  در ابتدای هر فصل. در صورت استفاده از این محیط، متون داخل آن با فونتی متفاوت با فونت متن نوشته شده و در ابتدای متن نیز یک کلمه چکیده اضافه می شود. 
\renewenvironment{abstract}{%
	\section*{چکیده}
	\settextfont[Scale=1.2]{XB Shafigh} 
	% \setlatintextfont[Scale=1]{Times New Roman}
	\setlatintextfont[Scale=1]{Linux Libertine}
}{%
	\bigskip
} % M

%% این دستور تعیین می‌کند که چه تا چه عمقی شماره‌گذاری شود. در خود متن نه در فهرست مطالب دقت کنید که برای تعیین این که در فهرست مطالب تا چه عمقی شماره گذاری صورت بگیرد باید از دستور
%% \setcounter{tocdepth}{....}
%% استفاده کرد که در ادامه می آید. 

\setcounter{tocdepth}{3}
%%  تنظیمات مربوط به فهرست مطالب، بازنویسی محیط فهرست مطالب برای تعیین فاصله خطوط، قرار دادن در bookmark ها
\let\Oldtableofcontents\tableofcontents
\renewcommand{\tableofcontents}{
%% این دستور مشخص می‌کند که نحوه شماره‌گذاری این قسمت به صورت یاد شده باشد. 
	\pagenumbering{alph}‎
	\baselineskip = \listlineskipVar
	\Oldtableofcontents\clearpage
	\baselineskip = \baselineskipVar
	\savepagenumber
	\pagenumbering{arabic}
}%

% با این دستور در فهرست مطالب در هنگام آوردن شماره و عنوان فصل در ابتدای آن یک کلمه فصل می گذارد یعنی مثلا می نویسد: (فصل اول: مقدمه ای بر شبکه ..................... ۱)
\renewcommand*{\cftchaptername}{
فصل
\space}%


%%% 0000000000000000000000000000000000000000000000000000000000000000000000000000000000000000000000000000000000000000000000000000000000000000 

%%=========================== تنظمیات محیط فهرست اشکال

%% این دستورات موجب می‌شود که یک تصویر بند انگشتی در فهرست مطالب ظاهر شود. برای این کار می‌بایست در قسمت caption به صورت زیر فایل تصویر نیز وارد شود. 
%% ٍExample: \caption{\lofimage{/Introduction/Person/Shannon} \lr{Claude Shannon}}

\newlength{\lofthumbsize}
\setlength{\lofthumbsize}{2em}

\newif\iflofimage
\DeclareRobustCommand*{\lofimage}[2][]{%
  \iflofimage
    $\vcenter to \lofthumbsize{\vss%
      \hbox to \lofthumbsize{\hss\includegraphics[{width=\lofthumbsize,height=\lofthumbsize,keepaspectratio=true,#1}]{#2}\hss}%
    \vss}$%
    \quad
  \fi
  \ignorespaces
}


%% در دستورات زیر محیط فهرست اشکال باز تعریف شده و اولا این محیط به bookmark اضافه شده و ثانیا مشکل صفحات اضافی حل شده است. ثالثا فاصله خطوط برای زیبایی در این 
%% محیط اندکی کم شده است، ولی دوباره بعد از آوردن این محیط به حالت اولیه برگشته است. در ضمن از شماره گذاری حرفی برای این محیط استفاده شده است. 
\let\Oldlistoffigures\listoffigures
\renewcommand{\listoffigures}{
%% این دستور مشخص می‌کند که نحوه شماره‌گذاری این قسمت به صورت یاد شده باشد. 
	\pagenumbering{alph}
	\restorepagenumber
	\baselineskip = \listlineskipVar
	\cleardoublepage
	\phantomsection
	\lofimagetrue
	\Oldlistoffigures\clearpage
	\lofimagefalse
	\baselineskip = \baselineskipVar
	\savepagenumber
%% با این دستور نحوه شماره‌گذاری به حالت اولیه یعنی (arabic) باز می‌گردد. 
	\pagenumbering{arabic}
}%

%%% 0000000000000000000000000000000000000000000000000000000000000000000000000000000000000000000000000000000000000000000000000000000000000000 

%% در دستورات زیر محیط فهرست جداول باز تعریف شده و اولا این محیط به bookmark اضافه شده و ثانیا مشکل صفحات اضافی حل شده است. ثالثا فاصله خطوط برای زیبایی در این 
%% محیط اندکی کم شده است، ولی دوباره بعد از آوردن این محیط به حالت اولیه برگشته است. در ضمن از شماره گذاری حرفی برای این محیط استفاده شده است. 
\let\Oldlistoftables\listoftables
\renewcommand{\listoftables}{
%% این دستور مشخص می‌کند که نحوه شماره‌گذاری این قسمت به صورت یاد شده باشد. 
	\pagenumbering{alph}
	\restorepagenumber
	\baselineskip = \listlineskipVar
	\cleardoublepage
	\phantomsection
	\Oldlistoftables\clearpage
	\baselineskip = \baselineskipVar
	\savepagenumber
%% با این دستور نحوه شماره‌گذاری به حالت اولیه یعنی (arabic) باز می‌گردد. 
	\pagenumbering{arabic}
}%

%%% 0000000000000000000000000000000000000000000000000000000000000000000000000000000000000000000000000000000000000000000000000000000000000000 

%% دستورات لازم برای واژه‌نامه‌ها، فهرست اختصارات و فهرست نمادها در  فایل (Gloss) آورده شده است.
%%% Data: 2013/04/15    Time: 20:27:46 
%%% تعاریف مربوط به تولید واژه نامه و فهرست اختصارات و فهرست نمادها
%%%  در این فایل یکسری دستورات عمومی برای وارد کردن واژه‌نامه آمده است.
%%%  به دلیل این‌که قرار است این دستورات پایه‌ای را بازنویسی کنیم در این‌جا تعریف می‌کنیم. 
\makeatletter
\let\oldgls\gls
\let\oldGls\Gls
\let\oldglspl\glspl
\let\oldsglspl\sglspl
\let\oldGlspl\Glspl
\let\oldglsuseri\glsuseri
\let\Oldprintglossary\printglossary
\let\oldnewglossaryentry\newglossaryentry

%%% تعریف استایل برای واژه نامه فارسی به انگلیسی، در این استایل واژه‌های فارسی در سمت راست و واژه‌های انگلیسی در سمت چپ خواهند آمد. از حالت گروه ‌بندی استفاده می‌کنیم، 
%%% یعنی واژه‌ها در گروه‌هایی به ترتیب حروف الفبا مرتب می‌شوند، مثلا:
%%% الف
%%% افتصاد ................................... Economy
%%% اشکال ........................................ Failure
%%% ش
%%% شبکه ...................................... Network
\newglossarystyle{mylistFa}{
	\glossarystyle{list}
	\renewenvironment{theglossary}{}{}
	\renewcommand*{\glossaryheader}{}
%%% این دستور یدر حقیقت عملیات گروه‌بندی را انجام می‌دهد. بدین صورت که واژه‌ها در بخش‌های جداگانه گروه‌بندی می‌شوند، 
%%%عنوان بخش همان نام حرفی است که هر واژه در آن گروه با آن شروع شده است. 	
	\renewcommand*{\glsgroupheading}[1]{\vskip 12mm\section*{  \glsgetgrouptitle{##1}} }
	\renewcommand*{\glsgroupskip}{}
%%% در این دستورر نحوه نمایش واژه‌ها می‌آید. در این جا واژه فارسی در سمت راست و واژه انگلیسی در سمت چپ قرار داده شده است، و بین آن با نقطه پر می‌شود. 
	\renewcommand*{\glossaryentryfield}[5]     {\noindent \Glsentryname{##1} \dotfill \space \Glsentryplural{##1} \vskip 0mm}
	\renewcommand*{\glossarysubentryfield}[6]{\glossaryentryfield{##2}{##3}{##4}{##5}{##6}}
}%

%% % تعریف استایل برای واژه نامه انگلیسی به فارسی، در این استایل واژه‌های فارسی در سمت راست و واژه‌های انگلیسی در سمت چپ خواهند آمد. از حالت گروه ‌بندی استفاده می‌کنیم، 
%% % یعنی واژه‌ها در گروه‌هایی به ترتیب حروف الفبا مرتب می‌شوند، مثلا:
%% % E
%%% Economy ............................... اقتصاد
%% % F
%% % Failure................................... اشکال
%% %N
%% % Network ................................. شبکه

\newglossarystyle{mylistEn}{
	\glossarystyle{list}
	\renewenvironment{theglossary}{}{}
	\renewcommand*{\glossaryheader}{}
%%% این دستور یدر حقیقت عملیات گروه‌بندی را انجام می‌دهد. بدین صورت که واژه‌ها در بخش‌های جداگانه گروه‌بندی می‌شوند، 
%%% عنوان بخش همان نام حرفی است که هر واژه در آن گروه با آن شروع شده است. 	
	\renewcommand*{\glsgroupheading}[1]{\vskip 12mm\begin{LTR} \section*{\lr{\glsgetgrouptitle{##1}}} \end{LTR}}
	\renewcommand*{\glsgroupskip}{}
%%% در این دستورر نحوه نمایش واژه‌ها می‌آید. در این جا واژه فارسی در سمت راست و واژه انگلیسی در سمت چپ قرار داده شده است، و بین آن با نقطه پر می‌شود. 
	\renewcommand*{\glossaryentryfield}[5] {\noindent \Glsentryplural{##1}  \dotfill \space  \Glsentryname{##1} \vskip 0mm}
	\renewcommand*{\glossarysubentryfield}[6]{\glossaryentryfield{##2}{##3}{##4}{##5}{##6}}
}%



%%% تعریف استایل برای فهرست اختصارات، در این استایل واژه‌های بازشده واژه مختصر در سمت راست و واژه‌های انگلیسی اختصاری در سمت چپ خواهند آمد. از حالت گروه ‌بندی استفاده می‌کنیم، 
%%% یعنی واژه‌ها در گروه‌هایی به ترتیب حروف الفبا مرتب می‌شوند.
%% % F
%% % FDM................................... Frequency Division Multiplexing 
%% %T
%% % TDM ................................. Time Division Multiplexing

\newglossarystyle{mylistAbbr}{
	\glossarystyle{list}
	\renewenvironment{theglossary}{}{}
	\renewcommand*{\glossaryheader}{}
	\renewcommand*{\glsgroupheading}[1]{\vskip 10mm\begin{LTR} \section*{\lr{\glsgetgrouptitle{##1}}} \end{LTR}}
	\renewcommand*{\glsgroupskip}{}
	\renewcommand*{\glossaryentryfield}[5]     {\noindent \glsentryplural{##1}  \dotfill \space  \Glsentryname{##1}  \vskip 0mm}
	\renewcommand*{\glossarysubentryfield}[6]{\glossaryentryfield{##2}{##3}{##4}{##5}{##6}}
}%


%%% تعریف استایل برای فهرست نمادها، در این استایل ابتدا نام نماد و سپس در کنا آن توضیح نماد خواهد آمد
%%% مثلا: N تعداد گره های شبکه 
\newglossarystyle{mylistNotation}{
	\glossarystyle{list}
	\renewenvironment{theglossary}{}{}
	\renewcommand*{\glossaryheader}{}
	\renewcommand*{\glsgroupheading}[1]{}
	\renewcommand*{\glsgroupskip}{}
	\renewcommand*{\glossaryentryfield}[5]     {\noindent \glsentrysymbol{##1} \glsentrydesc{##1} \vskip 0mm}
	\renewcommand*{\glossarysubentryfield}[6]{\glossaryentryfield{##2}{##3}{##4}{##5}{##6}}
}%


\newglossarystyle{mylistFasubsec}{
	\glossarystyle{list}
	\renewenvironment{theglossary}{}{}
	\renewcommand*{\glossaryheader}{}
	\renewcommand*{\glsgroupheading}[1]{\vskip 12mm\subsection*{  \glsgetgrouptitle{##1}} }
	\renewcommand*{\glsgroupskip}{}
	\renewcommand*{\glossaryentryfield}[5]     {\noindent \Glsentryname{##1} \dotfill \space \Glsentryplural{##1} \vskip 0mm}
	\renewcommand*{\glossarysubentryfield}[6]{\glossaryentryfield{##2}{##3}{##4}{##5}{##6}}
}%


\newglossarystyle{mylistEnsubsec}{
	\glossarystyle{list}
	\renewenvironment{theglossary}{}{}
	\renewcommand*{\glossaryheader}{}	
	\renewcommand*{\glsgroupheading}[1]{\vskip 12mm\begin{LTR} \subsection*{\lr{\glsgetgrouptitle{##1}}} \end{LTR}}
	\renewcommand*{\glsgroupskip}{}
	\renewcommand*{\glossaryentryfield}[5] {\noindent \Glsentryplural{##1}  \dotfill \space  \Glsentryname{##1} \vskip 0mm}
	\renewcommand*{\glossarysubentryfield}[6]{\glossaryentryfield{##2}{##3}{##4}{##5}{##6}}
}%

\newglossarystyle{mylistAbbrsubsec}{
	\glossarystyle{list}
	\renewenvironment{theglossary}{}{}
	\renewcommand*{\glossaryheader}{}
	\renewcommand*{\glsgroupheading}[1]{\vskip 10mm\begin{LTR} \subsection*{\lr{\glsgetgrouptitle{##1}}} \end{LTR}}
	\renewcommand*{\glsgroupskip}{}
	\renewcommand*{\glossaryentryfield}[5]     {\noindent \glsentryplural{##1}  \dotfill \space  \Glsentryname{##1}  \vskip 0mm}
	\renewcommand*{\glossarysubentryfield}[6]{\glossaryentryfield{##2}{##3}{##4}{##5}{##6}}
}%



%%% برای اجرا xindy بر روی فایل .tex و تولید واژه‌نامه‌ها و فهرست اختصارات و فهرست نمادها یکسری  فایل تعریف شده است.‌ Latex داده های مربوط به واژه نامه و .. را در این 
%%%  فایل‌ها نگهداری می‌کند. مهم‌ترین option‌ این قسمت این است که 
%%% عنوان واژه‌نامه‌ها و یا فهرست اختصارات و یا فهرست نمادها را می‌توانید در این‌جا مشخص کنید. 
%%% در این جا عباراتی مثل glg، gls، glo و ... پسوند فایل‌هایی است که برای xindy بکار می‌روند. 
\newglossary[glg]{english}{gls}{glo}{واژه‌نامه انگلیسی به فارسی}
\newglossary[blg]{persian}{bls}{blo}{واژه‌نامه فارسی به انگلیسی}
\newglossary[alg]{abrr}{acr}{acn}{فهرست اختصارات}
\newglossary[nlg]{symbols}{not}{ntn}{فهرست نمادها}
%\newglossary[alg2]{tem}{acr2}{acn2}{موقت}


%%% باز تعریف محیط newacronym برای فهرست اختصارات
\renewcommand{\newacronym}[4]{
	\oldnewglossaryentry{#1}{type = abrr,  name={\lr{#2}} ,plural={\lr{#3}} , description = {}}
}%

%%% باز تعریف محیط newnotation برای فهرست نمادها
\newcommand{\newnotation}[3]{
	\oldnewglossaryentry{#1}{type = symbols,  name={#1} ,plural={#1} , description = {#3},symbol={\lr{#2} }}
}%


%% توسط این دستور محیط newglossaryentry را بازنویسی می‌ کنیم. شما یک واژه به صورت زیر در واژه‌نامه وارد می‌کنید:
%%\newglossaryentry{Fading}
%%{name={Fading}, 
%%plural={محوشدگی},
%%description={}
%%}
%% با بازنویسی این دستور ما با هر بار فراخوانی newglossaryentry دو newglossaryentry با type های مختلف فراخوانی می‌شود، چرا که ما دو واژه‌نامه داریم. 
%\def \myMacro{RRR}
\def\@plpl{plural}
\renewcommand{\newglossaryentry}[2]{%
	\setkeys{glossentry}{#2}%
	\let\thename\@glo@name
	\let\theplural\@glo@plural
	\let\thedesc\@glo@desc
	\ifx\@glo@type\@plpl
		\oldnewglossaryentry{#1}{name={\lr{\thename}} ,plural={\theplural} , description = {\thedesc}}
	\else
		\oldnewglossaryentry{fa-#1}{type = persian,  name={\theplural},  plural={\lr{\thename}} , description={\thedesc}}
		\oldnewglossaryentry{#1}{type = english,  name={\lr{\thename}} ,plural={\theplural} , description = {\thedesc}}	
	\fi	
}%

\makeglossaries
\glsdisablehyper
\makeindex

%%% با دستورات زیر نحوه گذاشتن واژه در متن مشخص می‌شود. 
\renewcommand*{\SetCustomDisplayStyle}[1]{%
%%% این دستور مشخص می‌کند که برای اولین بار که واژه در متن ظاهر می شود چه اتفاقی بیافتد. مثلا در این‌جا تعریف شده است که برای اولین بار معادل انگلیسی پاورقی شود. 
	\defglsdisplayfirst[#1]{##1\protect\hspace{-.1em}\LTRfootnote{\Glsentryname{\glslabel} }}
%%% این دستور مشخص می‌کند که بار دوم و سوم و ... که واژه در متن ظاهر می شود چه اتفاقی بیافتد. مثلا در این‌جا تعریف شده است که اتفاق خاصی نیافتد. 
	\defglsdisplay[#1]{##1}%
}%
%%% این دستور می‌گوید که استایلی که در بالا برای نحوه قرار گرفتن واژه‌ها در متن بیان شد، به چه type از واژه‌هایی اعمال شود. 
\SetCustomDisplayStyle{english}

%%% همان مواردی که در بالا اشاره شد، اما این بار برای type واژگان اختصاری
\renewcommand*{\SetCustomDisplayStyle}[1]{%
	\defglsdisplayfirst[#1]{\lr{##1\protect\LTRfootnote{\Glsentryplural{\glslabel} }}}
	\defglsdisplay[#1]{##1}%
}
%%% این دستور می‌گوید که استایلی که در بالا برای نحوه قرار گرفتن واژه‌ها در متن بیان شد، به چه type از واژه‌هایی اعمال شود. 
\SetCustomDisplayStyle{abrr}

%%% توسط این دستور واژه مورد نظر در متن، هر دو واژه نامه و پاورقی می آید.
%%%  این دستور برای مواردی استفاده می‌شود که واژه‌ها را در فایل
%%% واژه‌نامه (\begin{document}) وارد نکرده‌ایم، و می‌خواهیم در متن هم واژه‌ها را وارد کنیم و هم پاورقی کنیم. 
\newcommand{\inpdic}[2]{\newglossaryentry{#2}{name={#2}, plural={#1},description = {}}\glsplp{#2}}

%%% توسط این دستور واژه مورد نظر در متن، هر دو واژه نامه  می آید.
 %%%  این دستور برای مواردی استفاده می‌شود که واژه‌ها را در فایل
%%% واژه‌نامه (\begin{document}) وارد نکرده‌ایم، و می‌خواهیم در متن هم واژه‌ها را وارد کنیم ولی در پاورقی وارد نمی‌کند. 
\newcommand{\indic}[2]{\newglossaryentry{#2}{name={#2}, plural={#1},description = {} }\glspl{#2}}

%%% توسط این دستور واژه مورد نظر در متن، هر دو واژه نامه  می آید.
 %%%  این دستور برای مواردی استفاده می‌شود که واژه‌ها را در فایل
%%% واژه‌نامه (\begin{document}) وارد نکرده‌ایم، با این دستور نه واژه‌ها وارد متن می‌شود و نه در پاورقی می‌آید. 
\newcommand{\ingls}[2]{\newglossaryentry{#2}{name={#2}, plural={#1},description = {} }\glsuseri{#2}}
%%% توسط این دستور اختصارات مورد نظر در متن، در فهرست اختصارات می‌آید. 
 %%%  این دستور برای مواردی استفاده می‌شود که اختصارات‌ را در فایل
%%% ‌اختصارات قبل از \begin{document} وارد نکرده‌ایم،
%% توسط این دستور می‌توانید در داخل متن یک واژه به عنوان اختصارات وارد کنیم. با این دستور هم اختصار وارد متن می‌شود، هم پاورقی و هم به فهرست اختصارات اضافه می‌شود. 
\newcommand{\inpabr}[2]{\oldnewglossaryentry{#1}{type = abrr,  name={\lr{#1}} ,plural={\lr{#2}} , description = {}}\oldgls{#1}\LTRfootnote{#2}}

%%% توسط این دستور اختصارات مورد نظر در متن، در فهرست اختصارات می‌آید. 
 %%%  این دستور برای مواردی استفاده می‌شود که اختصارات‌ را در فایل
%%% ‌اختصارات قبل از \begin{document} وارد نکرده‌ایم،
%% توسط این دستور می‌توانید در داخل متن یک واژه به عنوان اختصارات وارد کنیم. با این دستور هم اختصار وارد متن می‌شود،  به فهرست اختصارات اضافه می‌شود. ولی پاورقی نمی‌شود. 
\newcommand{\inabr}[2]{\oldnewglossaryentry{#1}{type = abrr,  name={\lr{#1}} ,plural={\lr{#2}} , description = {}}\oldgls{#1}}


%%% بازتعریف برخی از دستورات استاندارد بسته glossaries به منظور استفاده در متن. 
%%% این دستور برای زمانی است که واژه ها، اختصارات و نمادها را در یک فایل جداگانه تعریف کرده باشیم. 


\renewrobustcmd*{\glspl}{\@ifstar\@msglspl\@mglspl}
\newcommand*{\@mglspl}[1] {\ifthenelse{\equal{\glsentrytype{#1}}{english}}{\oldglspl{#1}\oldglsuseri{fa-#1}}{\oldglspl{#1}}}
\newcommand*{\@msglspl}[1]{\ifthenelse{\equal{\glsentrytype{#1}}{english}}{\glsentryplural{#1}\oldglsuseri{#1}\oldglsuseri{fa-#1}}{\oldglspl{#1}}}

\renewrobustcmd*{\gls}{\@ifstar\@msgls\@mgls}
\newcommand*{\@mgls}[1] {\ifthenelse{\equal{\glsentrytype{#1}}{english}}{\oldgls{#1}\oldglsuseri{fa-#1}}{\oldgls{#1}}}
\newcommand*{\@msgls}[1]{\ifthenelse{\equal{\glsentrytype{#1}}{english}}{\glsentryname{#1}\oldglsuseri{#1}\oldglsuseri{fa-#1}}{\glsentryname{#1}\oldglsuseri{#1}}}

\renewrobustcmd*{\glsuseri}{\@ifstar\@msglsuseri\@mglsuseri}
\newcommand*{\@msglsuseri}[1] {\ifthenelse{\equal{\glsentrytype{#1}}{english}}{\oldglsuseri{#1}\oldglsuseri{fa-#1}}{\oldglsuseri{#1}}}
\newcommand*{\@mglsuseri}[1]   {\ifthenelse{\equal{\glsentrytype{#1}}{english}}{\oldglsuseri{#1}\oldglsuseri{fa-#1}}{\oldglsuseri{#1}}} 

\newcommand{\glsplp}[1]{\ifthenelse{\equal{\glsentrytype{#1}}{english}}{\glsentryname{fa-#1}\hspace{-.1em}\LTRfootnote{\Glsentryname{#1}}\oldglsuseri{#1}}{\ifthenelse{\equal{\glsentrytype{#1}}{abrr}}{\glsentryplural{#1}}{\glsentryplural{#1}\hspace{-.1em}\LTRfootnote{\Glsentryname{#1}}}}}%

\newcommand{\glsp}[1]{\ifthenelse{\equal{\glsentrytype{#1}}{english}}{\glsentryname{#1}\footnote{\glsentryplural{#1}}\oldglsuseri{fa-#1}}{\ifthenelse{\equal{\glsentrytype{#1}}{abrr}}{\lr{\glsentryname{#1}\LTRfootnote{\glsentryplural{#1}}}}{\ifthenelse{\equal{\glsentrytype{#1}}{symbols}}{\oldglsuseri{#1}\footnote{\glsentryplural{#1}}\glsentrysymbol{#1}}{}}}}%


\newcommand{\glsabr}[1]{\gls{#1}} 
\newcommand{\glsno}[1]{\oldglsuseri{#1}\glsentrysymbol{#1}}

\makeatother




%%% 0000000000000000000000000000000000000000000000000000000000000000000000000000000000000000000000000000000000000000000000000000000000000000 

%%============================ دستور برای قرار دادن فهرست اختصارات 
\newcommand{\printabbreviation}{
%% این دستور مشخص می‌کند که نحوه شماره‌گذاری این قسمت به صورت یاد شده باشد. 
	\pagenumbering{alph}
	\restorepagenumber
	\baselineskip=.75cm
	\glossarystyle{mylistAbbr}
	\cleardoublepage
	\phantomsection
%% با این دستور عنوان فهرست اختصارات به فهرست مطالب اضافه می‌شود. 
	\addcontentsline{toc}{chapter}{فهرست اختصارات}
	\Oldprintglossary[type=abrr]	
	\clearpage
	\baselineskip = \baselineskipVar
	\savepagenumber
%% با این دستور نحوه شماره‌گذاری به حالت اولیه یعنی (arabic) باز می‌گردد. 
	\pagenumbering{arabic}
}%

%%% 0000000000000000000000000000000000000000000000000000000000000000000000000000000000000000000000000000000000000000000000000000000000000000 

\newcommand{\printnotation}{
	\pagenumbering{harfi}
	\restorepagenumber
	\baselineskip = \listlineskipVar
	\cleardoublepage
	\phantomsection
%% با این دستور عنوان فهرست نمادها به فهرست مطالب اضافه می‌شود. 
	\addcontentsline{toc}{chapter}{فهرست نمادها}
	\glossarystyle{mylistNotation}
	\Oldprintglossary[type=symbols]
	\clearpage
	\baselineskip = \baselineskipVar
	\savepagenumber
%% با این دستور نحوه شماره‌گذاری به حالت اولیه یعنی (arabic) باز می‌گردد. 
	\pagenumbering{arabic}
}%

%%% 0000000000000000000000000000000000000000000000000000000000000000000000000000000000000000000000000000000000000000000000000000000000000000 

\let\Oldbibliography\bibliography
\renewcommand{\bibliography}[1]{
	\let\appendix\relax
	\baselineskip=.5cm
%% با این دستور عنوان این قسمت به (مراجع) تغییر پیدا می‌کند. 
	\renewcommand{\bibname}{مراجع}
	\clearpage
	\phantomsection
	\bibliographystyle{ieeetr-fa}
	\Oldbibliography{#1}
}%

%%% 0000000000000000000000000000000000000000000000000000000000000000000000000000000000000000000000000000000000000000000000000000000000000000 

%%% در این جا محیط هر دو واژه نامه را باز تعریف کرده ایم، تا اولا مشکل قرار دادن صفحه اضافی را حل کنیم، ثانیا عنوان واژه نامه ها را با دستور addcontentlist وارد فهرست مطالب کرده ایم.

\renewcommand{\printglossary}{
	\let\appendix\relax
	\baselineskip=.75cm
	\clearpage
	\phantomsection
%% این دستور موجب این می‌شود که واژه‌نامه‌ها در  حالت دو ستونی نوشته شود. 
	\twocolumn{}
%% با این دستور عنوان واژه‌نامه به فهرست مطالب اضافه می‌شود. 
	\addcontentsline{toc}{chapter}{واژه نامه انگلیسی به فارسی}
	\glossarystyle{mylistEn}
	\Oldprintglossary[type=english]
	
	\clearpage
	\phantomsection
%% با این دستور عنوان واژه‌نامه به فهرست مطالب اضافه می‌شود. 
	\addcontentsline{toc}{chapter}{واژه نامه فارسی به انگلیسی}
	\glossarystyle{mylistFa}
	\Oldprintglossary[type=persian]
	\onecolumn{}
	\baselineskip = \baselineskipVar
}%

%%% 0000000000000000000000000000000000000000000000000000000000000000000000000000000000000000000000000000000000000000000000000000000000000000 

% دستوری برای رنگ‌آمیزی محیط  Item در dinglist
\newcommand{\itemcolor}[1]{\renewcommand{\makelabel}[1]{\color{#1}\hfil ##1}}
%%\begin{dinglist}{110}
%%  	\itemcolor{myblue}
%%  	\item Sample text
%%    	\itemcolor{green!70}
%%    	\item Sample text
%%    	\itemcolor{red!50}
%%\end{dinglist}






%%% 
%%% 
%%% %%% نام  قالب را تعیین می کند و همچنین بیان می کند که آخرین به روز رسانی  این قالب
%%% در چه زمانی بوده است. یک توصف مختصر هم از این بسته در اینجا امده است.
\ProvidesClass{Boostan-Thesis}
%%% تمام پارامترهای ورودی برای تنظیم متن را به کلاس زیر ارسال می‌کنیم
\DeclareOption*{\PassOptionsToClass{\CurrentOption}{Boostan-UserManual}}
\ProcessOptions 
\LoadClass[oneside]{Boostan-UserManual}


%%% AAAAAAAAAAAAAAAAAAAAAAAAAAAAAAAAAAAAAAAAAAAAAAAAAAAAAAAAAAAAAAAAAAAAAA


%%% برای تنظیم حاشیه صفحات
%\geometry{top=3cm, bottom=2.5cm, left=2cm, right=2.5cm}

%%  با دستور زیر می توانید فونتی مخصوص عبارات فارسی تعیین کنید:
\settextfont[Scale=1.3]{B Nazanin} 
%%\settextfont[Scale=1.2]{XB Niloofar}
%% شما با دستور زیر بعد از فراخوانی بسته xepersian می توانید فونت انگلیسی را تعیین کنید
%% دقت کنید که عبارات انگلیسی شما باید در دستور \lr{} قرار گیرد تا xepersian بتواند بفمهد که این عبارات انگلیسی است
% \setlatintextfont[Scale=1]{Times New Roman}
\setlatintextfont[Scale=1.1]{Linux Libertine}

% % تعریف برای فونت اعداد و ارقام
%\setdigitfont[Scale=1.1]{XB Zar}

%% توسط دستورات defpersianfont و deflatinfont به ترتیب می توان یکسری فونت فارسی و انگلیسی دیگر تعریف کرد که در جاهای دیگر متن بتوان از آن استفاده کرد. برای استفاده کافی است که عبارتی که می خواهیم فونت آن عوض شود را به صورت زیر به عنوان نمونه بنویسیم.
%% \versionfont{این یک مثال است. }

%% تعریف یکسری فونت برای قسمت عنوان پروژه و ما بقی قسمت ها فونت قسمت "موسسه " در صفحه عنوان

\defpersianfont\tablefont[Scale=.8]{XM Traffic}
\defpersianfont\pejoheshfont[Scale=1.4]{Titr}
%%فونت اسم گروه XB Titre
\defpersianfont\groupfont[Scale=1.4]{XB Zar}
%%% فونت عنوان گزارش
\defpersianfont\titlefont[Scale=2.4]{Titr}
%% فونت نسخه گزارش
\defpersianfont\versionfont[Scale=1.6]{B Mitra}
\defpersianfont\payanFont[Scale=1.8]{XB Yas}
\defpersianfont\nastaliq[Scale=2]{IranNastaliq}
\defpersianfont\farsifontshafigh[Scale=1.3]{XB Shafigh}
\defpersianfont\titrt[Scale=1]{XB Titre}
\defpersianfont\traffict[Scale=1]{XM Traffic}
\defpersianfont\farsifontsayeh[Scale=1.5]{XB Kayhan Sayeh}
\defpersianfont\titlefont[Scale=2.4]{Titr}

%% فونت‌های مورد نیاز برای صفحه شروع 
\defpersianfont\logofontR[Scale=1.2]{XB Zar}
\defpersianfont\typefontR[Scale=1.3]{B Nazanin}
\defpersianfont\titlefontR[Scale=2]{Titr}
\defpersianfont\datafontR[Scale=1.4]{XB Zar}

\deflatinfont\tableFontEn[Scale=.9]{XB Shafigh}

%%  با استفاده از این دستور می‌توان فونت و فارسی و یا انگلیسی بودن اعداد در فرمول‌ها را به حالت اولیه (یعنی پیش‌فرض لاتک) برگرداند.
\DefaultMathsDigits

%\DeclareMathSizes{textsize}{mathsize}{scriptsize}{scriptscriptsize}
% گزینه اول: این برای چه دسته فونتی است. پیش فرض استایل ما فونت 10pt است. 
% گزینه دوم: اندازه فونت توابع و موجودات ریاضی درون متن.
% گزینه سوم: برای اسکریپت ها، اندازه زیرنویس و بالانویس.
% گزینه چهارم: برای زیرنویس زیرنویس.

% در دستورات زیر ما برای سه حالت، اندازه‌های مورد نظر را تعریف کرده ایم. 
\DeclareMathSizes{10}{11}{9}{8}   % For size 10 text
\DeclareMathSizes{11}{12}{11}{10}   % For size 11 text
\DeclareMathSizes{12}{13}{12}{11}  % For size 12 text

%%%%%%%%%%%%%%%%%%%%%%%%%%%%%%%%%%%%%%%%%%%%%%%%%%%%%%%%%%%
%%%%%%%%%%%%%%%%%%%%%%%%%%%%%%%%%%%%%%%%%%%%%%%%%%%%%%%%%%%

%%% قالب صفحه و انتخاب حاشیه ها

%%% برای تعیین اندازه صفحه ابتدا باید اندازه stock و page را با دو دستور زیر تعیین کنیم. برای کارهای ما که هر صفحه ی خروجی در یک صفحه پرینت می شود، این دو مقدار به صورت مساوی مقدار دهی می شود. 
%%% در دو دستور زیر ما مقادیر استاندارد برای کاغذ A4 را قرار داده ایم.
\setlength{\headsep}{30pt}
%%%%\setstocksize{29.7cm}{21cm}
%%%%\settrimmedsize{29.7cm}{21cm}{*}
%%%\setlength{\trimtop}{0pt}
%%%\setlength{\trimedge}{\stockwidth}
%%%\addtolength{\trimedge}{-\paperwidth}
%%% در مرحله ی بعدی باید ابعادمتن در کاغذ را مشخص کنیم. 
%%%%\settypeblocksize{24.2cm}{16cm}{*}
%%% تعیین کننده حاشیه از بالا (آرگومان اول) و یا از پایین (آرگومان دوم) و یا نسبی (آرگومان سوم)
%%% دقت شود برای جلوگیری از ابهام تنها باید یکی از موارد فوق داده شود.
%%%%\setulmargins{2.5cm}{*}{*}
%%% به مانند دستور بالا، حاشیه از سمت چپ (آرگومان اول) و ...
%%%%\setlrmargins{2cm}{*}{*}
%%%\setmarginnotes{10pt}{10pt}{\onelineskip}
%%%%\setheadfoot{\onelineskip}{2\onelineskip}
%%%\setheaderspaces{*}{2\onelineskip}{*}
%%%%\checkandfixthelayout



%% محیطی برای قرار دادن abstract در گزارش و یا  در ابتدای هر فصل. در صورت استفاده از این محیط، متون داخل آن با فونتی متفاوت با فونت متن نوشته شده و در ابتدای متن نیز یک کلمه چکیده اضافه می شود. 


\let\oldabstract\abstract
\let\endoldabstract\endabstract

\renewenvironment{abstract}{
\section*{چکیده}
\settextfont[Scale=1.2]{XB Shafigh} 
% \setlatintextfont[Scale=1]{Times New Roman}
\setlatintextfont[Scale=1]{Linux Libertine}
}{\\*[8pt]}





\newcommand{\entitle}{Fullerene graphs with exponentially many perfect matchings}
\newcommand{\enAuthor}{Farhad Mehrvarzi}
\newcommand{\ensupervisor}{Prof. H. Yousefi Azari}
\newcommand{\engdate}{February 2011}
\newcommand{\enlevel}{{A thesis submitted to the Graduate Studies Office
\\In partical fulfillment of the requirements for}}
\newcommand{\enmajor}{ The degree of Master in thesis \\Applied Mathematics}
\newcommand{\enDep}{University College of Science\\ School of Mathematics, Statistics and\\ Computer Science}



\gdef\@englishtitle{Your Thesis Title} 
\def\englishtitle#1{\gdef\@englishtitle{#1}}


\gdef\@englishAuthor{Moh Da} 
\def\englishAuthor#1{\gdef\@englishAuthor{#1}}


\gdef\@englishsupervisor{Dr} 
\def\englishsupervisor#1{\gdef\@englishsupervisor{#1}}

\gdef\@englishadvisor{Dr} 
\def\englishadvisor#1{\gdef\@englishadvisor{#1}}



\gdef\@englishDate{Aug 2012} 
\def\englishDate#1{\gdef\@englishDate{#1}}




\newcommand{\thesisInfo}{

 \clearpage\newpage
 \thispagestyle{empty}
\begin{center}
\large\bfseries به نام او
\end{center}
\vspace*{\stretch{3}}
\noindent 
پایان نامه جهت دریافت \@farsilevel
 \\*[20pt] 
عنوان: \@farsititle
 \\*[20pt] 
نگارش: \@farsiAuthor  
 \\*[20pt]
استاد راهنما:  \@farsisupervisor
 \\*[20pt]

}

%%%%%%%%%%%%%%%%%%%%%%%%%%%%%%%%%%%%%%%%%%%%%%%%%%%%%%%%%%%%%%%
%%%%%%%%%%%%%%%%%%%%%%%%%%%%%%%%%%%%%%%%%%%%%%%%%%%%%%%%%%%%%%%




%%%%%%%%%%%%%%%%%%%%%%%%%%%%%%%%%%%%%%%%%%%%%%%%%%%%%%%%%%%%%%


\makeatletter
\defpersianfont\chaptertitlefont[Scale=1.6]{B Titr}

\newcommand\mycustomraggedright{%
 \if@RTL\raggedleft%
 \else\raggedright%
 \fi}
\def\@part[#1]#2{%
\ifnum \c@secnumdepth >-2\relax
\refstepcounter{part}%
\addcontentsline{toc}{part}{\thepart\hspace{1em}#1}%
\else
\addcontentsline{toc}{part}{#1}%
\fi
\markboth{}{}%
{\centering
\interlinepenalty \@M
\ifnum \c@secnumdepth >-2\relax
 \huge\bfseries \partname\nobreakspace\thepart
\par
\vskip 20\p@
\fi
\LARGE\bfseries #2\par}%
\@endpart}
\def\@makechapterhead#1{%
  \vspace*{200\p@}%
  {\parindent \z@ \raggedleft \normalfont
    \ifnum \c@secnumdepth >\m@ne
      \if@mainmatter
        \huge\bfseries \@chapapp\space \thechapter
        \par\nobreak
        \vskip 20\p@
      \fi
    \fi
    \interlinepenalty\@M
    \Huge \bfseries \raggedleft{ #1}\par\nobreak
    \vskip 50\p@
  }}

%اگه می‌خواین که کلمه «فصل» رو هم داشته باشین، خط پایین رو غیرفعال کنین.
%\renewcommand{\chaptername}{}
%  نکته جانبی و بی‌ربط به این بحث: اگه می‌خواین که صفحات اول هر فصل، شماره صفحه نداشته باشن، ۹ خط پایین رو فعال کنین.
%\let\origchapter\chapter
%\renewcommand*{\chapter}{% 
%  \fancypagestyle{plain}{%
%    %\fancyhf{}%
%    \renewcommand{\headrulewidth}{0pt}%
%    \renewcommand{\footrulewidth}{0pt}%
%  }%
%\origchapter
%}

\makeatother

\nouppercaseheads
\makepagestyle{mystyle}
\makeevenhead{mystyle}{}{}{\itshape\leftmark\vskip -2mm}
\makeoddhead{mystyle}{}{}{\itshape\leftmark\vskip -2mm}
\makeheadrule{mystyle}{ \textwidth }{.8pt}
\makeevenfoot{mystyle}{}{\thepage}{}
\makeoddfoot{mystyle}{}{\thepage}{}
%\makepsmarks{mystyle}{%
%  \createmark{chapter}{left}{nonumber}{}{}}






\makeatletter
\newcommand{\englishcover}{
\clearpage
\thispagestyle{empty}

\begin{latin}
	\begin{center}
	\begin{table}
	\begin{tabular}{ccc}
	\includegraphics[width=.15\textwidth]{Pic/logo}&
	\begin{minipage}{0.55\linewidth}
	\vskip 0.5cm
	\begin{center}\Large
	{\Large
	University of Tehran}\\* [0.2cm]
	Faculity of Engineering\\*[0.2cm]
	Epartment of Electical and\\*[0.2cm]
	Computer Engineering\\*[0.2cm]
	\end{center}
	\end{minipage}
	&
	\includegraphics[width=.15\textwidth]{Pic/logoF}
	\end{tabular}
	\end{table}
		\vskip 2cm
		\huge{\@englishtitle}
		\vskip 2cm
		\Large{By :}         \\ \Large\textit{{\@englishAuthor}}
		\vskip 1.5cm
		\Large{Supervisor :} \\ \Large\textit{{\@englishsupervisor}}
		\vskip 1.5cm
		\Large{Consulating Advisor :} \\ \Large\textit{{\@englishadvisor}}
		\vskip 1.5cm
		\large{Thesis submitted to the Graduate Studies Office in partial fulfillment of the}\\*[5pt]
		\large{requirements for the degree of }\\*[5pt]
		\large{Master of Science in Computer Engineering - Master Intelligence and Robotics,}
		\vskip 1cm
		\Large{\@englishDate}
\end{center}
\end{latin}
}
\makeatother
























