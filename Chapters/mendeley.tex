\chapter{
آشنایی با نرم‌افزار \lr{Mendeley}}
\section{مقدمه}
در این فصل قصد دارم در مورد نرم افزار \lr{Mendeley} سخن به میان آورم. اگر بخواهیم در یک جمله عملکرد این نرم افزار را بیان کنیم، می بایست بگوییم که \lr{Mendeley} نرم افزار مدیریت مقالات و متون علمی است. قابلیت هایی که این نرم افزار برای ما فراهم می کند، بی نظیر است، و به نظر من از کارکردن با آن ضرر نخواهید کرد. اجازه بدهید تا در ادامه مرور مختصری بر قابلیت های این نرم افزار داشته باشیم. 

با استفاده از این نرم افزار می توانید به سادگی مقالات خود را منظم و دسته‌بندی شده در اختیار داشته باشید. این نرم افزار امر ارجاع به یک مقاله در \lr{Word}  و \lr{Latex} را بسیار ساده کرده است، و  بسیاری از قابلیت‌های دیگر که در ضمن این فصل بدان ها اشاره می شود.

\section{بازگشایی حساب کاربری و دانلود نرم افزار }
این نرم افزار به صورت رایگان است. به سایت \url{http://www.mendeley.com} بروید. ابتدا یک حساب کاربری برای خود ایجاد کنید. برای این کار پست الکترونیکی خود را وارد کنید و به چند سوال کوتاه آن جواب بدهید. 

بعد از چند لجظه یک پست الکترونیکی به سبد پست شما وارد می شود. با بازگشایی آن بر روی لینک موجود کلیک کرده تا بتوانید حساب خود را فعال کنید. بعد از کلیک کردن لینک صفحه ای باز می شود، و از شما می خواهد رمز دلخواه خود را وارد کرده و سپس بر روی گزینه \lr{Activate} کلیک کنید. اکنون حساب شما فعال می شود. بعد از چند لحظه لینک دانلود نرم افزار به شما داده می شود.

\section{اقدامات اولیه}
بعد از نصب برنامه، به قسمت $tools\textrightarrow option\textrightarrow general$ رفته و پست الکترونیکی و رمز خود را وارد نمایید. دقت نمایید که برای کارکردن نرم افزار این کار ضروری است.

در ادامه در صفحه ی اصلی نرم افزار می توانید بر روی دکمه ی \lr{Add Documents} کلیک کنید. سپس شاخه‌ای که مقالات و متون علمی شما در آن است را انتخاب کنید.

مدت زمانی طول می کشد تا نرم افزار مقالات مورد نظر شما به داخل نرم افزار \lr{import} کند. نرم افزار قابلیت دیده بانی نیز برای شما قرار داده است. مثلا فرض کنید که من تنظیم کرده ام که \lr{Download manager} من فایل هایی را که از اینترنت می گیرد به صورت پیش فرض به پوشه \lr{ C:\Users\Abolfazl\Documents\Downloads} انتقال دهد. شما می توانید به نرم افزار بفهمانید که همواره محتویات این شاخه را دیده بانی کند، و هرگاه فایلی به آن اضافه شد، به صورت خودکار وارد نرم افزار کند. بدین منظور به قسمت  بروید و شاخه ی مورد نظر را به نرم افزار شناسایی کنید.

در ضمن شما می‌توانید به صورت \lr{drug and drop} فایل‌های خود را به نرم‌افزار اضافه کنید. 


آشنایی با نرم افزار
صفحه ی اصلی نرم افزار به صورت زیر است.



پنجره ی وسطی مقالات را نمایش می دهد. با کلیک بر روی هر مقاله جزئیات آن در پنجره ی سمت راست نمایش داده می شود. در حقیقت نرم افزار سعی می کند تا اطلاعات مقاله ی انتخاب شده از مقاله بیرون بکشد. به هر علتی ممکن است، مقاله اطلاعاتش نباشد. در این هنگام شما می توانید بر روی گزینه ی search by title  کلیک کنید. نرم افزار به طور خودکار در اینترنت جستجو می کند و اطلاعات مقاله ی شما را کامل می کند.

البته دقت کنید که می توانید به صورت دستی نیز خود،  اطلاعات مقاله را تکمیل کنید. 
در گوشه ی پایین سمت چپ صفحه ی اصلی، می توانید بر حسب نام نویسنده، کلمات کلیدی و ... مقالات خود را جدا بکنید. 

یکی از قابلیت های بسیار خوب این نرم افزار این است که می تواند مقالات انتخابی شما را به صورت منظم دسته بندی کرده و در پوشه ای مجزا قرار دهد. برای این کار به قسمت  بروید.


در قسمت اول می توانید مکانی که قرار است فایل های منظم شده در آن قرار گیرد را برای نرم افزار مشخص کنید. در قسمت دوم sort files into subfolder می توانید مشخص کنید که فایل ها را چگونه در پوشه های مجزا بچیند. نام پوشه ها در قمست folder path مشخص می شود. فرض کنید من می خواهم مقالاتم را در پوشه های مجزا بر حسب سال مرتب کند. لذا دو گزینه ی author  و journal را با drag کردن به قسمت unused field انتقال می دهم، و از همان قسمت year را وارد قسمت folder path می کنم. 
در قسمت سوم rename document file می توانید  برای نرم افزار تعیین کنید که نام فایل شما چگونه باشد. برای مثال در شکل زیر من تعیین کردم که ابتدا عنوان، سپس نویسنده و سپس سال مقاله را بیاورد. 

نتیجه کار در پوشه ی مشخص شده به صورت زیر است:


ویژگی ممتاز دیگر این نرم افزار این است که شما می توانید مقالات خود را در دسته بندی های مجزا (collection) دسته بندی کنید. بدین منظور کافی است بر روی گزینه ی create collection کلیک کرده و نام collection را وارد کنید. برای وارد کردن مقالات کافی است تنها مقالات را به collection مورد نظر خود با drag کردن فایل آن اضافه کنید.


یک ویژگی بسیار جالب و زیبای این نرم افزار وارد کردن ارجاعات یک مقاله به درون نرم افزار word و یا Latex است. مثلا فرض کنید ما این کار را می خواهیم برای نرم افزار word انجام دهیم. ابتدا از قسمت tool گزینه ی install ms word pluging را بزنید. سپس به فایل word خود مراجعه کنید. 
فرض کنید در قسمتی از فایل خود می خواهم به مقاله ای ارجا بدهم. ابتدا از گزینه ی Add-Ins که در منوهای بالایی word است. گزینه ی insert citation را انتخاب کنید. 

بعد از این کار به طور خودکار صفحه ی نرم افزار بر روی شما باز می شود. شما مقاله ی خود را انتخاب کنید و از گزینه های بالای نرم افزار گزینه ی  send citation to word را انتخاب کنید. دوباره به word باز گردید. از گزینه های Add-Ins شما می توانید استاندارد مرجه نویسی خود را انتخاب کنید. مثلا من این مورد را بر روی IEEE  تنظیم کردم.



سپس در مکان مناسب (یعنی بخش مراجع) بر روی گزینه ی insert bibliography را کلیک کنید. بر طبق استانداردی که انتخاب کردید، مرجع را وارد می کند. مثلا برای IEEE یک شماره به صورت [1] در متن وارد کرده و سپس در پایان نیز مرجه را می آورد به صورت زیر:
[1] 	 V. Shoup, "Practical threshold signatures," Advances in Cryptology—EUROCRYPT 2000, 2000. 
شما می توانید در جای جای متن خود مراجع بیشتری را وارد کنید. ولی نگران شماره گزاری نباشید. خود نرم افزار به صورت خودکار شماره ها را بروز می کنید و تنظیمات لازم را انجام می دهد. دقت کنید برای بار دوم که مقاله ای را وارد می کنید دیگر نیازی نیست insert bibliography را بزنید. در همان مکانی که در اول مراجع را وارد کردید، مرجع جدید به همان جا اضافه می شود. 
نرم افزار چنین قابلیتی را نیز برای Latex نیز فراهم آورده است. 
نم افزار قابلیت بازگشایی فایل های پی دی اف، اصلاح آن، قرار دادن Note در آن، مهر خواندن و یا نخواندن و ... بر روی آن و بسیاری قابلیت های دیگر را نیز دارا است. 

یکی دیگر از قابلیت های بی نظیر این نرم افزار این است که تا 500 مگابایت فضا بر روی سرور خودش به صورت مجانی به شما می دهد. شما می توانید مقالات خود را به آن جا انتقال دهید. اکنون جایی بروید که کامپیوتر شخصی خودرا به همراه خود نبردید، می توانید از فایل های بر روی سرور استفاده کنید، تغییرات صورت گرفته به طور خودکار زمانی که کامپیوتر شخصی شما به اینترنت متصل باشد، بر روی فایل های شما در کامپیوتر شخصی خودتان اعمال می شود.
نکته ی بسیار مهم دیگر این است شما می توانید مجموعه ی مقالات خود را با دیگران به اشتراک گذاشته و یا از مجموعه مقالات دیگران استفاده کنید.

با تشکر

