\thispagestyle{empty} %prevents tex from numbering of this page
\begin{latin} % xepersian enviorment

\centerline{\textbf{\large{Abstract}}}
\vskip 1cm

Steganography is the art and science of writing hidden messages in such a way that no one, apart from the sender and intended recipient, suspects the existence of the message, a form of security through obscurity.

In this thesis, we focus on  entropic issue of multimedia signal in the two branches of Information hiding namely Steganography and  Watermarking. How to choose the block and noise estimation in the watermarking, and analysis of the singular values decomposition in steganography are examples of using entopic issue which we use in our thesis. 

The two new designs for video signals AVI are presented in Watermarking. For the both proposed method ,first AVI video signal will be divided into several part and each part  will be divided into several three-dimensional block. 3D block with maximum entropy is chosen, then the watermarks are embedded in the low-frequency wavelet transform coefficients of 3D chosen block. Inthe first proposed method, the sender must send some statistical data of AVI signal through a secure channel to the receiver. The receiver uses these information to detect Watermark using Maximum Likelihood scheme. Inthe second proposed method, the receiver do not need to statistical information of signal. We evaluate  the performance of the proposed method with theory and simulation.

In the Steganalysis, we suppose  LSB as steganography method. We focus on singular value decomposition which shows the entropy of the our image signal. If a message is added to the signal, the entropy of the signal increase. Amount of entropy related to the main content of the message signal. We try to estimate the singular values ​​to detect which signal is a  clean or stego. We can also use this method as approximation of  Steganography rate. Support Vector Machine and the noise estimation of the image signal are two tools in order to achieve a better and more accurate steganalysis. To better evaluate the proposed method, a comprehensive database of images is used in the simulations.

Keywords: Steganography, Steganalysis, Watermarking, Blind Watermarking, semi blind Watermarking, LSB Method.

\end{latin}
