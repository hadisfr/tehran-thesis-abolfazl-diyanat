\clearpage\newpage
\section*{چکیده}
\baselineskip=.90cm

نهان‌سازی اطلاعات به دانش درج پیام یا نشانه در یک سیگنال یا فایل اطلاق می شود و امروزه به عنوان یکی از شاخه­ های امنیت اطلاعات مورد توجه بسیار قرار گرفته است. این پایان‌نامه، بر روی استفاده از نسخه آنتروپیک سیگنال در دو شاخه از علم نهان‌سازی، یعنی نشان‌گذاری و نهان‌کاوی، متمرکز گردیده است.  نحوه انتخاب قالب‌ها و تخمین نویز سیگنال در نشان‌گذاری، و تحلیل مقادیر تکین در نهان‌کاوی، نمونه‌هایی از کاربردهای نسخه آنتروپیک سیگنال در نهان‌سازی اطلاعات محسوب می‌شوند که در طرح‌های پیشنهادی در این پایان‌نامه مورد توجه و تحقیق قرار گرفته اند. 

در نشان‌گذاری، دو طرح جدید  برای سیگنال‌های ویدئویی {\lr{AVI}} ارایه می‌شوند. در هر دو طرح ابتدا سیگنال ویدئو را به چندین بخش و هر بخش را به چندین قالب سه بعدی تقسیم‌ می‌کنیم. قالب‌های سه‌بعدی با بیشینه آنتروپی را انتخاب کرده، و سپس نشانه را در ضرایب فرکانس پایین تبدیل موجک این قالب‌های سه‌بعدی درج می‌نماییم. در طرح اول که یک طرح نشان‌گذاری نیمه‌کور محسوب می شود، فرستنده می‌بایست اطلاعات آماری سیگنال پوشش را از طریق یک کانال امن برای گیرنده ارسال نماید. گیرنده با استفاده از این اطلاعات و استفاده از آشکارساز بیشینه شباهت سعی در آشکارسازی نشانه می‌نماید. اما در طرح دوم، نیازی به ارسال اطلاعات آماری سیگنال پوشش نیست. به جای آن، فرستنده برخی از ضرایب فرکانس پایین تبدیل موجک سیگنال را بدون تغییر رها می‌کند تا گیرنده بتواند بوسیله آن‌ها ابتدا خواص آماری سیگنال پوش را استخراج نماید و سپس به آشکارسازی نشانه بپردازد. 

در بخش نهان‌کاوی تحلیل نهان‌نگاری به روش {\lr{LSB}} در حوزه مکان مورد توجه قرار دارد. محور ثقل ما در نهان‌کاوی، تجزیه مقادیر تکین است که به نحوی بیانگر میزان آنتروپی سیگنال هستند. اگر پیامی به سیگنال اضافه شده باشد، میزان این آنتروپی افزایش می‌یابد. مقداری از آنتروپی مربوط به محتوی اصلی سیگنال و مقداری نیز ناشی از پیام درج شده است که عموماً دارای ماهیت شبه تصادفی است. ما سعی می‌کنیم که تخمینی از مقادیر تکین سیگنال پاک را بدست آوریم. با تخمین این مقدار می‌توان به وجود و یا عدم وجود پیام در سیگنال ارسالی دست یافت. همچنین  با استفاده از این روش می‌توان نرخ نهان‌نگاری را نیز تقریب زد. ماشین یادگیری {\lr{SVM}} و تخمین نویز سطوح مختلف سیگنال تصویر، یاریگر ما در جهت رسیدن به یک تخمین بهتر و دقیق تر از سیگنال پوشش بوده است. برای ارزیابی هرچه بهتر روش پیشنهادی، در شبیه‌سازی‌ها از پایگاه های جامع تصاویر استفاده شده است. 

کلمات کلیدی: نهان‌نگاری، نهان کاوی، نشان‌گذاری کور، نشان گذاری نیمه کور، روش {\lr{LSB}}


