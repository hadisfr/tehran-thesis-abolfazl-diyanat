\chapter{مروری بر کارهای پیشین}

\begin{intro}
این فصل درصدد بیان کارهای گذشته در زمینه نهان‌سازی اطلاعات می باشد. سعی می‌کنیم تا یک خط سیر مشخص و مرتبط با موضوع پایان‌نامه را در بیان کارهای گذشته پیش بگیریم. همانند فصل گذشته، مطالب در دو بخش نشان‌گذاری و نهان‌کاوی به صورت جدا بیان می‌شود. 
\end{intro}


\section{نشان گذاری}
\subsection{نشان‌گذاری در تصویر و ویدئو}
\index{نشان گذاری}\index{همه پخشی}
نشان‌گذاری سیگنال‌های ویدئویی، امروزه به شدت در حال گسترش و توسعه می‌باشد. نشان‌گذاری در سیگنال‌های ویدئویی کاربردهای فراوانی از قبیل جلوگیری از کپی‌برداری‌های غیرمجاز، احراز اصالت ویدئو، کنترل سیستم‌های  {\inpdic{همه پخشی}{Broadcast monitoring}}، بهبود کدبندی ویدئو و ...، دارد  {\cite{Doerr2003}}.
شفافیت و مقاومت دو موضوعی است که در نشان‌گذاری غالبا در جهت عکس یکدیگر عمل می‌کنند. تاکنون روش های نشان گذاری زیادی در زمینه ی تصویر ارایه شده است {\cite{Jagadeesh2010,Bandyopadhyay2009,Buse2009,Wang2009a}}، ولی به دلایلی این روش‌ها به طور خام، مناسب برای اعمال بر روی ویدئو نیستند {\cite{Chen2010c}}:
\begin{dingautolist}{182}
\item
یک فایل ویدئویی از  تعدادی فریم تشکیل شده است، و فریم‌های همسایه با یکدیگر همبستگی شدیدی دارند. 
\item
حملاتی از قبیل قطع فریم، میانگین‌گیری از فریم‌ها، انواع فشرده‌سازی‌های فایل‌های ویدئویی و ... وجود دارند، که تنها مخصوص فایل‌های ویدئویی است.
\item\index{زمان واقعی}
در کاربردهای زمان واقعی، الگوریتم نشان‌گذاری می‌بایست زمان واقعی باشد، تا بتواند برای این کاربردها مناسب باشد. 
\end{dingautolist}
اگر بخواهیم به طور خلاصه ویژگی های روش  پیشنهادی را ذکر کنیم، می توان موارد زیر را برشمرد:
\begin{itemize}
\incl
نشان‌گذاری در حوزه غیرفشرده
\incl
نشان‌گذاری غیرجمع‌شونده
\incl
استفاده از تبدیل موجک
\incl
استفاده از آشکارساز بیشینه شباهت
\end{itemize}

در ادامه سعی می‌کنیم، با برخی از کارهای انجام‌شده در حوزه نشان‌گذاری با ویژگی‌های یادشده، آشنا شویم. 


\subsection{نشان‌گذاری در حوزه فشرده و غیرفشرده }
\index{حوزه!فشرده}\index{حوزه!غیرفشرده}
در حالت کلی دو دسته روش برای نشان‌گذاری در ویدئو وجود دارد. نشان‌گذاری در  {\inpdic{حوزه فشرده}{Compressed Domain}} \cite{Belhaj2010,Xu2010a,Zhang2010a,Langelaar1998,Bavipati2010} و در {\inpdic{غیرفشرده}{Uncompressed Domain}} {\cite{Lancini2002,Chan2003,Deguillaume1999}}. 

\index{QIM}\index{پوشش ادراکی}\index{بردارحرکت}
در نشان‌گذاری در حوزه فشرده، بدون این که ویدئوی فشرده شده را غیرفشرده کنیم، عملیات نشان‌گذاری را انجام می‌دهیم. 
{\lr{Belhaj}} در \cite{Belhaj2010} با روشی بر مبنای \lr{QIM\LTRfootnote{Quantization Index Modulation}} و {\inpdic{پوشش ادراکی}{Perceptual Masking}} نشان‌گذاری را در استاندارد \lr{MPEG-4} انجام می‌دهد. {\lr{Langelaar}} و همکاران در {\cite{Langelaar1998}} نیز دو روش را پیشنهاد داده‌اند که توسط آن بیت‌های پیام به صورت مستقیم در ویدئوی فشرده‌شده با استاندارد {\lr{MPEG}}، درج می‌شود. در روش اول از تغییر طول کد برای نشان‌گذاری استفاده می‌شود، و در روش دوم با حذف برخی از فرکانس‌های بالای تبدیل {\lr{DCT\LTRfootnote{discrete cosine transform}}} عملیات درج انجام می‌پذیرد. ایده دیگر استفاده از {\inpdic{بردارحرکت}{Motion Vector}} (\lr{MV}) برای نشان‌گذاری است \cite{Zhang2001,Kung2003}. 
روش \lr{Meng} در \cite{Meng1996} و روش استفاده از \lr{SS\LTRfootnote{spread spectrum}} در ضرایب غیر صفر \lr{DCT} سیگنال {\cite{Hartung1998}}،  را می‌توان به عنوان نمونه‌ای دیگر از این روش‌ها نام برد. 

روش‌های نشان‌گذاری در حوزه فشرده برای کاربردهای زمان واقعی بسیار مناسب هستند، اما این روش‌ها معمولا محدود به استاندارد خاصی می باشند. 

روش‌های حوزه غیرفشرده محدود به استاندارد خاصی نمی‌شوند. اما این دسته از روش‌ها را باید بگونه‌ای طراحی نمود که در برابر حملات فشرده‌سازی مقاوم باشند، چرا که معمولا ویدئوها به صورت فشرده وارد کانال انتقال می‌شوند. در حوزه غیر‌فشرده عملیات نشان‌گذاری به دو صورت می‌تواند انجام شود:
\begin{dinglist}{202}
\mord\index{حوزه!مکان}
نشان‌گذاری در {\inpdic{حوزه مکان}{Spatial Domain}}: مانند {\cite{Lancini2002,Mobasseri1998}}. 
\mord\index{حوزه!تبدیل}
نشان‌گذاری در {\inpdic{حوزه تبدیل}{Transform Domain}}: مانند استفاده از تبدیل موجک ({\lr{DWT\LTRfootnote{Discrete Wavelet Transform}}}) {\cite{Chan2003,Al-Taweel2010,Campisi2005,Reyes2010}}، استفاده از تبدیل فوریه گسسته 
({\lr{DFT}}) {\cite{Deguillaume1999}} و یا استفاده از {\lr{DCT}} {\cite{Swanson2002,Sun2010a}}.
\end{dinglist}
روش‌های حوزه مکان معمولا ساده هستند، اما غیرمقاوم. ولی بالعکس روش‌های حوزه تبدیل، می‌توانند مقاومت قابل توجهی را برای ما به ارمغان بیاورند {\cite{Abdallah2009}}. روش پیشنهادی ما در این پایان‌نامه نیز از جمله  روش‌های حوزه تبدیل می‌باشد. ما از تبدیل موجک سه‌بعدی برای امر نشان‌گذاری استفاده کردیم. در میان روش‌های نشان‌گذاری در تصویر و ویدئو طرح‌های بسیاری مبتنی بر تبدیل موجک می باشد. 
{\lr{Chan}} و {\lr{Lyu}} قسمت های مختلف نشانه را در تبدیل موجک صحنه های ویدئو درج می کنند. آن ها این کار را با جابه جایی مکان های برخی از ضرایب تبدیل موجک انجام می دهند \cite{Chan2003}. 

{\lr{Guo-juan}} در {\cite{Guo-juan2009}} یک روش نشان گذاری کور بر مبنای گشتاور {\lr{Zernike}} و تجزیه مقادیر تکین برای فایل‌های ویدئویی ارایه داده است. او عملیات {\lr{SVD}} را بر روی ضرایب فرکانس پایین تبدیل موجک اعمال کرده، و پیام را با تغییر بزرگترین مقدار ویژه هر فریم در سیگنال درج می کند. {\lr{Wang}} از هر فریم تبدیل موجک گرفته و سپس بر اساس روش {\lr{QIM}} عملیات درج در برخی از ضرایب فرکانس بالای تبدیل موجک انجام می‌شود. 

\subsection{نشان‌گذاری جمع‌شونده و غیرجمع‌شونده}
\index{جمع شونده}\index{غیرجمع شونده}
سامانه‌های نشان‌گذاری را از دیدگاه شیوه درج نشانه، به دو دسته {\inpdic{جمع شونده}{Additive}} و {\inpdic{غیرجمع شونده}{Non-addative}} می‌توان تقسیم‌بندی نمود. در روش‌های نهان‌نگاری جمع‌شونده، نشانه به ویژگی در نظر گرفته شده برای درج اضافه می‌شود. این ویژگی می‌تواند ضرایب موجک، مقدار سطح پیکسل و یا ... باشد {\cite{Motwani2010,Pu2010,Zhang2010a}}. 

در نشان‌گذاری غیرجمع‌شونده بر اساس خواص سیگنال پوشش، یک عملیات غیرخطی بر روی ویژگی مورد نظر انجام می‌شود. روش‌های غیر جمع‌شونده معمولا در حوزه تبدیل اعمال می‌شوند \cite{Cox2002}.  روش پیشنهادی ما جزو دسته غیرجمع‌شونده قرار می‌گیرد. در طرح پیشنهادی بیت‌های نشانه در ضرایب تبدیل موجک سیگنال ضرب می‌شود. 

\subsection{آشکارسازها}
\index{آشکارسازی}
آشکارسازهای مبتنی بر همبستگی برای نشان‌گذاری ضربی چندان کارایی ندارد \cite{Cox2002}. بنابراین چندین آشکارساز جایگزین در این زمینه پیشنهاد شده است {\cite{Barni2001,Wang2008,Solachidis2004,Ng2006}}. در {\cite{Barni2001}} آشکارسازی بر مبنای معیار {\lr{Neyman-Pearson}} ارایه شده است، که ادعا می‌شود عملکرد بهتری نسبت به روش‌های مبتنی بر همبستگی دارد. {\lr{Wang}} در {\cite{Wang2008}}  از ضرایب تبدیل موجک به منظور آشکارسازی استفاده نموده است. او یک آشکارساز بهینه محلی برای نشان‌گذاری ضربی {\lr{Barni}} ارایه داده است. در بسیاری از مقالات، ضرایب تبدیل {\lr{i.i.d}} در نظر گرفته می‌شود، که این فرض چندان صحیح به نظر نمی‌رسد. 

در این پایان نامه دو روش نشان گذاری ارایه شده  است، که بر مبنای نشان گذاری ضربی است که در مراجع {\cite{Akhaee2009a,Akhaee2010,Akhaee2009c}} برای تصاویر ارایه شده است. در این روش از آشکارساز بیشینه شباهت به منظور آشکارسازی نشانه استفاده می کنیم. 





\section{نهان کاوی}
\index{نهان کاوی}\index{ویژگی}\index{سیگنال پوش}
روش‌های پیشرفته در نهان نگاری به دنبال یافتن {\inpdic{ویژگی}{Feature}}هایی از سیگنال پوش هستند، تا بوسیله ی آن بتوانند به حضور پیام مخفی شده در سیگنال پی ببرند.  در سال های اخیر روش های زیادی برای تحلیل \lr{LSB} پیشنهاد شده است 
\cite{Harmsen2003_Steganalysis,Fridrich2001_Detecting,Dumitrescu2002_steganalysis,Agaian2010,
Chiew2010,Goyal2010,He2010,Kashyap2010}.
در ادامه مروری مختصر بر روی برخی از مهم ترین این روش ها داریم. اهمیت بررسی روش های یاد شده، این است که علاوه بر پایه ای بودن، برخی از این روش های نهان کاوی الهام بخش ما در ارایه طرح های پیشنهادی، بوده است. 

\subsection{نهان کاوی توسط ماتریس نقاب}
به عنوان اولین روش نهان کاوی، سراغ روش خانم فردریچ{\LTRfootnote{Fridrich}} می رویم \cite{Fridrich2001_Detecting}. این روش بر روی نهان نگاری {\lr{LSB}} تمرکز دارد.
\subsubsection{پیش زمینه}
فرض کنید، تصویر پوششی با ابعاد {$ M\times N $} در اختیار داریم. مقادیر هر پیکسل از مجموعه ی مفروض $ P $ انتخاب می شود. برای مثال هنگامی که با تصاویرخاکستری  سروکار داریم، با مجموعه ی  $ P={0,...,255} $    سروکار داریم. طرح  {\inpdic{درج}{Embeding}} بدین صورت است، که ابتدا تصویر را به گروه پیکسل های ناهم پوشان تقسیم بندی می کنیم. هر گروه متشکل از $ n $ پیکسل {\inpdic{مجاور}{Adjacent}} خود است، که با $ (x_{1},x_{2},\ldots ,x_{n}) $ نمایش می دهیم. برای مثال اگر $ n=4 $ باشد، می توانیم چهار پیکسل کنار هم در یک ردیف را انتخاب کنیم. 

تابعی چون $ f $ را که تابع معینی است، در نظر بگیرید. این تابع به گروه پیکسلی مفروض، مقداری حقیقی نسبت می دهد.
\begin{equation}
f(x_{1},\ldots ,x_{n})\in \mathbb{Z}
\end{equation}
هدف غایی تابع $ f $ این است، که {\inpdic{همواری}{Smoothness}} و {\inpdic{یکنواختی}{Regularity}} گروه پیکسلی مورد نظر را حفظ کند. از این به بعد ما گروه پیکسل مفروض را با $ G $ نمایش می دهیم. بدیهی است که هر چقدر مقادیر عضو $ G $ حالت نویزی تری داشته باشد، می بایست مقدار تابع $ f $ بزرگتر باشد. تابع زیر پیشنهادی است که شرط فوق را برآورده می کند. 
\begin{equation}
\label{fFun}
f(x_{1},x_{2},\ldots ,x_{n})=\Sigma _{i=1}^{n-1}|x_{i+1}-x_{i}|
\end{equation}
\index{تابع پرش}
در ادامه ی راه یک تابع معکوس پذیر به نام $ F $ تعریف می کنیم، و آن را تابع {\inpdic{پرش}{Flipping} } نام می نهیم. این تابع  {\inpdic{جایگشتی}{Permutation}} از سطوح خاکستری است، که با دو بار اعمال تابع $ F $ عناصر ورودی به خودشان نقش بسته می شوند. 
\begin{equation}
\label{FFun}
F^{2}=Identity\qquad F(F(x))=x\qquad \forall x\in P
\end{equation}
برای مثال تابع $ F_{1}=0\leftrightarrow 1,2\leftrightarrow 3,\ldots ,254\leftrightarrow 255 $ را در نظر بگیرید. با نگاهی به این تابع می توان به سادگی دریافت که با اعمال تابع فوق {\inpdic{کم ارزش ترین بیت}{Least Significant Bit}} هر سطح خاکستری تصویر ما، اگر صفر است که یک خواهد شد، و اگر یک است تبدیل به صفر خواهد شد. علاوه بر این تابع $ F_{-1}=-1\leftrightarrow 0,1\leftrightarrow 2,\ldots ,255\leftrightarrow 256 $ را که در واقع جابه جا کننده ی کم ارزش ترین بیت است، را نیز مدنظر قرار می دهیم. 
\begin{equation}
\label{F-1}
F_{-1}(x)=F_{1}(x+1)-1 \qquad \forall x
\end{equation}
برای تکمیل بحث تابع $ F_{0} $ که هر سطح را به خودش نقش می کند، را نیز در نظر می گیریم. 
\begin{equation}
F_{0}(x)=x \qquad \forall x
\end{equation}
\index{گروه!تکین}\index{گروه!یکنواخت}\index{گروه!غیرقابل استفاده}
با استفاده از تابع $ f $ و $ F $ پیکسل های تصویر را در سه گروه قرار می دهیم. این گروه ها را به ترتیب
\inpdic{گروه های یکنواخت}{Regular groups}، \inpdic{گروه های تکین}{Singular groups}
 و {\inpdic{گروه های غیر قابل استفاده}{Unusable groups}} نام می نهیم و با نمادهای $ R $، $ S $ و $ U $ نمایش می دهیم. روابط زیر نحوه ی جایگیری گروه های $ G $ را در سه گروه یاد شده بیان می کند.
\begin{eqnarray}
G\in R \Longleftrightarrow f(F(G))>f(G)\nonumber \\
G\in S \Longleftrightarrow f(F(G))<f(G)\nonumber \\
G\in U \Longleftrightarrow f(F(G))=f(G)
\end{eqnarray}
در روابط بالا، $ F(G) $ بدان معنا است که تابع $ F $ را به گروه پیکسل $ G=(x_{1},\ldots ,x_{n} ) $ اعمال بکنیم. البته ممکن است بخواهیم که توابع پرش متفاوت به هر یک از پیکسل های گروه $ G $ اعمال بکنیم. بدین منظور ماتریس $ M $ را در نظر بگیرید. ماتریس $ M $ ماتریسی به طول $ n $ است که عناصر آن مقادیر $ 0 $، $ 1 $ و $ -1 $ است. به عنوان مثال عنصر $ i $ ام ماتریس $ M $ نشان دهنده ی تابع پرشی است که می بایست به عنصر $ i $ ام گروه $ G $ اعمال شود. در نهایت داریم:
\begin{equation}
F(G)=\lbrace F_{M(1)}(x_{1}),F_{M(2)}(x_{2}),\ldots ,F_{M(n)}(x_{n}) \rbrace
\end{equation}
هدف غایی تابع $ F $  این است که سطوح خاکستری هر پیکسل را به صورتی جایگشت دهد، که بتوان دوباره به حالت اولیه بازگشت. یعنی تابع ما معکوس پذیر باشد. در حقیقت با این کار می خواهیم اضافه شدن نویز به سیستم را مدل کنیم. ولی این نویز اضافه شده به صورتی است که می توان به سادگی آن را به صورت کامل حذف کرد. اضافه کردن نویز به تصاویر باعث می شود که مقدار تابع $ f $ افزایش یابد. این نکته نیز قبلا بیان شده بود. لذا انتظار بر این است که با اعمال تابع $ F $ بر روی گروه $ G $، تعداد گروه های یکنواختی بیشتر از گروه های تکین باشد. این {\inpdic{اریبی}{Bias}}، تعیین کننده ی میزان اطلاعاتی است که می توان در تصویر پنهان کرد، بدون این که خواص بینایی تصویر آسیب ببیند. 

\subsubsection{روش تحلیل}
\index{ماتریس نقاب}
فرض کنید که برای گروه $ G $ با {\inpdic{ماتریس نقاب}{Mask Matrix}}، {$ M $} تعداد نسبی گروه های سه گانه ی تعریف شده در بخش قبلی را به ترتیب با $ R_{M} $، $ S_{M} $ و $ U_{M} $ نمایش بدهیم. برای $ M $ های منفی می توان روابط زیر را نوشت:
\begin{equation}
\label{expect}
R_{M}+S_{M}\leq 1\qquad R_{-M}+S_{-M}\leq 1
\end{equation}
فردریچ{\LTRfootnote{Fridrich}}  در روشی که در \cite{Fridrich2001_Detecting} ارایه کرده است، فرض کرده است که برای تصاویر مورد تحلیل، {\inpdic{امید ریاضی}{Expected Value}} { $ R_{M} $} با {$ R_{-M} $} برابر است. در \cite{Fridrich2001_Detecting} فردریچ ادعا کرده است، که درستی \ref{expect} را می توان به صورت شهودی و با شبیه سازی نشان داد. در ضمن ادعا شده است که بعد از تصادفی کردن {\inpdic{سطح کم ارزش ترین بیت}{LSB Plane}}، رابطه ی \ref{expect} دیگر برقرار نیست. 
 
 فرض می کنیم که با یک پیام تصادفی سروکار داریم. طول پیام را $ m $ گرفته، اگر از روش درج در کم ارزش ترین بیت استفاده کنیم، هر چقدر نرخ پنهان نگاری ما بیشتر می شود، می توان نتیجه گرفت که سطح {\lr{LSB}} تصادفی تر می شود. لذا انتظار می رود که اختلاف بین $ R_{M} $ و $ S_{M} $ با افزایش طول پیام به صفر میل می کند. اما نکته ی قابل توجه این است که تاثیر پنهان نگاری در {\lr{LSB}} نتیجه ی عکس بر روی اختلاف مقادیر $ R_{-M} $ و $ S_{-M} $ دارد؛ یعنی با افزایش $ m $ این اختلاف بیشتر می شود. شکل \ref{RMSM} در صدد بیان این موضوع است.
 
 \begin{figure}[!htbp]
\centerline{\includegraphics[width=10cm]{Chapters/images/RMSM}}
\caption{
نمودار { \lr{RS}} برای تصویر با ماتریس نقاب $ M=[0 1 1 0] $}
\label{RMSM}
\end{figure}

در ادامه مقاله فردریچ سعی در تحلیل و توضیح این اتفاق تا حدودی دور از انتظار کرده است. ماتریس نقابی برابر با $ M=[0 1 0] $ در نظر بگیرید. مجموعه های زیر را تعریف می کنیم.
\begin{eqnarray}
C_{i}  &=&{2i,2i+1}\qquad i=0,\ldots ,127\\
C_{rst}&=& \lbrace G|G\in C_{r}\times C_{s}\times C_{t}\rbrace
\end{eqnarray}
 تعداد $ 128^{3} $ مجموعه $ C_{rst} $ می توان داشت. هر مجموعه در برگیرنده ی هشت گروه است. مجموعه ی تعریف شده نسبت به عمل پنهان نگاری {\lr{LSB}}، {\inpdic{بسته}{Closed}}  است. چهار مجموعه مختلف از $ C_{rst} $ در نظر می گیریم. 
\begin{table}[!htbp]
\caption{چهار مجموعه مختلف از $ C_{rst} $}
\centering
\begin{tabular}{|c|c|c|}
\hline
\tablefont {نوع مجموعه}&\tablefont{$ F_{1} $}&\tablefont{$ F_{-1} $}\\\hline
\tablefont {$ r=s>t $}&\tablefont {$ 2R,2S,4U $}&\tablefont {$ 4R,4U $}\\\hline
\tablefont{$ r<s>t $}&\tablefont{$ 4R,4S $}&\tablefont{$ 4R,4S $}\\\hline
\tablefont{$ r>s>t $}&\tablefont{$ 8U $}&\tablefont{$ 8U $}\\\hline
\end{tabular}
\label{clique}
\end{table}
\index{برون یابی}
 مطالب بیان شده در قسمت قبل در جدول \ref{clique} به خوبی نمایان است. طرحی که فردریچ در این مقاله ارایه داده است، و آن را تحلیل {\lr{RS}} نام نهاده است، بر این پایه استوار است که ابتدا چهار منحنی {\lr{RS}} را بدست می آورد. سپس با استفاده از {\inpdic{برون یابی}{Extrapolation}} نقاط تقاطع منحنی ها را بدست می آورد. نتایج حاصل از شبیه سازی نشان داده است که منحنی های $ R_{-M} $ و $ S_{-M} $ را می توان با خط مستقیم و $ R_{M} $ و $ S_{M} $ را با یک منحنی درجه ی دو مدل کرد. 

در ادامه فرض کنید که در تصویر پوش، پیامی به طول  $ p $ (نسبت به تعداد پیکسل ها) پنهان کرده ایم. تحلیل گر از مقدار $ p $ آگاهی ندارد. بار دیگر شکل \ref{RMSM} را در نظر بگیرید. در این شکل محور افقی درصد پیام مخفی را نشان می دهد، که در تصویر پوش پنهان شده است. محور عمودی تعداد گروه های $ R $ و $ S $ نشان می دهد. 
 
ضریب $ .5 $ در معادلات فوق به این دلیل است که چون پیام ما یک رشته ی تصادفی است، به طور میانگین عملیات درج پیام در نیمی از پیکسل ها پرش ایجاد می کند. اگر ما در {\lr{LSB}} تمامی پیکسل ها پرش انجام دهیم، و دوباره مقادیر $ R $ و $ S $ را محاسبه کنیم، چهار نقطه ی جدید بدست می آوریم. این نقاط به صورت زیر است:
\begin{equation}
R_{M}(1-p/2),S_{M}(1-p/2),R_{-M}(1-p/2),S_{-M}(1-p/2)
\end{equation}
اگر سیگنال پنهان نگاری شده را در نظر بگیرید، به دلیل این که در سطح {\lr{LSB}} با رشته ی تصادفی مواجه هستیم، لذا دو نقطه در
 {$ R_{M}(1/2) $} و {$ S_{M}(1/2) $} را بدست خواهیم آورد. ما باید این رویه را بارها انجام دهیم، تا بتوانیم با استفاده از نمونه های آماری تخمین خوبی از این دو نقطه بدست آوریم.
\subsection{
تحلیل تطابق {\lr{LSB}} با روش واسنجی}

\index{واسنجی}\index{تابع مشخصه هیستوگرام}\index{مرکز جرم}
در این بخش به بررسی مقاله ی آقای لی{\LTRfootnote{Xiaolong Li}} درباره تحلیل طرح تطابق {\lr{LSB}} می پردازیم \cite{Li2008_Detecting}. فرض کنید که $ I $ نشان دهنده ی یک تصویر خاکستری باشد. هیستوگرام این تصویر را با $ h $ و {\lr{DFT\LTRfootnote{Discrete Fourier Transform}}} آن را با   $ \hat{h} $ نمایش می دهیم. هارمسن{\LTRfootnote{Harmsen}} در \cite{Harmsen2003_Steganalysis} به پارامتر $ \hat{h} $، {\inpdic{تابع مشخصه هیستوگرام}{Histogram Characteristic Function}} {\lr{HCF}} می گوید. {\inpdic{مرکز جرم}{Center of Mass}} {\lr{HCF}} به صورت زیر تعریف می شود.
\begin{equation}
C(I)=\frac{\Sigma _{k=0}^{N/2}k|\hat{h}(k)|}{\Sigma _{k=0}^{N/2}|\hat{h}(k) | }\qquad (N=256)
\end{equation}
\index{LSB Matching}
در ادامه راه فرض کنید که $ I_{c} $ تصویر پوش ما باشد، و $ I_{s} $ تصویر پنهان نگاری شده ی ما باشد. طرح پنهان نگاری مورد استفاده، همان طور که حدس زدید طرح  {\inpdic{تطابق کم ارزش ترین بیت}{LSB Matching}} با نرخ پنهان نگاری $ \alpha $  است. $ h_{c} $ و $ h_{s} $ به ترتیب نمایشگر هیستوگرام تصویر پوش و تصویر پنهان نگاری شده است. رابطه زیر نیز که بدیهی به نظر می آید.
\begin{equation}
h_{s}=f_{\alpha }*h_{c}
\end{equation}
در حقیقت این گونه می توان به قضیه نگاه کرد، که پنهان نگاری در حقیقت یعنی اضافه کردن نویز به تصویر پوش مورد نظر. لذا $ f_{\alpha } $ توزیع نویز پنهان نگاری است. به گونه ای که:
\begin{equation}
f_{\alpha }(0)=1-\frac{\alpha }{2}\qquad f_{\alpha }(1)=f_{\alpha }(-1)=\frac{\alpha }{4}
\end{equation}
در مرحله ی بعدی از رابطه ی بیان شده، تبدیل {\lr{DFT}} می گیریم.
\begin{equation}
\label{DFTHist}
\hat{h}_{s}(k)=\hat{f}_{\alpha }(k)\times \hat{h}_{c}(k)
\end{equation}
بعد  از انجام عملیات درج تطابق {\lr{LSB}}، مرکز جرم {\lr{HCF}} کاهش پیدا می کند. \cite{Harmsen2003_Steganalysis} با استفاده از همین ایده سیگنال پنهان نگاری شده را از سیگنال پوش تمایز می دهد. پس برای تصویر مورد بررسی مقدار $ C(I) $ را محاسبه می کنیم. اگر این مقدار از آستانه ای که آن را با $ T $ مشخص می کنیم، بیشتر بود، حدس می زنیم که پیامی در سیگنال پنهان شده است. \cite{Ker2005_Steganalysis} این آشکارساز را بهبود بخشیده است. $ \tilde{I} $ را به عنوان سیگنال نمونه برداری شده ی $ I $ در نظر بگیرید. 
\begin{equation}
\tilde{I}_{i,j}=\lfloor (I_{2i,2j}+I_{2i+1,2j}+I_{2i,2j+1}+I_{2i+1,2j+1})/4 \rfloor
\end{equation}

شبیه سازی های انجام گرفته نشان داده است، که برای سیگنال پوش $ C(I)\simeq C(\tilde{I} ) $، ولی برای سیگنال پنهان نگاری شده { $ C(I) < C(\tilde{I} ) $}. شبیه سازی ها نشان داده است که روش ارایه شده بهتر از روش قبلی است. در \cite{Ker2005a_Resampling} علت بهتر بودن این روش نسبت به روش قبلی بیان شده است. 
 
 در ادامه روش ارایه شده در این مقاله را مورد بررسی قرار می دهیم. پارامتر $ I^{d} $ را به عنوان تفاضل تصویر تعریف می کنیم. این تفاضل به صورت تفاضل پیکسل های همسایه به صورت زیر تعریف می شود. 
 \begin{equation}
\label{method}
I^{d}_{i,j}=I_{2i,j}-I_{2i+1,j}+255
\end{equation}
ایده اصلی این جا است که $I^{d} $ به نوعی نشان دهنده ی نویز پنهان نگاری است. نکته ی اساسی و پایه ای که در ادامه ی راه شالوده و بنیان تحلیل ارایه شده است، این است که رویه ی نمونه برداری نویز پنهان نگاری را کاهش می دهد. مقدار $ I^{d}_{i,j} $ بعد از عملیات درج پیام به صورت $ I^{d}_{i,j}+s+t $ تغییر پیدا می کند. $ s $ و $ t $ دو متغیر تصادفی مستقل با توزیع $ f_{\alpha } $ هستند. با توجه به مطلب یاد شده می توان رابطه ی بین هیستوگرام ها را به صورت زیر نوشت.
\begin{equation}
h^{d}_{s}=f_{\alpha }*f_{\alpha }*h_{c}^{d}
\end{equation}
در مرحله ی بعدی از تصویر مورد بررسی به صورت زیر نمونه برداری می کنیم.
\begin{equation}
\tilde{I}^{d}_{i,j}=\lfloor (I^{d}_{i,j}+I^{d}_{i,2j+1})\rfloor
\end{equation}
برای ادامه ی راه قضیه زیر می توان کمک کار ما باشد.
\begin{theorem}
فرض کنید که $ X,Y_{1},Y_{2},\ldots ,Y_{2M} $ رشته متغیر تصادفی گسسته باشد. $ X $ دارای توزیع یکنواخت در مجموعه ی $ \{ 0,1,\ldots , M-1\} $ است. $ Y_{i} $ ها بگونه ای هستند که شروط زیر را برآورده می کند.
\begin{eqnarray}
P(Y_{i}=0)&=&1-\frac{\alpha }{2}\nonumber \\
P(Y_{i}=1)&=&P(y_{i}=-1)=\frac{\alpha }{4}\qquad \alpha \in [0,1]
\end{eqnarray}
اگر داشته باشیم.
\begin{equation}
\mu =P\left( \lfloor \frac{x+\Sigma _{i=1}^{2M}Y_{i}}{M}\rfloor =t \right) 
\end{equation}
آن گاه داریم:
\begin{equation}
\mu _{2}=\mu _{-2}\leq (\frac{g_{M}\alpha }{4})^{2}\qquad \mu _{1}=\mu _{-1}\leq \frac{g_{M}(\alpha )}{2}-\frac{g_{M}(\alpha )}{4}
\end{equation}
\end{theorem}
فرض می کنیم که $ (I^{d}_{c})_{i,2j}+(I^{d}_{c})_{i,2j+1} $ دارای توزیع یکنواخت به پیمانه ی دو باشد. با استفاده از قضیه ای که در بالا مطرح شد، داریم.
\begin{eqnarray}
\tilde{h}^{d}_{s}&=&\tilde{f}_{\alpha }*\tilde{h}_{c}^{d}\nonumber \\
\mu _{2}&=&\mu _{-2}<(\frac{\alpha }{4})^{2}\qquad \mu _{1}=\mu _{-1}<\frac{\alpha }{2}-\frac{\alpha ^2}{4} \nonumber \\
\tilde{f}_{\alpha }(t)&<& (f_{\alpha }*f_{\alpha })(t)\qquad \forall t\in {\pm 1,\pm 2 } 
\end{eqnarray}
علاوه بر این در حوزه ی {\lr{DFT}} داریم.
\begin{eqnarray}
\tilde{h}^{d}_{s}(k)&=&(1-(2\alpha - \alpha ^{2})\sin ^{2}\theta -\frac{\alpha ^{2} }{4}\sin ^{2}2\theta )\tilde{h}^{d}_{c}(k)\\
\tilde{\tilde{h}}^{d}_{s}(k)&=&(1-4\mu _{1}\sin ^{2}\theta -4\mu _{2}\sin ^{2}2\theta )\tilde{\tilde{h}}^{d}_{c}(k)
\end{eqnarray}
در روابط بالا $ \theta = k\pi\setminus N $. نتیجه ی کار در این قسمت نمایان می شود. 
\begin{equation}
d^{d}(k,I_{s})\leq d^{d}(k,I_{c})\qquad d^{d}(k,I)=\frac{|\tilde{h}^{d}(k) |}{|\tilde{\tilde{h}}^{d}(k) |}
\end{equation}


\subsection{نهان کاوی توسط ویژگی های مراتب بالا}
\index{مجموعه های دریچه ای}
در ادامه به بررسی روش، آقای خسروی در نهان کاوی می پردازیم \cite{Khosravirad2009_Higher}.  در این مقاله بر روی تحلیل \lr{LSB} تمرکز شده است. آقای خسروی در این مقاله به ارایه روش جدیدی  در تحلیل پنهان نگاری \lr{LSB} پرداخته اند. در این روش به طور خلاصه از آمارگان مراتب بالا برای امر تحلیل استفاده می کنند. در این روش فضایی از عناصر تعریف می شود، که به خواص آمارگان مراتب بالای سیگنال پوش وابسته است. در این فضا به دنبال مجموعه هایی می گردیم، که به آن ها {\inpdic{مجموعه های دریچه ای}{Closure of Sets}} (\lr{CoS}) می گوییم. 
\begin{definition}
برای یک تابع تصادفی بر روی فضای $ E $ مجموعه های \lr{CoS} را مجموعه ای از زیرمجموعه های غیرهمپوشان در $ E $ تعریف می کنیم، 
{$ C=S_{1}\cup S_{2}\cup \ldots \cup S_{M} $}، به گونه ای که شرایط زیر برآورده شود:
\begin{dingautolist}{182}
\item
برای هر عنصر $ e\in C $، داریم که $ f(e)\in C $.
\item
برای هر عنصر $ e\in E $ که در $ C $ وجود ندارد، داریم $ f(e)\notin C $.

\item
برای هر $ i,j\in {1,\ldots ,M} $  و هر عنصر $ e\in S_{i}$، $ f(e) $ با یک احتمال غیر صفر در $ S_{j} $ باشد. 
\end{dingautolist}
همان طور که می توان حدس زد با در نظر گرفتن $ S=E $ و $ C=S $، در این حالت $ E $ یک \lr{CoS} بدیهی است. 
\end{definition}
هر سیگنال دیجیتالی را می توان به صورت آرایه ای از مقادیر کوانتیزه شده به صورت $ (p_{1},p_{2},\ldots ,p_{n}) $ در نظر گرفت. فضای $ E $ را $ k $گانه هایی به صورت $ (p_{i},\ldots ,p_{i+k-1}) $ در نظر می گیریم. هنگامی که پیام را در \lr{LSB} سیگنال مخفی می کنیم، مانند این است که بر روس هر $ p_{i} $ به طور مستقل با احتمال $ \rho $ یکی از روابط \ref{lsbFun} اتفاق می افتد.
\begin{equation}
\label{lsbFun}
(i) (2l\longrightarrow 2l+1)\qquad or\qquad (2l+1\longrightarrow 2l)
\end{equation}
مقدار $ \rho $ از حاصل تقسیم تعداد تغییرات \lr{LSB} بر $ n $ بدست می آید. فرض کنید که $ C $ یک مجموعه ی \lr{CoS} در $ E $ باشد، و $ S_{1},S_{2},\ldots ,S_{M} $ زیر مجموعه هایی در $ E $ باشد. بعد از اعمال \lr{LSB} مجموعه های $ S_{i} $ تبدیل به مجموعه های $ S_{i}^{*} $ می شوند. قضیه ی زیر تغییر در {\inpdic{اندازه}{Cardinality}} مجموعه های یاد شده را بعد از اعمال \lr{LSB} بیان می دارد.
\begin{theorem}
برای یک تابع تصادفی مثل $ f $ بر روی یک \lr{CoS} مثل $ C $، اگر احتمال تبدیل از $ S_{i} $ به $ S_{j} $ توسط تابع $ f $ برای هر $ i $ و $ j $ ($ i\neq j $) برابر با $ \rho _{1} $ و احتمال تغییر $ S_{i} $ به خودش برابر با $ \rho _{2} $ باشد، آن گاه برای هر $ i $ و $ j $ مخالف هم، رابطه ی زیر برقرار است.
\begin{equation}
\lvert |S_{i}^{*}|-|S_{j}^{*}| \rvert <\lvert |S_{i}|-|S_{j}| \rvert
\end{equation}
\textbf{اثبات}:
برای هر $ i $، داریم: 
$ |S_{i}^{*}|=\rho _{2}.|S_{i}|+\rho _{1}.(|C|-|S_{i}|)  $.
بنابراین می توان نوشت: 
$ \lvert |S_{i}^{*}|-|S_{j}^{*}| \rvert = |\rho _{2}.(|S_{i}|-|S_{j}|) - \rho _{1}.(|S_{i}|-|S_{j}|)| =|\rho _{2}-\rho _{1}|.||S_{i}|-|S_{j}||<||S_{i}|-|S_{j}||$
\end{theorem}
اگر احتمال گذار از $ i $ و $ j $ های مختلف متفاوت باشد، در حالت کلی تعداد گذار از زیرمجموعه های با اندازه های بزرگتر به زیر مجموعه های کوچکتر بیشتر است. لذا می توان انتظار داشت در نهایت اندازه ی زیرمجموعه ها به صورت یکنواخت درآید. کاهش فاصله ی بین ابعاد زیرمحموعه ها در یک $ CoS $ به ما کمک می کند، تا بفهمیم که آیا یک سیگنال پاک است یا نه؟

ایده ی مطرح شده مستقل از نوع سیگنال پوش است، یعنی می توان آن را برای هر سیگنالی اعم از صدا، تصویر و ویدئو بکار برد. اگر برای تصاویر دیجیتالی مقدار $ k=1 $ را در نظر بگیریم، می توان به همان روش \lr{PoV} رسید، که در \cite{Westfeld2000_Attacks} مطرح شده بود. 

\subsubsection{روش ارایه شده برای تصاویر خاکستری}
در این بخش به طور مشروح توضیح داده خواهد شد، که چگونه ایده ی یاد شده را برای تشخیص وجود پیامی که به صورت \lr{LSB} در تصویر خاکستری پنهان شده است، بکار بریم. بدین منظور بر مبنای همین ایده ویژگی از تصاویر را مطرح می کنیم، که نشان دهنده ی آمارگان مراتب دو و سه تصویر هستند. 

\index{همبستگی}
تصویر یک سیگنال دو بعدی است. در نظر گرفتن تصویر به عنوان یک آرایه ی یک بعدی می تواند باعث شود که {\inpdic{همبستگی}{Correlation}}  پیکسل ها به خوبی نمایان نشود. برای بهره بردن از همبستگی موجود بین پیکسل ها تمامی حالات همسایگی بین $ k $ پیکسل را در نظر می گیریم. از بین تمامی حالات های یاد شده برخی دارای آنتروپی پایین (همبستگی بالا) هستند. با استفاده از این ایده تغییری که \lr{LSB} در همبستگی بین پیکسل ها می توان ایجاد کند، بیشتر قابل تشخیص است. زیرا سعی می شود تمامی همبستگی های بین پیکسل ها در نظر گرفته شود. 
 
 به عنوان مثال شکل \ref{adjacencyState} نمایشگر 5 حالت مختلف همسایگی پیکسل ها برای $ k=3 $ است. 
 \begin{figure}[!htbp]
\centerline{\includegraphics[width=11cm]{Chapters/images/Adjancy}}
\caption{پنج حالت همسایگی برای {$ k=3 $}}
\label{adjacencyState}
\end{figure}
از بین حالت های یاد شده مقادیر با آنتروپی کمتر را انتخاب می کنیم. در مرحله ی بعدی می بایست ویژگی های آماری تصویر مناسب برای کار خود را انتخاب و استخراج کنیم. 
\subsubsection{ویژگی های مرتبه دوم}
فرض کنید مجموعه ای از تمامی زوج همسایگی های $ (p,q) $ در اختیار داریم. این مجموعه را می توان به چهار زیرمجموعه مجزا به نام های $ S_{1} $، $ S_{2} $، $ S_{3} $ و $ S_{4} $ تقسیم بندی کرد. این چهار مجموعه با الهام گیری از روش \lr{Dumitrescu} به صورت زیر تشکیل می شود.
\begin{eqnarray}
S_{1}&=&\lbrace (p,q)|(q is even and p<q) or (q is odd and p>q) \rbrace\nonumber \\
S_{2}&=&\lbrace (p,q)|(q is even and p-q>1) or (q is odd and q-p>!) \rbrace\nonumber \\
S_{3}&=&\lbrace (p,q)|(q is even and p-q=1) or (q is odd and q-p=1) \rbrace\nonumber \\
S_{4}&=&\lbrace (p,q)|p=q \rbrace
\end{eqnarray}
همان طور که در شکل \ref{StateDiagram} نشان داده است، $ C_{1}=S_{1}\cup S_{2} $ و $ C_{2}=S_{3}\cup S_{4} $ دو مجموعه ی دریچه ای برای زوج های همسایه ی $ (p,q) $ است.
\begin{figure}[!htbp]
\centerline{\includegraphics[width=10cm]{Chapters/images/fig2}}
\caption{دیاگرام حالت برای {\lr{Cos}}}
\label{StateDiagram}
\end{figure}
انتظار داریم برای تصاویر طبیعی از لحاظ آماری رابطه ی {$ |S_{1}|=|S_{2}| +|S_{3}|$} برقرار است. در ضمن برای $ C_{1} $ داریم: 
\begin{equation}
|S_{1}| +|S_{2}|=|S_{1}^{*}| +|S_{2}^{*}| 
\end{equation}
در کل داریم:
\begin{equation}
|S_{1}^{*}| - |S_{1}|=\frac{|S_{1}^{*}| - |S_{2}^{*}|-|S_{3}|}{2}
\end{equation}
روشن است که مقدار $ ||S_{1}^{*}| - |S_{1}|| $ با تعداد تغییراتی که \lr{LSB} در سیگنال ایجاد می کند، متناسب است. در ادامه می بایست به نوعی این پارامتر را محاسبه کرد. در نظر داشته باشید که ما تصویر پاک را در اختیار نداریم. لذا مقادیر $ |S_{1}| $ و $| S_{3}| $ را در اختیار نداریم. لذا به دنبال تخمین این پارامترها می رویم. $ S_{3} $ را به عنونا تخمینی از $ |S_{3}^{*}| $ در نظر می گیریم. 

\subsubsection{ویژگی آماری مرتبه ی سوم}
\index{ویژگی آماری مرتبه ی سوم}
برای بدست آوردن ویژگی مرتبه ی سوم سه پیکسل همسایه $ (p,q,r) $ را در نظر می گیریم. چهار زیرمجموعه ی مجزا را به صورت زیر تعریف می کنیم.
\begin{eqnarray}
S_{1}&=&\lbrace (p,q,r)|(p=q=2l and r=2l+1) or (p=q=2l+1 and r=2l)\rbrace\nonumber\\
S_{2}&=&\lbrace (p,q,r)|(p=r=2l and q=2l+1) or (p=r=2l+1 and q=2l)\rbrace\nonumber\\
S_{3}&=&\lbrace (p,q,r)|(r=q=2l and p=2l+1) or (r=q=2l+1 and p=2l)\rbrace\nonumber\\
S_{4}&=&\lbrace (p,q,r)|p=q=r\rbrace
\end{eqnarray}
قضیه و روابط بیان شده در قسمت قبلی برای این قسمت نیز صحیح است.


\subsection{تحلیل با استفاده از ماتریس همسایگی پیکسل}
\index{ماتریس همسایگی پیکسل}
در این قسمت به بررسی مقاله ی 2010 آقای \lr{Pevny} و خانم \lr{Fridrich} می پردازیم {\cite{Pevny2010_Steganalysis}}. در حقیقت وابستگی بین پیکسل ها در تصاویر طبیعی را می توان با استفاده از هیستوگرام گروه پیکسل های دوتایی، سه تایی و یا چندتایی همسایه مدل کرد. البته این هیستوگرام ها دارای خواص نامطلوبی هستند، که همین امر موجب می شود، تا نتوان از آن ها در تحلیل استفاده کرد.
\begin{dingautolist}{182}
\item
تعداد مقادیر تابع هیستوگرام با افزایش تعداد پیکسل ها به صورت نمایی رشد می کند. برای یک تصویر 8 بیتی، $ 256^{2}=65536 $ مقدار در تابع هیستوگرام داریم. 
\item
تخمین برخی مقادیر هیستوگرام می تواند نویزی باشد، زیرا که برخی مقادیر در هیستوگرام به ندرت رخ می دهد. 
\item
یافتن یک مدل برای گروه پیکسل ها بسیار مشکل است، زیرا آمارگان آن ها وابسته به محتوای تصویر است. 
\end{dingautolist}
شکل \ref{histFig_1} نشان دهنده ی احتمال $ Pr(I_{i,j},I_{i,j+1}) $ رویداد مقدار $ ( I_{i,j},I_{i,j+1}) $ برای دو پیسکل که به صورت افقی با یکدیگر همسایه هستند، را نشان می دهد. این احتمال با استفاده از 10700 تصویر از پایگاه داده ی \lr{BOWS2}  تخمین زده شده است. با توجه به همبستگی زیاد در حوزه ی مکان در تصاویر طبیعی، انتظار می رود رنگ پیسکل های همسایه شبیه به هم باشند. لذا همان طور که انتظارمی رود نمودار مذکور می بایست به صورت قطری باشد. 
\begin{figure}[!htbp]
\centerline{\includegraphics[width=10cm]{Chapters/images/histFig_1}}
\caption{توزیع سطح خاکستری دو پیکسل همسایه افقی}
\label{histFig_1}
\end{figure}
با مشاهده ی نمودار برای چندین  {$ I_{i,j} $} ملاحظه می شود، که شکل همگی یکسان است، و یا به عبارت دیگر مقدار اختلاف $ I_{i,j+1}-I_{i,j} $ مستقل از مقدار $ I_{i,j}  $ است. در ادامه سعی می کنیم، این موضوع را به صورت کمی بیان کنیم. بدین منظور اطلاعات متقابل $ I(I_{i,j+1}-I_{i,j},I_{i,j}) $ را محاسبه می کنیم. 
\begin{eqnarray}
\mathcal{I}(I_{i,j+1}-I_{i,j},I_{i,j})&=&\mathcal{H}(I_{i,j+1}-I_{i,j})-\mathcal{H}(I_{i,j+1}-I_{i,j}|I_{i,j})\nonumber\\
&=&\mathcal{H}(I_{i,j+1}-I_{i,j})-\mathcal{H}(I_{i,j+1}|I_{i,j})
\label{entropyEq}
\end{eqnarray}
توسط رابطه ی \ref{entropyEq} توانستیم مقدار اطلاعات متقابل را توسط مقادیر آنتروپی بدست آوریم. تنها لازم است که مقادیر این دو آنتروپی را تخمین بزنیم. با محاسباتی بر روی پایگاه داده ی در اختیار مقادیر زیر برای آنتروپی و در نتیجه اطلاعات متقابل حاصل می شود.
\begin{eqnarray}
\mathcal{H}(I_{i,j+1}-I_{i,j})=4.6757\qquad\mathcal{H}(I_{i,j+1}|I_{i,j})=4.5868\\
\Longrightarrow\mathcal{I}(I_{i,j+1}-I_{i,j},I_{i,j})=8.89\times 10^{-2} 
\end{eqnarray}
با توجه به نتیجه ی بدست آمده می توان این نتیجه را گرفت که مقدار $ I_{i,j+1}-I_{i,j} $ و $ I_{i,j} $ مستقل از همدیگر هستند.  بحث بیان شده به ما کمک می کند تا به جای استفاده از مقادیر حقیقی پیکسل ها از اختلاف آن ها استفاده کنیم. زیرا همان طور که دیدید اختلاف پیکسل ها مستقل از مقدار خود پیکسل ها است. برای ادامه ی راه فرض کنید که ناحیه ی تغییرات تفاوت پیکسل ها در بازه ی $ [-T,T] $ است. 

\subsubsection{
 ویژگی های {\lr{SPAM}}}
 \index{SPAM}\index{احتمال انتقال}
در ادامه مدل {\inpdic{پیکسل های تفریقی همسایه}{Subtractive Pixel Adjacency Matrix}} (\lr{SPAM}) را بیان می کنیم. در ابتدا {\inpdic{احتمال انتقال}{Transition Probability}} در طول هشت جهت محاسبه می شود. اختلاف و احتمال انتقال هر دو در یک جهت می بایست محاسبه شوند. در ادامه ی مقاله تنها این کار برای حهت افقی انجام شده است، سایر جهت ها به روشی مشابه بدست می آید.
 $\{ \leftarrow ,\rightarrow ,\downarrow ,\uparrow ,\searrow ,\swarrow ,\nwarrow ,\nearrow  \} $
محاسبه ی این ویژگی با محاسبه ی آرایه ی اختلاف ($ D $) آغاز می شود. برای جهت افقی داریم.
\begin{equation}
\label{DiffMat}
D^{\rightarrow }_{i,j} = I_{i,j}-I_{i,j+1}
\end{equation}
مدل مرتبه اول \lr{SPAM} ماتریس اختلاف $ D $ را با مارکف مرتبه ی اول مدل می کنیم. 
\begin{equation}
\label{MEq}
M^{\rightarrow }_{u,v}=Pr(D^{\rightarrow }_{i,j+1}=u|D^{\rightarrow }_{i,j}=v)
\end{equation}
اگر مقدار $ Pr(D^{\rightarrow }_{i,j}=v)=0 $ آن گاه مقدار رابطه ی \ref{MEq} نیز برابر صفر خواهد شد. \lr{SPAM} مرتبه ی دوم را نیز با مارکف مرتبه ی دو به صورت زیر مدل می کنیم.
\begin{equation}
M^{\rightarrow }_{u,v,w}=Pr(D^{\rightarrow }_{i,j+2}=u|D^{\rightarrow }_{i,j+1}=v,D^{\rightarrow }_{i,j}=w)
\end{equation}
بعد از بدست آوردن مقادیر $ M $ می توان از آن ها بر طبق روابط زیر میانگین گیری کرد.
\begin{eqnarray}
F_{1,\ldots ,k}=\frac{1}{4}[M^{\rightarrow}+M^{\leftarrow}+M^{\downarrow}+M^{\uparrow}]\\
F_{k+1,\ldots ,2k}=\frac{1}{4}[M^{\searrow}+M^{\nwarrow}+M^{\swarrow}+M^{\nearrow}]
\end{eqnarray}
مقدار $ K=(2T+1)^{2} $ برای ویژگی های مرتبه ی اول و $ k=(2T+1)^{3} $ برای ویژگی های مرتبه ی دوم. به طور خلاصه برای بدست آوردن ویژگی یک تصویر دیجیتال با روش یاد شده، موارد زیر را بر میشماریم.
\begin{dingautolist}{182}
\item
بدست آوردن ماتریس اختلاف $ D $ با استفاده از رابطه ی \ref{DiffMat}.
\item
تشکیل زنجیره ی مارکف. 
\end{dingautolist}
\index{آشکارساز لبه}
پیچیدگی این مدل وابسته به مرتبه ی مدل مارکف و مقدار بازه ی تغییرات $ T $ است. محاسبه ی ماتریس اختلاف را می توان به مانند یک فیلتر {\inpdic{بالاگذر}{High Pass}} در نظر گرفت. هسته ی این فیلتر برابر با $ [-1,+1] $ است. در واقع فیلتر فوق یک {\inpdic{آشکارساز لبه}{Edge Detector}} ساده است. فیلتر فوق محتوی تصویر را مخفی می کند، و سعی در آشکارسازی نویز حاصل از متن پنهان شده دارد. ایده ی استفاده از فیلتر برای بهبود \lr{SNR} در تحلیل پنهان نگاری امروزه کاربرد فراوانی پیدا کرده است. 

\subsection{نهان کاوی با استفاده از توزیع اختلاف پیکسل ها}
\index{توزیع اختلاف پیکسل ها}
در این قسمت، در مورد مقاله ی آقای \lr{Zhang} ر سال 2010 را بیان خواهیم کرد \cite{Zhang2010_Detection}. 

\subsubsection{توزیع آماری اختلاف پیکسل ها}

تصویر دیجیتالی $ I $ را با ابعاد $ M\times N$ در نظر می گیریم. فرض کنید $ I_{ij} $ نشان دهنده ی سطح روشنایی تصویر در مکان $ (ij) $ باشد. هر پیکسل در اطراف خود با 8 پیکسل همسایه است. این همسایه ها را با $ I_{\sharp ij}^{*} $ می توان نشان داد.
{  $ \sharp $ } و {$ * $} می تواند مقدار $ + $، $ - $ و یا تهی را به خود بگیرد. 

تابع توزیع اختلاف بین پیسکل ها را می توان در هر 8 جهت ممکن محاسبه کرد. برای مثال برای حالت افقی داریم.
\begin{equation}
p_{o+}(x)=p(x=I_{ij}-I_{ij+})
\end{equation}
در چهار جهت اصلی این کار را انجام می دهیم. سپس از این چهار جهت به صورت زیر میانگین می گیریم.
\begin{equation}
p(x)=\frac{1}{4}(p_{0+}(x)+p_{0-}(x)+p_{+0}(x)+p_{-0}(x))
\end{equation}
برای آشنایی بهتر با این مفهوم و ایده ای که از آن می خواهیم استفاده کنیم، دو تصویر \lr{Lena} و \lr{Man} را درنظر بگیرید. تابع توزیع میانگین اختلاف پیکسل ها را برای این دو تصویر در شکل \ref{diffFig} آمده است. خطوط ممتد این توزیع را برای تصاویر پوش نشان می دهد، و خطوط به صورت خط تیره ای همین توزیع را برای تصاویر پنهان نگاری شده نشان می دهد.
\begin{figure}[!htbp]
\centerline{\includegraphics[width=12cm]{Chapters/images/fig4}}
\caption{تابع توزیع جهتی اختلاف پیکسل ها}
\label{diffFig}
\end{figure}
با استفاده از تست آماری مثل تست \lr{Kolomogrov-Smirnov} برای یافتن بهترین تخمین می توان دریافت که توزیع تابع اختلاف پیکسل ها را می توان با توزیع لاپلاس به صورت زیر تخمین زد.
\begin{equation}
p(x)=\frac{\mu}{2}exp\lbrace -\mu |x|\rbrace
\end{equation}
که $ \mu $ پارامتر {\inpdic{مقیاس}{Scale}} است. 
\subsubsection{روش نهان کاوی }
با توجه به خواص طرح پنهان نگاری تطابق \lr{LSB}، می دانیم که اگر زمانی که پیام با نرخ {$ 100\% $} در تصویر پوش پنهان شود، رابطه ی بین توزیع تصویر پوش و پنهان نگاری شده به صورت زیر است.
\begin{equation}
\label{conv}
p_{s}(x)=p_{c}(x)* \lbrace \frac{1}{16},\frac{1}{4},\frac{3}{8},\frac{1}{4},\frac{1}{16}\rbrace
\end{equation}
\index{عملگر کانولوشن}
در رابطه ی  {\ref{conv}}،  $ * $ نمایشگر عملگر کانولوشن است. همان طور که در شکل \ref{diffFig} می توان مشاهده کرد، تصویر بعد از پنهان نگاری کردن در آن یک پیک در صفر می زند. در نقطه ی صفر بالاترین تفاوت بین سیگنال پنهان نگاری شده و پاک رخ می دهد. برای سیگنال هایی که $ 100\% $ در آن پنهان نگاری انجام شده است، می توانیم رابطه ی زیر را بنویسیم.
\begin{equation}
\label{estimate01}
\log p_{c}(0)-\log p_{c}(1)>\log p_{s}(0)-\log p_{s}(1)
\end{equation}
اگر ما بتوانیم تخمینی از  $ p(0) $، فرکانس مقدار صفر برای اختلاف پیکسل ها، از تصویر مورد آزمایش بدست آوریم، آن گاه تمایز بین تصویر پاک و پنهان نگاری شده آسان خواهد شد. برای سیگنال پوش که در آن پیامی مخفی نشده است، مقدار $ p(0) $ به مقدار تخمین زده شده نزدیک است، ولی برای تصاویر پنهان نگاری شده این مقدار از مقدار تخمینی دور است. 

مشکل کار این جا است که تخمین مقدار $ p(0) $ کار آسانی نیست. این مساله  می تواند با فرضیات زیر آسانتر شود.
\begin{itemize}
\defpa\index{توزیع لاپلاس}
اختلاف بین پیکسل های همسایه از توزیع لاپلاس پیروی می کند.
\defpa
فرکانس مقادیر غیر صفر برای اختلاف پیکسل ها بعد از پنهان نگاری بدون تغییر باقی می ماند.
\end{itemize}
با توجه به فرض اول که اختلاف پیکسل ها از توزیع لاپلاس پیروی می کند، $ \log p(x) $ تابعی خطی از $ x $ است؛ لذا می توان آن را با یک خط تخمین زد. با استفاده از دو نقطه از این خط می توان معادله آن را بدست آورد.
\begin{equation}
\label{line}
\log p(x)=\frac{\log p(a)-\log p(b)}{a-b}(x-b)+\log p(b)
\end{equation}
همان طور که به یاد دارید هدف ما تخمین مقدار $ p(0) $ بود. با توجه به \ref{line} این مقدار به صورت زیر محاسبه می شود.
\begin{equation}
\label{lineEstimate}
\log \hat{p}(0)=2\times \log p(1)-\log p(2)
\end{equation}
با توجه به خواص توزیع لاپلاس در حوالی صفر، داریم $ p(x)=p(-x) $. با توجه به این مطلب می توانیم به جای $ \log p(x) $ بنویسیم:
$\frac{1}{2}(\log p(x)+\log p(-x))  $
اما داستان به همین جا خاتمه نمی یابد. رابطه ی \ref{lineEstimate} به دلایل زیر کار تخمین را با مشکل مواجه می سازد.
\begin{itemize}
\nfeat
بعد از پنهان سازی پیام، مقادیر $ p(1) $ و $ p(2) $ تغییر پیدا می کند. 
\nfeat
تخمین خطی یاد شده زمانی است که اختلاف پیکسل ها از توزیع لاپلاس پیروی کند، ولی همه ی تصاویر این گونه نیستند. 
\end{itemize}
برای تصویر پوش، انتظار داریم که مقدار $ \log \hat{p}(0) $ برابر با مقدار $ \log p(0) $ است، ولی برای تصویر پنهان نگاری شده این مقدار تخمینی بزرگتر باشد از مقدار حقیقی.
\begin{equation}
\label{estimate0}
\log\hat{p}_{c}(0)-\log p_{c}(0)<\log\hat{p}_{s}(0)-\log p_{s}(0)
\end{equation}
با در نظر گرفتن روابط \ref{estimate01}و \ref{estimate0}  می توان داشت.
\begin{equation}
\frac{\log\hat{p}_{c}(0)-\log p_{c}(0)}{\log p_{c}(0)-\log p_{c}(1)}<
\frac{\log\hat{p}_{s}(0)-\log p_{s}(0)}{\log p_{s}(0)-\log p_{s}(1)}
\end{equation}

\subsubsection{نهان کاوی با استفاده از توزیع شرطی اختلاف پیکسل ها}
در این قسمت سعی می شود، روش تحلیل یاد شده در بخش قبل را بهبود ببخشیم. انتظار داریم که مقدار $ p(0) $ از مقدار $ p(1) $ برای تصاویر پوش، دور باشد. توزیع شرطی اختلاف پیکسل ها را به صورت زیر در نظر بگیرید.
\begin{equation}
\label{conditionalEst}
p_{0+}^{1}(x)=p(x=I_{ij}-I_{ij+}|I_{ij}-I_{ij+}^{+}=0,I_{ij}-I_{ij}^{-}=0)
\end{equation}
به همین ترتیب می توان در سه جهت دیگر نیز توزیع شرطی را محاسبه کرد، و در نهایت از آن ها میانگین گیری می کنیم. 
\begin{equation}
p^{1}(x)=\frac{1}{4}(p_{0+}^{1}(x)+p_{0-}^{1}(x)+p_{+0}^{1}(x)+p_{-0}^{1}(x))
\end{equation}
اکنون تابع تشخیص را می توان به صورت زیر نوشت.
\begin{equation}
f^{1}(I)=\frac{\log p^{1}(1)-\log p^{1}(2)}{\log p^{1}(0)-\log p^{1}(1)}-1
\end{equation}
شکل \ref{figure5} تابع توزیع شرطی اختلاف پیکسل ها برای دو تصویر \lr{Lena} و \lr{Man} نشان می دهد. 

\begin{figure}[!htbp]
\centerline{\includegraphics[width=12cm]{Chapters/images/fig5}}
\caption{توزیع شرطی اختلاف پیکسل ها }
\label{figure5}
\end{figure}

\subsection{تفاضل تصویر در حوزه ی فرکانس}
\index{تشخیص الگو}
در این قسمت به بررسی  مقاله آقای \lr{Deng} در سال 2009 می پردازیم \cite{Deng2009_Universal}. تحلیل عمومی پنهان نگاری را می توان به عنوان یک مساله {\inpdic{تشخیص الگو}{Pattern recognition}} بین دو کلاس از تصاویر مطرح کرد. به دلیل این که طبیعت سیگنال پوش با سیگنال پنهان نگاری شده متفاوت است، به دست آوردن ویژگی از تصاویر امر مهم و حیاتی است. یک ویژگی خوب،  ویژگی است که حساس به تغییر و اصلاح باشد، و عدم حساس به محتوی تصویر. 
\subsubsection{هیستوگرام تفاضل تصویر}
تصاویر طبیعی تصاویر پیوسته ای هستند، که بین پیکسل های همسایه در آن ها همبستگی زیادی وجود دارد. پنهان سازی اطلاعات باعث می شود که همبستگی بین پیکسل ها از بین برود. فرض کنید که تصویر خاکستری را با $ I $ نشان دهیم. ابعاد این تصویر را با $ M\times N $ نمایش می دهیم. تفاضل تصویر را می توان در سه جهت عمودی، افقی و قطری به صورت زیر محاسبه می کنیم.
\begin{eqnarray}
I'_{h}(i,j)&=&I(i,j)-I(i,j+1)\nonumber\\
I'_{v}(i,j)&=&I(i,j)-I(i+1,j)\nonumber\\
I'_{d}(i,j)&=&I(i,j)-I(i+1,j+1)
\end{eqnarray}
\index{توزیع گاوسی تعمیم یافته}
هیستوگرام تفاضلی تصویر با هیستوگرام گیری از تفاضل تصویر حاصل می شود. باور بر این است که هیستوگرام تفاضلی تصویر را می توان با {\inpdic{توزیع گاوسی تعمیم یافته}{Generalized Gaussian Distribution}} توصیف کرد. 
\begin{equation}
p_{v,\beta }(x)=\frac{v}{2\beta\Gamma (1/v)}\exp \lbrace -(\frac{|x|}{\beta } )^{v} \rbrace 
\end{equation}
\index{پارامتر شکل}
پارامتر $ v $ را اصطلاحا {\inpdic{پارامتر شکل}{Shape factor}} می نامیم. 

 
\subsection{نهان کاوی با استفاده از آنتروپی و نویز تصویر }
\index{آنتروپی}
از دیدگاه تئوری اطلاعات محتوی یک تصویر را می توان در دو دسته ی کلی جای داد. این دو دسته را می توان به صورت زیر بیان کرد.
\begin{equation}
ClearImage = Information+Redundant Data
\end{equation}
نهان نگاری را می توان به نوعی نویزی دانست، که در سیگنال ما اثر خواهد کرد. بررسی تاثیر نهان­نگاری بر روی مقدار اطلاعات حمل­شده توسط سیگنال كه از طريق آناليز آنتروپیك امكان­پذير مي­نمايد مسیر جدیدی را پیش روی ما می گذارد. در زمینه بررسی تاثیر نویز بر روی آنتروپی کارهای معدودي در شاخه های دیگر مخابرات انجام گرفته است \cite{Narayanan2001}. 

\index{نسخه آنتروپیک}
باور ما بر این است که پنهان سازی پیام موجب تغییر آنتروپی سیگنال می شود. یافتن یک نسخه ی آنتروپیک از سیگنال کمک کار ما در یافتن این تغییر خواهد بود. نسخه­ی آنتروپیک سیگنال را می توان به صورت مستقیم از حوزه ی مکان و یا حوزه های دیگر مانند حوزه ی تبدیل موجک و ... استخراج کرد. تغییرات نسخه ی آنتروپیک سیگنال با پنهان سازی پیام را، می بایست در تغییر توزیع و مقدار مقادیر ویژه و تکین یافت.

\index{تخمین نویز}
بر روی این زمینه بسیار کم کار شده است. یکی از ایده های مورد استفاده در این زمینه تخمین نویز سیگنال است. آقای {\lr{Smith}} در {\cite{Smith2007a}} از این ایده استفاده نموده است. او با مدل سازی سیگنال تصویر با توزیع گاوسی و پواسن و تخمین نویز سیگنال توانسته است یک روش نهان کاوی ارایه دهد. روشی که در تخمین نویز در این مقاله استفاده شده است، و همچنین نحوه استفاده از آن، در فصل های بعدی توضیح به طور کامل توضیح داده خواهد شد. 

\subsection{نهان کاوی با استفاده از مقادیر تکین}
\index{مقادیر تکین}
در مورد استفاده از مقادیر تکین در امر نهان کاوی تاکنون بسیار کم کار شده است. یکی از مهم ترین کارها در این زمینه کارهای آقای \lr{Gul} در {\cite{Gul2010a,Gu1l2009_NOVEL}} است. ایشان روشی را بر مبنای تحلیل \lr{SVD} به صورت قالبی، و با استفاده از یک تابع لگاریتمی ارایه داده اند. روش ارایه شده یک روش کلی و کور برای تصاویر است. در این روش از تحلیل مقادیر تکین برای نهان کاوی استفاده می شود، و با استفاده از فیلتر وینر استقلال محتوایی تصویر استخراج می شود. 

آقای {\lr{Gul}} ابتدا تصویر رسیده در سمت گیرنده را قالب بندی می کند. ابعاد قالب می تواند بین 3 تا 27 باشد، لذا بدین طریق ما می توانیم به 24 شیوه تصاویر رسیده را قالب بندی کنیم. تابع همپوشانی قالب ها به صورت زیر تعریف می شود.
\begin{equation}
Block = 
\begin{cases}
No overlapping, &\text{ $W< 8$}\\*[5pt]
50\% over lapping, &\text{ $8\leqslant W\leqslant 13$}\\*[5pt]
100\% over lapping, &\text{ $W>13$}
\end{cases}
\end{equation}
بعد از قالب بندی تصویر، مقادیر تکین هر یک از قالب ها را محاسبه می نماییم. 
\begin{equation}
SVD(subblock_{j})\longrightarrow Sv_{j}=[\sigma _{1j},\sigma _{2j},\ldots ,\sigma _{Wj}]
\end{equation}
در نهایت با اعمال یک تابع لگاریتمی به مقادیر تکین بدست آمده و میانگین گیری یک ویژگی برای تصویر حاصل می شود.
\begin{eqnarray}
SVB_{j}&=&\Sigma _{i=1}^{w}\log (\sigma _{ij}^{-1} )\\
F_{W}&=&\frac{1}{TW}\Sigma _{j=1}^{TW}SVB_{j}
\end{eqnarray}
برای هر یک از 24 نوع قالب بندی این ویژگی را بدست می آوریم. این 24 ویژگی را اکنون می توانیم برای امر نهان کاوی استفاده نماییم. 

%%%%%%%%%%%%%%%%%%%%%%%%%%%%%%%%%%%%%%%%%%%%%%%%%%%%
%%%%%%%%%%%%%%%%%%%%%%%%%%%%%%%%%%%%%%%%%%%%%%%%%%%%
%%%%%%%%%%%%%%%%%%%%%%%%%%%%%%%%%%%%%%%%%%%%%%%%%%%%
