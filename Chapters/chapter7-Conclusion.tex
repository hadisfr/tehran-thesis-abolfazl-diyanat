\chapter{نتیجه‌گیری و کارهای آینده}
محور کار ما در این پایان‌نامه بر روی دو مسئله نشان‌گذاری و نهان‌کاوی متمرکز بود. در نشان‌گذاری به سراغ فایل‌های ویدئویی {\lr{AVI}} و در نهان‌کاوی به سراغ تصاویر خاکستری رفتیم. در این فصل به صورت خلاصه حاصل کار را بیان خواهیم‌نمود. 
\section{نشان‌گذاری}
در نشان‌گذاری به ارایه دو طرح جدید و نوین مبادرت ورزیدیم. در  فصل سه یک روش نشان‌گذاری نیمه‌کور، برای فایل‌های ویدئویی {\lr{AVI}} ارایه دادیم.  روش نشان‌گذاری نیمه‌کور پیشنهادی بر اساس طرح‌های پیشنهادی در مراجع \cite{Akhaee2010}
و \cite{Akhaee2009a}
ارایه شد. 

اولین گام ما در طرح پیشنهادی، قالب‌بندی و انتخاب قالب‌های مناسب برای درج نشانه می‌باشد. ما برای این کار معیار آنتروپی قالب‌ها را برگزیدیم، و بیان شد که مکان‌های با آنتروپی بالا مناسب‌ترین مکان‌ها برای درج نشانه می‌باشند. سپس نشان‌گذاری به صورت ضربی در ضرایب فرکانس پایین تبدیل موجک سیگنال، انجام پذیرفت، و بعد از درج نشانه و گذر سیگنال از یک کانال با نویز گاوسی، در سمت گیرنده، آشکارساز بیشینه شباهت بر مبنای کمینه‌کردن خطای گیرنده را طراحی کردیم.  

برای طراحی آشکارساز فرض کردیم که ضرایب تبدیل موجک ویدئو را می‌توان با توزیع گاوسی مدل نمود. سپس با استفاده از این حقیقت، احتمال درج به ازای بیت‌های صفر و یک محاسبه شد، و در رابطه آشکارساز بیشینه شباهت قرار داده شد. گیرنده برای بدست آوردن نشانه از رابطه بیشینه شباهت نیازمند میانگین و واریانس هر یک از قالب‌هایی دارد، که عملیات درج در آن انجام شده‌است. در ادامه نیز آشکارساز پیشنهادی را در شرایط خاص (محیط کم نویز و مقادیر {$\alpha$} کوچک) نیز بدست آوردیم. در آن‌جا بیان شد، که در محیط یادشده، کافی است تنها یک پارامتر (مقدار انرژی ضرایب) را برای گیرنده ارسال کنیم. 

در انتها نیز روش پیشنهادی را مورد ارزیابی به صورت تئوری و عملی (توسط شبیه‌سازی) قرار دادیم. 
به منظور ارزیابی تئوریکی روش پیشنهادی،  احتمال خطای گیرنده را به صورت ریاضی بدست آوردیم. در نهایت با استفاده از شبیه‌سازی، کارایی روش پیشنهادی را در شرایط گوناگون با اعمال حملات مختلف به طرح نشان‌گذاری ارایه شده، مورد ارزیابی قرار دادیم. با توجه به نتایج شبیه‌سازی می‌توان به موفقیت نسبی طرح نشان‌گذاری پیشنهادی پی برد. 

اما کار ما به همین جا متوقف نشد. مشکلی که در طرح نیمه‌کور وجود داشت، این بود که فرستنده لازم بود تا اطلاعاتی را از سیگنال پوشش برای گیرنده ارسال کند. گیرنده بدون داشتن این اطلاعات توانایی آشکارسازی نشانه را نداشت. برای رفع این مشکل به سراغ ارایه یک روش کور  رفتیم. 

لذا در فصل چهار، روشی را پیشنهاد دادیم، که توسط آن قادر بودیم، پیامی را به صورت کور در یک سیگنال ویدئویی درج کنیم. ابتدا سعی در مدل سازی تابع توزیع، حاصل تقسیم دو متغیر تصادفی گاوسی کردیم. سپس  این توزیع را با یک توزیع گاوسی تقریب زده، و پارامترهای این توزیع را بدست آوردیم. پس از بیان این پیش زمینه، به سراغ روش پیشنهادی رفتیم. برخی از قسمت‌های طرح پیشنهادی مانند انتخاب مکان‌های مناسب برای درج نشانه، همانند فصل سه می‌باشد. روش نشان‌گذاری همان روش ضربی است، اما با یک هوشمندی خاصی که برای کاستن اثرات دیداری درج نشانه، بر آن اعمال شده بود.

در طرح پیشنهادی فرستنده برخی از ضرایب تبدیل موجک را برای درج نشانه مورد استفاده قرار نمی‌دهد. برای مثال در طرح ما فرستنده عملیات نشان‌گذاری را در ضرایب زوج انجام می‌دهد، اما کاری به ضرایب فرد سیگنال ندارد. در واقع  فرستنده تنها در نصف ضرایب تبدیل موجک، نشانه را درج می‌کند؛ بدین‌سان گیرنده می‌تواند اطلاعاتی را که برای آشکارسازی نیاز دارد، از این نصفه ضرایب دست نخورده استخراج نماید. به دلیل همین عمل روابطی که برای آشکارساز بیشینه شباهت در حالت کور بدست می‌آید، بسیار متفاوت با روابط حالت نیمه‌کور می‌باشد. 
در نهایت نیز در بخش‌های {\ref{PerformanceBlind}} و {\ref{SimulationBlind}} به ترتیب به بررسی عملکرد طرح پیشنهادی به صورت تئوری و توسط شبیه سازی مبادرت ورزیدیم. در شبیه‌سازی‌های انجام گرفته، طرح پیشنهادی را در معرض تعداد متعددی از حملات قرار دادیم. 

پر واضح است که کارهای صورت گرفته هنوز جای کار فراوانی دارد. برای مثال می‌توان از موارد زیر نام برد:
\begin{itemize}
\incl
استفاده از تبدیل‌های دیگری مانند {\lr{Contourlet }} و یا {\lr{Rigid}} به جای تبدیل موجک.
\incl
همان‌گونه که مطالعه نمودید، در نشان‌گذاری نیمه‌کور ضرایب تبدیل موجک را مستقل از هم در نظر گرفتیم، اما این فرض به خصوص در فایل‌های ویدئویی فرض صحیحی نمی‌باشد.
\incl
طراحی آشکارساز بهینه برای دیگر حملات. همان‌طور که مشاهده نمودید، آشکارساز بهینه تنها در حضور نویز سفید طراحی شده بود. یک کار بهتر نیز این می‌باشد، که گیرنده توسط یک روش معین، نوع حمله را تشخیص دهد، سپس بر حسب نوع حمله وارد شده، آشکارساز مرتبط با آن فراخوانی شود. 
\incl
یکی از مهم‌ترین ضعف‌های روش پیشنهادی این است که در برابر حملات هندسی بسیار ضعیف می‌باشد. اگر بتوان به نوعی این خلا را نیز پوشش داد، می‌توان به یک نشان‌گذاری بسیار خوب برای سیگنال‌های ویدئویی دست یافت. 
\end{itemize}


\section{نهان‌کاوی}
در اين پایان‌نامه سعي کردیم، تا روشي در نهان‌کاوي، براي نهان‌نگاري به روش {\lr{LSB}} در تصاوير خاکستري با استفاده از تحليل مقادير تکين ارايه دهيم.  باور ما بر این است که مقادیر تکین به نوعی نمایانگر اطلاعات نهفته در سیگنال می‌باشد. مقادیر تکین با قدرت کمتر به نوعی در برگیرنده اطلاعات مربوط به نویز سیگنال می‌باشد. مقداری از این نویز مربوط به خود سیگنال و مقداری دیگر مربوط به پیام اضافه شده به سیگنال می‌باشد. اگر بتوان به نوعی مقدار نویز خود سیگنال را تخمین زد، نویز باقیمانده در حقیقت ناشی از پیام اضافه شده به سیگنال می‌باشد. یکی از ویژگی‌های مهم روش پیشنهادی این است که با اعمال آن می‌توان تخمینی از مقدار پیام نهان‌نگاری شده را نیز بدست آورد. 

 در ابتداي کار پارامتري به نام {$DS$} را تعريف مي کنيم، به طوري که مستقل از محتوي تصوير و حساس نسبت به افزايش نرخ درج پيام در تصوير باشد. با محاسبه {$DS$} به ازاي نرخ‌هاي مختلف درج {\lr{LSB}}، با يک منحني مواجه مي‌شويم. هدف ما اين است که براي هر تصوير مشکوک منحني {$DS$} را تخمين بزنيم. با تخمين اين منحني، مي توان تقريبي از نرخ درج را نيز بدست آورد. در اين کار با يک منحني درجه دو، منحني {$DS$} را تقريب مي زنيم. در ادامه با استفاده از يک ماشين يادگيري ({\lr{SVM}}) و ويژگي‌هاي بدست آمده از منحني {$DS$} و ويژگي هاي ديگري نظير تخمين نويز سطوح مختلف تصوير وجود و يا عدم وجود پيام پنهان‌شده در تصوير را بيان مي‌کنيم. شبيه سازي انجام گرفته بر روي 6000 تصوير، نشان‌دهنده کارايي مناسب روش ارايه شده، در مقايسه با سه روش مطرح در اين زمينه است. کارايي روش با نمودار خطا و {\lr{ROC}} نشان داده شده است. 

پر واضح است که کارهای صورت گرفته هنوز جای کار فراوانی دارد. برای مثال می‌توان از موارد زیر نام برد:
\begin{itemize}
\incl
شاید نقطه اتکا روش پیشنهادی در نهان‌کاوی، تخمین منحنی {$DS$} باشد. هرچه این تقریب دقیق‌تر باشد، می‌توان امیدواری بیشتری به موفقیت روش داشت. 
\incl
رفتن به سراغ ارایه یک روش کور برای بحث نهان کاوی با استفاده از ایده های ارایه شده.
\end{itemize}










